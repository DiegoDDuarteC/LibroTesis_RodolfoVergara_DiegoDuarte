 \documentclass[final,fmstyle]{./util/fpunathesis}

% paquetes recomendados
\usepackage{amsmath,amsthm}
\usepackage{textcomp}
\usepackage[T1]{fontenc}
\usepackage[spanish]{babel}
\usepackage[utf8]{inputenc}
\usepackage{csquotes}
\usepackage{enumerate}
\usepackage{float}
\usepackage{tikz}
\usetikzlibrary{shapes.geometric, arrows.meta, positioning, calc}
%\usepackage{enumitem}
\usepackage[shortlabels]{enumitem} %para enum con letras
\usepackage{caption}
\usepackage{subcaption}
\usepackage[style=numeric, uniquename=full, sorting=none,backend=biber, natbib=true]{biblatex}
\usepackage{listings}
% para la lista de simbolos
\usepackage{array} %for vertical thick lines in tables
\usepackage{multirow} %multirow tables
\usepackage{nicefrac} %for fractions like 1/4
%\usepackage[table]{xcolor} %para colorear las tablas
\usepackage{makecell}
\usepackage{tabto}    
\usepackage{xcolor}
%referencias
\addbibresource{referencias.bib}
%se importan las configuraciones custmizdas realizadas.
%\include{./util/custom}

% datos de la tesis y el/los autor/es
\title{Estrategia de disparo para el proceso de desfragmentación en redes ópticas elásticas multicore, utilizando técnicas de aprendizaje automático}
\author{RODOLFO SEBASTIÁN VERGARA FERREIRA \and DIEGO DANIEL DUARTE CENTURIÓN}
\degree{Informática}

\advisor{Phd. Msc. Ing.}{ENRIQUE DAVALOS}

%\newtheorem{definicion}{Definición}

\logosource{./graphics/logo.png}


\begin{document}
\renewcommand{\listtablename}{Lista de Tablas}
\renewcommand{\tablename}{Tabla}
\maketitle     % esto hace las portadas

% Agradecimientos
%\include{book/agradecimientos}

% los siguientes comandos producen 'indices.

% Tabla de contenidos
\tableofcontents
% Lista de figuras, incluye en la lista todas las figuras de forma automática
\listoffigures
% Lista de tablas, incluye en la lista todas las tablas de forma automática
\listoftables
% Lista de algoritmos, incluye en la lista todas los algoritmos de forma automática
%\listofalgorithms
%\include{acronimos}
\newpage
\chapter*{Lista de Símbolos\hfill}
\addcontentsline{toc}{chapter}{Lista de Símbolos}
\begin{tabbing}

\color{white}Zero \=\color{white}One \=\color{white}Twoooooo \=\color{white}Thre\\

\textbf{Parámetros de red y espectro}\\
\textit{N} \>\>\> Cantidad de \textit{FS} en un enlace.\\
\textit{\(FS_{i}\)} \>\>\> \textit{Frecuency Slot} de índice \textit{i}.\\
\textit{\(FS_{i+1}\)} \>\>\> \textit{Frecuency Slot} de índice \(\textit{i} + 1\).\\
|\textit{E}| \>\>\> Cantidad de enlaces de la red.\\
\textit{C} \>\>\> Número de núcleos por enlace.\\
\textit{L} \>\>\> Longitud física del enlace óptico.\\
\textit{\(S_{free}\)} \>\>\> Cantidad de \textit{FS} libres en un enlace o núcleo.\\
\textit{\(S^{occupied}\)} \>\>\> Cantidad de \textit{FS} ocupados.\\
\textit{\(S^{small}\)} \>\>\> Suma de \textit{FS} libres en bloques menores a 5 slots.\\
\textit{Bloques} \>\>\> Cantidad de bloques de ranuras libres en un enlace. \\
\textit{B} \>\>\> Cantidad de bloques libres.\\
\textit{G} \>\>\> Cantidad total de gaps.\\
\textit{\(g_{i}\)} \>\>\> Tamaño del gap libre \textit{i}.\\
\textit{\(s_{min}\)} \>\>\> Índice del primer \textit{FS} ocupado.\\
\textit{\(s_{max}\)} \>\>\> Índice del último \textit{FS} ocupado.\\
\\
\textbf{Parámetros de fibra multinúcleo}\\
\textit{\(\Lambda_{i,j}\)} \>\>\> Core pitch - Distancia entre los núcleos \textit{i} y \textit{j}.\\
\textit{\(N_{i}\)} \>\>\> Cantidad de núcleos adyacentes al núcleo \textit{i}.\\
\textit{k} \>\>\> Coeficiente de acoplamiento.\\
\textit{r} \>\>\> Radio de curvatura.\\
\textit{\(\beta\)} \>\>\> Constante de propagación.\\
\textit{\(h_{i,j}\)} \>\>\> XT por unidad de longitud entre núcleos \textit{i} y \textit{j}.\\
\textit{\(XT_{i}\)} \>\>\> Crosstalk total que impacta al núcleo \textit{i}.\\
\textit{\(XT_{TH}\)} \>\>\> Umbral de crosstalk admisible.\\
\\
\textbf{Métricas de fragmentación}\\
\textit{\(BFR_{link}\)} \>\>\> Relación de Fragmentación de ancho de banda de un enlace.\\
\textit{\(BFR_{link-i}\)} \>\>\> Relación de Fragmentación de ancho de banda del enlace \textit{i}. \\
\textit{\(BFR_{red}\)} \>\>\> Relación de Fragmentación de ancho de banda de la red. \\
\textit{MaxBlock()} \>\>\> Tamaño del mayor bloque de \textit{FS} libres.\\
\\
\textit{\(SHF_{link}\)} \>\>\> Entropía de Shannon del enlace.\\
\textit{\(SHF_{link-i}\)} \>\>\> Entropía de Shannon del enlace \textit{i}. \\
\textit{\(SHF_{red}\)} \>\>\> Entropía de Shannon de la red. \\
\\
\textit{MSI} \>\>\> Índice de slot máximo utilizado.\\
\textit{\(MSI_{link-i}\)} \>\>\> Índice de slot máximo utilizado del enlace \textit{i}. \\
\textit{\(MSI_{red}\)} \>\>\> Índice de slot máximo utilizado de la red. \\
\\
\textit{\(n_{1}, n_{2}\)} \>\>\> Parámetros de tamaños típicos de demanda (Golden Metric).\\
\textit{\(F_{spatial}\)} \>\>\> Factor de peso espacial.\\
\textit{\(D_{active}\)} \>\>\> Número de demandas activas en la red.\\
\\
\textbf{Utilización de red}\\
\textit{Uso} \>\>\> Porcentaje de utilización de la red.\\
\textit{\(U_{core}\)} \>\>\> Utilización de un núcleo.\\
\textit{\(U_{max}\)} \>\>\> Utilización máxima entre todos los núcleos.\\
\textit{\(U_{min}\)} \>\>\> Utilización mínima entre todos los núcleos.\\
\\
\textbf{Métricas de desempeño}\\
\textit{PB} \>\>\> Probabilidad de Bloqueo.\\
\textit{\(PB_{th}\)} \>\>\> Umbral para disparar el proceso de desfragmentación.\\
\textit{BL} \>\>\> Cantidad de Bloqueos.\\
\textit{RC} \>\>\> Cantidad de Reconfiguraciones.\\
\textit{SFP} \>\>\> Número de soluciones en el Frente Pareto.\\
\textit{CP} \>\>\> Cobertura Pareto.\\
\\
\textbf{Parámetros de aprendizaje automático}\\
\textit{M} \>\>\> Número total de iteraciones (Gradient Boosting).\\
\textit{\(h_{m}(x)\)} \>\>\> m-ésimo aprendiz débil.\\
\textit{\(\gamma_{m}\)} \>\>\> Coeficiente de peso asociado.\\
\textit{\(\nu\)} \>\>\> Tasa de aprendizaje.\\
\\
\textbf{Métodos comparados}\\
\textit{MP} \>\>\> Método Propuesto.\\
\textit{MR1} \>\>\> Método de Referencia 1 (Desfragmentación Periódica).\\
\textit{MR2} \>\>\> Método de Referencia 2 (Desfragmentación por Umbral).\\
\textit{SD} \>\>\> Sin Desfragmentación.\\

\end{tabbing}



\mainmatter  % inician los capítulos de la tesis


% incluye aqui los capítulos (un archivo .tex por capitulo)
%\include{book/capitulo-ej/capitulo-ej}
%\newpage
\chapter*{Lista de Símbolos\hfill}
\addcontentsline{toc}{chapter}{Lista de Símbolos}
\begin{tabbing}

\color{white}Zero \=\color{white}One \=\color{white}Twoooooo \=\color{white}Thre\\

\textbf{Parámetros de red y espectro}\\
\textit{N} \>\>\> Cantidad de \textit{FS} en un enlace.\\
\textit{\(FS_{i}\)} \>\>\> \textit{Frecuency Slot} de índice \textit{i}.\\
\textit{\(FS_{i+1}\)} \>\>\> \textit{Frecuency Slot} de índice \(\textit{i} + 1\).\\
|\textit{E}| \>\>\> Cantidad de enlaces de la red.\\
\textit{C} \>\>\> Número de núcleos por enlace.\\
\textit{L} \>\>\> Longitud física del enlace óptico.\\
\textit{\(S_{free}\)} \>\>\> Cantidad de \textit{FS} libres en un enlace o núcleo.\\
\textit{\(S^{occupied}\)} \>\>\> Cantidad de \textit{FS} ocupados.\\
\textit{\(S^{small}\)} \>\>\> Suma de \textit{FS} libres en bloques menores a 5 slots.\\
\textit{Bloques} \>\>\> Cantidad de bloques de ranuras libres en un enlace. \\
\textit{B} \>\>\> Cantidad de bloques libres.\\
\textit{G} \>\>\> Cantidad total de gaps.\\
\textit{\(g_{i}\)} \>\>\> Tamaño del gap libre \textit{i}.\\
\textit{\(s_{min}\)} \>\>\> Índice del primer \textit{FS} ocupado.\\
\textit{\(s_{max}\)} \>\>\> Índice del último \textit{FS} ocupado.\\
\\
\textbf{Parámetros de fibra multinúcleo}\\
\textit{\(\Lambda_{i,j}\)} \>\>\> Core pitch - Distancia entre los núcleos \textit{i} y \textit{j}.\\
\textit{\(N_{i}\)} \>\>\> Cantidad de núcleos adyacentes al núcleo \textit{i}.\\
\textit{k} \>\>\> Coeficiente de acoplamiento.\\
\textit{r} \>\>\> Radio de curvatura.\\
\textit{\(\beta\)} \>\>\> Constante de propagación.\\
\textit{\(h_{i,j}\)} \>\>\> XT por unidad de longitud entre núcleos \textit{i} y \textit{j}.\\
\textit{\(XT_{i}\)} \>\>\> Crosstalk total que impacta al núcleo \textit{i}.\\
\textit{\(XT_{TH}\)} \>\>\> Umbral de crosstalk admisible.\\
\\
\textbf{Métricas de fragmentación}\\
\textit{\(BFR_{link}\)} \>\>\> Relación de Fragmentación de ancho de banda de un enlace.\\
\textit{\(BFR_{link-i}\)} \>\>\> Relación de Fragmentación de ancho de banda del enlace \textit{i}. \\
\textit{\(BFR_{red}\)} \>\>\> Relación de Fragmentación de ancho de banda de la red. \\
\textit{MaxBlock()} \>\>\> Tamaño del mayor bloque de \textit{FS} libres.\\
\\
\textit{\(SHF_{link}\)} \>\>\> Entropía de Shannon del enlace.\\
\textit{\(SHF_{link-i}\)} \>\>\> Entropía de Shannon del enlace \textit{i}. \\
\textit{\(SHF_{red}\)} \>\>\> Entropía de Shannon de la red. \\
\\
\textit{MSI} \>\>\> Índice de slot máximo utilizado.\\
\textit{\(MSI_{link-i}\)} \>\>\> Índice de slot máximo utilizado del enlace \textit{i}. \\
\textit{\(MSI_{red}\)} \>\>\> Índice de slot máximo utilizado de la red. \\
\\
\textit{\(n_{1}, n_{2}\)} \>\>\> Parámetros de tamaños típicos de demanda (Golden Metric).\\
\textit{\(F_{spatial}\)} \>\>\> Factor de peso espacial.\\
\textit{\(D_{active}\)} \>\>\> Número de demandas activas en la red.\\
\\
\textbf{Utilización de red}\\
\textit{Uso} \>\>\> Porcentaje de utilización de la red.\\
\textit{\(U_{core}\)} \>\>\> Utilización de un núcleo.\\
\textit{\(U_{max}\)} \>\>\> Utilización máxima entre todos los núcleos.\\
\textit{\(U_{min}\)} \>\>\> Utilización mínima entre todos los núcleos.\\
\\
\textbf{Métricas de desempeño}\\
\textit{PB} \>\>\> Probabilidad de Bloqueo.\\
\textit{\(PB_{th}\)} \>\>\> Umbral para disparar el proceso de desfragmentación.\\
\textit{BL} \>\>\> Cantidad de Bloqueos.\\
\textit{RC} \>\>\> Cantidad de Reconfiguraciones.\\
\textit{SFP} \>\>\> Número de soluciones en el Frente Pareto.\\
\textit{CP} \>\>\> Cobertura Pareto.\\
\\
\textbf{Parámetros de aprendizaje automático}\\
\textit{M} \>\>\> Número total de iteraciones (Gradient Boosting).\\
\textit{\(h_{m}(x)\)} \>\>\> m-ésimo aprendiz débil.\\
\textit{\(\gamma_{m}\)} \>\>\> Coeficiente de peso asociado.\\
\textit{\(\nu\)} \>\>\> Tasa de aprendizaje.\\
\\
\textbf{Métodos comparados}\\
\textit{MP} \>\>\> Método Propuesto.\\
\textit{MR1} \>\>\> Método de Referencia 1 (Desfragmentación Periódica).\\
\textit{MR2} \>\>\> Método de Referencia 2 (Desfragmentación por Umbral).\\
\textit{SD} \>\>\> Sin Desfragmentación.\\

\end{tabbing}

\chapter{Introducción}
\section{Justificación}

Debido al incremento de la popularidad de internet y del uso de servicios en la nube, tales como \textit{Content Delivery Network} (CDN) y \textit{Video on Demand} (VoD), las demandas de  tasas de bits en las redes han crecido de manera exponencial, lo que obliga a estudiar nuevas y mejores tecnologías relacionadas a la transmisión de datos.

Las  Redes de Multiplexación por División de Longitud de Onda o \textit{Wavelength Division Multiplexing} (WDM), utilizan una grilla fija, de 50 o 100 GHz, dan una gran ventaja logrando velocidades muy superiores
frente a las viejas tecnologías, pero a pesar de esta ventaja señalada, la gruesa granularidad lleva a un
uso ineficiente del espectro, ya que cada demanda es asignada a un canal fijo y estas pueden requerir un
ancho de banda menor al tamaño del canal.

Esta desventaja da lugar a las Redes Elásticas Ópticas o \textit{Elastic Optical Networks} (EON) \cite{jinno2009spectrum}, las cuales surgen como una solución al problema anteriormente citado, ya que estas proporcionan una mayor flexibilidad en la división del espectro y de esa forma lograr que los requerimientos sean asignados de manera más eficiente.

A las redes EON tambien se la conocen como redes de grilla flexible, debido a que las ranuras de frecuencia o FS (\textit{Frequency Slot}) que reemplazan a los ``Canales WDM'', cuentan con una división más flexible. Cada FS tiene un ancho de banda de 12.5 GHz, de esta manera se logra una cantidad más apropiada de FS para satisfacer un requerimiento.
%

Sin embargo, a pesar de las mejoras introducidas por las redes EON, el crecimiento exponencial del tráfico de datos demanda soluciones aún mas avanzadas. En este contexto, surgen las Redes Ópticas Elásticas Multicore o \textit{Elastic Optical Networks with Multicore Fibers} (EON-MCF) y por consecuente \textit{Space Division Multiplexing-Elastic Optical Networks} (SDM-EON) , que incorporan fibras ópticas multinúcleo (MCF), para multiplicar la capacidad de transmisión mediante la explotación de la dimensión espacial, ademas de las dimensiones espectral y temporal ya utilizadas en las redes EON convencionales.
%

Las fibras multinúcleo contienen múltiples núcleos dentro de una única fibra, donde cada núcleo puede transmitir señales de manera independiente. Esta arquitectura permite aumentar significativamente la capacidad de la red sin necesidad de desplegar nuevas fibras, ofreciendo una solución escalable y económicamente viable para satisfacer las crecientes demandas de ancho de banda.
%

Los métodos de ruteo y asignación del espectro y núcleo tienen gran impacto sobre el uso eficiente de los recursos de la red. Los algoritmos RSCA (\textit{Routing, Spectrum and Core Assigment}) se encargan de resolver dicho problema encontrando el camino más apropiado desde el origen hasta el destino, el núcleo a utilizar y las ranuras que utilizará el requerimiento dentro del espectro de los enlaces.
%

Se han propuestos varios algoritmos RSCA con el fin de conseguir la mejor utilización de recursos, estos algoritmos están sujetos a tres principios fundamentales: la restricción de consecutividad del ancho de banda, la restricción de la continuidad del ancho de banda y la restricción de continuidad de núcleo. 
%

La restricción de continuidad espectral establece que se deben utilizar los mismos FS en todo el camino y la restricción de contigüidad dispone que los FS seleccionados para satisfacer la demanda deben ser contiguos. La restricción de continuidad de núcleo especifíca que se debe mantener el mismo núcleo a lo largo de toda la ruta establecida.
%

Adicionalmente, en las redes SDM-EON surge un nuevo fenómeno denominado \textit{Crosstalk} o diafonía entre núcleos \textit{inter-core crosstalk, XT}, que ocurre cuando las señales ópticas de núcleos adyacentes interfieren entre sí, degradando la calidad de la transmisión. Este fenómeno debe ser considerado como una restricción adicional en los algoritmos RSCA para garantizar la calidad del servicio.
%

Debido a las restricciones explicadas y a que las asignaciones de recursos son realizadas de manera dinámica, surge el fenómeno denominado "Fragmentación del Ancho de Banda y del Espacio", este problema es una de las principales dificultades de las redes SDM-EON ya que tiene un impacto directo en el uso eficiente del espectro y de los núcleos disponibles.
%

El fenómeno de la fragmentación espectro-espacial del ancho de banda sucede cuando en los enlaces se encuentran FS disponibles separados por FS que están siendo utilizados por otras conexiones, o cuando existen  núcleos con recursos fragmentados que no pueden ser eficientemente asignados, por lo que estas podrían quedar inutilizables para nuevas conexiones por no poder satisfacer a la demanda debido a las restricciones citadas anteriormente, en consecuencia, la probabilidad de bloqueo \cite{shi2013effect} aumenta considerablemente.
%

Un bloqueo sucede cuando el algoritmo RSCA no puede encontrar núcleos y FS disponibles para una demanda, esto puede deberse a una alta saturación del espectro o de los núcleos, pero también debido al problema mencionado anteriormente, donde existe la cantidad de FS libres que se solicitan, pero sin respetar las restricciones de continuidad y contigüidad, o donde no hay núcleos disponibles que cumplan con las restricciones de crosstalk, es decir el espectro y el espacio se encuentran fragmentados.
%

El problema de la fragmentación de redes SDM-EON es ampliamente estudiado en la literatura actual, para buscar manejarlo se han propuesto soluciones con distintos enfoques.
%

% 

Uno de los enfoques es el llamado \textit{Enfoque proactivo} el cual consiste en ejecutar un proceso de desfragmentación periódicamente o mediante un disparador. Tiene como principal objetivo prevenir futuros bloqueos en la red, este enfoque será el utilizado en este trabajo.
%

El proceso de desfragmentación consiste en la reconfiguración o re-ruteo de un sub-conjunto de conexiones ya establecidas en la red, teniendo como principal objetivo reducir la fragmentación del espectro y la fragmentación espacial mediante la eliminación de bloques de FS libres no contiguos y la registribución eficiente de conexiones entre núcleos. 
%

En el trabajo presentado por Zhang \cite{zhang2014dynamic}, se realizó un análisis del problema de desfragmentación en redes EON, en el cual lo dividen en cuatro subproblemas, los cuales son, (I) ¿Cómo reconfigurar?, (II) ¿Cómo migrar el tráfico?, (III) ¿Cuándo reconfigurar? y (IV) ¿Qué reconfigurar?. Estos subproblemas mantienen su vigencia en el contexto de las redes SDM-EON, con la complejidad adicional de considerar la dimensión espacial.
%

En este trabajo nos centraremos en el tercer subproblema, ¿Cuándo reconfigurar?, ya que considerando el enfoque proactivo para resolver el problema de la fragmentación, encontramos que los procesos de desfragmentación podrían ejecutarse en periodos de tiempo donde no son del todo necesarios, es decir cuando la red se encuentra con una baja fragmentación, provocando desfragmentaciones ineficientes, una cantidad mayor de disrupciones de conexiones y una elevación innecesaria del costo de procesamiento.
%

En los siguientes capítulos presentamos un novedoso modelo de predicción de probabilidades de bloqueo implementado con técnicas de aprendizaje automático o \textit{Machine Learning}, el cual se utiliza como disparador del proceso de desfragmentación pero en este caso redes SDM-EON Multinúcleo, proponiendo de esta manera una solución al sub problema planteado anteriormente. 
%
\section{Objetivos del trabajo}
\subsection{Objetivo General}
Diseñar un modelo de disparo para el proceso de desfragmentación en redes ópticas elásticas multicore basado en probabilidades de bloqueo, utilizando técnicas de aprendizaje automático (\textit{Machine Learning}), con el propósito de maximizar la eficiencia en el uso de los recursos de la red mediante la reducción de reconfiguraciones de conexiones existentes y la minimización de la probabilidad de bloqueo. Este trabajo extiende la propuesta de Enciso y Silva [], originalmente desarrollada para redes EON convencionales, al contexto de redes multicore. 

\subsection{Objetivos Específicos}
\begin{itemize}
    \item Realizar una revisión bibliográfica del estado del arte en técnicas de desfragmentación para redes ópticas elásticas, con énfasis en métodos basados en aprendizaje automático y su aplicación en redes multicore.

    \item Identificar y definir métricas de fragmentación apropiadas para redes ópticas elásticas multicore, considerando las particularidades de la asignación de recursos en múltiples núcleos.
    
    \item Desarrollar e implementar un modelo de aprendizaje automático capaz de predecir probabilidades de bloqueo y determinar momentos óptimos para activar el proceso de desfragmentación en redes EON multicore.
    
    \item Diseñar e implementar una interfaz de integración entre el simulador de redes ópticas elásticas multicore y el modelo de aprendizaje automático entrenado, permitiendo la evaluación en tiempo real del sistema propuesto.
    
    \item Evaluar el desempeño del modelo propuesto mediante simulaciones, comparando sus resultados con técnicas de desfragmentación existentes en términos de probabilidad de bloqueo, número de reconfiguraciones y eficiencia en el uso de recursos espectrales.
\end{itemize}

\section{Organización del libro}
El presente trabajo se encuentra organizado de la siguiente manera:

En el capítulo dos se trata sobre características y conceptos relacionados con las redes EON, su principal
dificultad (la fragmentación del ancho de banda), los diferentes enfoques para manejar la misma y una 
presentación de trabajos relacionados presentes en la literatura científica.

En el capítulo tres se hace una introducción al \textit{Machine learning}, enfocado al aprendizaje supervisado y redes neuronales.

En el capítulo cuatro se presenta el método propuesto para la selección del momento de desfragmentación, 
describiendo todo el proceso que conlleva.

El capítulo cinco se muestra las pruebas experimentales junto a un análisis de los resultados obtenidos.

Por último, el capítulo seis presenta las conclusiones del trabajo y sugerencias para trabajos futuros. 
\chapter{Redes Ópticas Elásticas Multicore y Fragmentación del Ancho de Banda}
Las redes ópticas que se basan en WDM dividen el espectro de cada enlace en canales cuyo ancho de banda se fija de 50 GHz o 100 GHz. Esto debido a que la Unión Internacional de Telecomunicaciones (\textit{ITU-T International Telecommunication Union}) especificó el estándar G.694.1 en el año 2002.
%
Estas redes WDM resultan muy rígidas, y debido a eso es posible que ocurra una utilización ineficiente de la capacidad, provocado por el hecho de que el espacio entre dos canales adyacentes es relativamente grande y si la señal que se transmite utiliza un ancho de banda muy bajo, gran parte del espectro será desperdiciado.
%

Una nueva tecnología denominada Redes Ópticas Elásticas o \textit{Elastic Optical Networks} (EON) y su evolución: Las Redes ópticas Elásticas Multinúcleo o \textit{Multicore Elastic Optical Networks} (MC-EON) el cual no solo dividen el espectro óptico en Ranuras de Frecuencia o \textit{Frequency Slots} (FS) de 12.5 GHz conforme a lo establecido por el estándar definido en ITU-T (G.694.1) en el año 2012, sino que las (MC-EON) introducen un nuevo dominio de multiplexación espacial, al permitir la transmisión simultánea de múltiples señales ópticas en diferentes núcleos dentro de una misma fibra.
%
Esta aproximación multidimensional proporciona una mayor escalabilidad, eficiencia en la asignación del espectro y reducción del consumo energético, posicionando a las MC-EON como una de las tecnologías mas prometedoras para la implementación de redes ópticas de ultra alta capacidad en escenarios de próxima generación.
%

% En la Figura \ref{fig:eonwdm}, se muestra una comparación en la asignación de demandas en ambas tecnologías, observándose un mejor aprovechamiento del espectro óptico para el caso de las redes EON, debido a un menor desperdicio de ancho en banda.

% % \begin{figure}
% %     \centering
% %     \includegraphics[width=0.95\textwidth]{capitulos/img/eonwdm.png}
% %     \caption{Redes WDM v s Redes EON}
% %     \label{fig:eonwdm}
% % \end{figure}


\section{Fragmentación del Ancho de Banda en MC-EON}
Las redes ópticas elásticas multinúcleo (MC-EON) permiten optimizar el uso del ancho de banda necesario para satisfacer una demanda, respetando tres restricciones fundamentales:
%

\begin{itemize}
    \item \textbf{Restricción de continuidad:} esta restricción implica que un cambio de luz o lightpath debe utilizar los mismos \textit{Frecuency Slots} (FS) a lo largo del camino establecido entre los nodos de origen y destino, tanto en la dimensión espectral como en la dimensión espacial (núcleo).
    \item \textbf{Restricción de consecutividad:} esta restricción establece que todos los FS utilizados para establecer un lightpath deben ser contiguos en el dominio espectral, formando un solo bloque contiguo de FS dentro del mismo núcleo. 
\end{itemize}
%

Estas restricciones conducen a que, tras sucesivas asignaciones y liberaciones de recursos, se genere la aparición de bloques aislados de FS no utilizados tanto en la dimensión espectral como en la dimesión espacial (núcleos) de los enlaces ópticos.
Dichos bloques fragmentados presentan desalineación tanto entre enlaces consecutivos de la ruta como entre los diferentes núcleos de una misma fibra multinúcleo. Como consecuencia, se incrementa significativamente la probabilidad de bloqueo de solicitudes, 
pudiendo la red rechazar demandas incluso cuando existe ancho de banda disponible suficiente en los enlaces. Este fenómeno se denomina \textbf{Fragmentación de la red} y en arquitecturas multinúcleo se manifiesta en dos domensiones complementarias:
%

\begin{itemize}
    \item \textbf{Fragmentación espectral:} se refiere a la presencia de bloques aislados de FS no utilizados en el dominio espectral, que no pueden ser aprovechados para establecer nuevas conexiones debido a las restricciones de continuidad y consecutividad.
    \item \textbf{Fragmentación espacial:} se refiere a la desalineación de bloques de FS disponibles entre los diferentes núcleos de una misma fibra multinúcleo, lo que dificulta la asignación eficiente de recursos en la dimensión espacial.
\end{itemize}
%

\textbf{Ejemplo ilustrativo del fenómeno:}
\begin{enumerate}[1 -]
   \item Se presenta el estado inicial del enlace mostrando las asignaciones activas de lightpaths distribuidos en los múltiples núcleos de la fibra.
   \item Se produce la liberación de recursos al finalizar el tiempo de vida de determinadas conexiones, generando segmentos espectrales disponibles dispersos en diferentes núcleos y posiciones del espectro.
   \item Se evidencia el rechazo de una nueva solicitud de conexión debido a que, pese a existir una cantidad agregada suficiente de FS libres en la red, estos no satisfacen simultáneamente las restricciones de contigüidad espectral dentro de un único núcleo y continuidad espacial a lo largo de la ruta. La conexión resulta bloqueada en todos los núcleos disponibles como resultado de la fragmentación tanto espectral como espacial inherente al sistema multinúcleo.  
\end{enumerate}
%

Para ilustrar estas restricciones, se presenta un ejemplo en las Figuras [] y [], donde se simula la conexión de una demanda de fod FS, con un nodo origen en 0 y un nodo destino en 3. En este escenario, existen dos posibblews caminos: 0-1-3 y 0-1-2-3.
%

La trayectoria de menor longitud corresponde a la ruta 0-1-3. No obstante, al procurar el establecimiento del lightpath mediante esta alternativa, la solicitud de conexión resulta denegada, dado que los enlaces 0-1 y 1-3 carecen de dos FS consecutivos y alineados espectralmente, tal como se evidencia en la Figura [].
%


\begin{figure}[H]
    \centering
    \includegraphics[width=1\textwidth]{capitulos/img/fragmentacionNuevo.png}
    \caption{Restricciones de contigüidad y continuidad aplicadas- Conexión Rechazada}
    \label{fig:fragmentacionNueva}
\end{figure}

\begin{figure}[H]
    \centering
    \includegraphics[width=1\textwidth]{capitulos/img/fragmentacion2Nuevo.png}
    \caption{Restricciones de contigüidad y continuidad aplicadas- Conexión Establecida}
    \label{fig:fragmentacion2Nueva}
\end{figure}

En contraste, al considerar la asignación del lightpath a través de la trayectoria de mayor extensión, específicamente 0-1-2-3, empleando los FS 4 y 5, la conexión se establece satisfactoriamente, puesto que esta configuración dispone de dos FS contiguos y alineados espectralmente, según se observa en la Figura [].
%

\subsection{Enfoques de gestión de fragmentación}
La problemática descrita previamente genera consecuencias perjudiciales para la infraestructura de red, ocasionando un incremento en la probabilidad de bloqueo y comprometiendo significativamente su desempeño óptimo y continuidad operacional. En consecuencia, resulta fundamental identificar estrategias que permitan prevenir, mitigar o reducir la fragmentación del espectro disponible.
%

De acuerdo con la bibliografía especializada, existen diversas aproximaciones que pueden ser consideradas para abordar la gestión de la desfragmentación. En la figura [] se presentan las principales estrategias de gestión de la fragmentación \cite{chatterjee2017fragmentation}.
%


La Desfragmentación constituye un procedimiento mediante el cual se ejecuta la reconfiguración o el re-ruteo de un subconjunto de conexiones existentes en la infraestructura de red. Su propósito fundamental consiste en reacomodar las asignaciones espectrales de las solicitudes de tráfico vigentes, consolidando de este modo los recursos disponibles en segmentos contiguos y continuos de mayor magnitud, los cuales pueden ser aprovechados para el establecimiento de futuras demandas \cite{talebi2014spectrum}.
%

Es factible abordar la problemática de la fragmentación prescindiendo de técnicas de desfragmentación espectral (Sin Desfragmentación), lo cual se alcanza mediante una administración del espectro orientada a la prevención de su fragmentación.
%

En el tratamiento de la fragmentación bajo un esquema Sin Desfragmentación, se pueden mencionar los algoritmos denominados Sensibles a la Fragmentación o \textit{Fragmentation Aware RSA} (FA-RSA). Estos consideran la fragmentación espectral durante el establecimiento de las demandas, empleando diversos indicadores de fragmentación, procurando así minimizar la fragmentación del espectro.
%

Alternativamente, es posible emplear técnicas de desfragmentación, las cuales se fundamentan en dos aproximaciones principales:
\begin{itemize}
         \item Desfragmentación Reactiva: El procedimiento se ejecuta como respuesta al bloqueo de una solicitud, con la finalidad de lograr su establecimiento exitoso.
         \item Desfragmentación Proactiva: Se lleva a cabo de manera periódica o en función de determinados umbrales que activan el proceso, permitiendo así reducir la fragmentación de la infraestructura de red y minimizar la ocurrencia de futuros bloqueos de solicitudes.
\end{itemize}
Las aproximaciones que implementan técnicas de desfragmentación pueden clasificarse además en: (i) estrategias sin re-ruteo, las cuales realizan únicamente una reasignación espectral en los \textit{lightpaths} o caminos ópticos establecidos, y (ii) estrategias con re-ruteo, que constituyen técnicas capaces de modificar tanto las rutas como el espectro asignado a los lightpaths existentes.
%

En el presente trabajo, para la gestión de la fragmentación se adoptó la aproximación con desfragmentación, de naturaleza proactiva y con re-ruteo de lightpaths preexistentes. En la figura [] se puede observar resaltada dicha estrategia.
%

\begin{figure}[H]
    \centering
    \includegraphics[width=1\textwidth]{capitulos/img/Gestion_Fragmentacion.PNG}
    \caption{Esquema de Gestión de la Fragmentación}
    \label{fig:gestion_fragmentacion}
\end{figure}
%


\section{Descripción del problema tratado}
La Fragmentación del Espectro en Redes Ópticas Elásticas Multinúcleo (Multi-Core EON) constituye una problemática que compromete la eficiencia en la utilización de recursos espectrales y espaciales. El desempeño de la infraestructura de red resulta severamente afectado, dado que este fenómeno puede ocasionar bloqueos de solicitudes debido a la ausencia de ranuras espectrales contiguas y alineadas entre enlaces consecutivos, así como por la indisponibilidad de núcleos adecuados, sin que necesariamente el espectro en todos los núcleos se encuentre completamente ocupado. En secciones previas se expusieron estrategias para el manejo de la fragmentación en la red; en el presente trabajo se examina la estrategia con desfragmentación, adoptando un enfoque proactivo.
%

Un método ampliamente implementado consiste en ejecutar el procedimiento de desfragmentación de manera periódica con el propósito de prevenir bloqueos futuros, abordando así una de las cuatro interrogantes planteadas por Zhang [], ¿Cuándo reconfigurar?.
%

En la figura [] se puede observar una posible solución a la problemática de la selección del momento óptimo para realizar la desfragmentación, la cual consiste en ejecutar desfragmentaciones periódicas en intervalos temporales fijos. En este caso, cada 100 unidades de tiempo, el eje vertical representa el volumen de tráfico cuantificado mediante el número de conexiones activas, mientras que el eje horizontal indica las unidades temporales; cada punto azul denota el instante en que el proceso de desfragmentación se ejecuta. Siguiendo este patrón, se evidencian situaciones donde se realizan procesos de desfragmentación cuando la red podría no requerirlos, considerando que la utilización de los recursos espectrales y de los núcleos constituye un indicador significativo del grado de fragmentación.
%

Además de la utilización de la red, existen otras métricas de fragmentación relevantes en redes multinúcleo, cuyos valores deben considerarse para el disparo de los procesos de desfragmentación, incluyendo la fragmentación por núcleo y la disponibilidad de recursos en la dimensión espacial.
%

De este modo, se evidencia la necesidad de un disparador inteligente para ejecutar el proceso de desfragmentación que considere todos estos parámetros o ``características'' para seleccionar apropiadamente el momento del disparo, dado que realizar múltiples desfragmentaciones de manera frecuente afecta directamente al desempeño de la red, pudiendo ocasionar disrupciones en las conexiones activas; mientras que ejecutar pocas desfragmentaciones muy dispersas resultaría en efectos prácticamente imperceptibles.
%

En síntesis, la selección del momento para ejecutar el proceso de desfragmentación resulta crítica debido a su impacto significativo en la cantidad de procesos de desfragmentación, lo cual incide directamente en las dos métricas globales más relevantes en el enrutamiento de redes ópticas elásticas multinúcleo: Cantidad de bloqueos y Cantidad de reconfiguraciones.
%

En los capítulos subsiguientes se presenta y aborda en profundidad un modelo de disparo inteligente que contempla numerosos factores tales como métricas de fragmentación de la red, utilización de recursos espectrales y espaciales, y bloqueos de solicitudes.
%

\begin{figure}[h!]
    \centering
    \includegraphics[width=1\textwidth]{capitulos/img/ejemploPeriodico.png}
    \caption{Ejemplo de desfragmentaciones periódicas con volumen de carga de tráfico variado}
    \label{fig:ejemploPeriodico}
\end{figure}
%


\section{Redes EON Multinúcleo}
Las redes EON Multinúcleo constituyen una variante avanzada de las redes EON que integran el concepto de fibras multinúcleo (MCF) para incrementar significativamente la capacidad de transmisión y la eficiencia espectral. Fundamentalmente, las redes EON Multinúcleo aprovechan múltiples núcleos independientes dentro de una misma fibra óptica para transmitir señales de forma simultánea y paralela, posibilitando una multiplicación de la capacidad de transmisión en comparación con las fibras convencionales [15].
 Estas características mencionadas en las EON Multinúcleo se materializan mediante la implementación de la Multiplexación por División de Espacio (SDM). Por esta razón, las redes EON Multinúcleo también se denominan SDM-EON[5]. En las redes EON Multinúcleo, se integra el concepto de asignación flexible de espectro con la utilización de múltiples núcleos, alcanzando una distribución más eficiente de la capacidad total de transmisión. Esta integración de tecnologías posibilita un incremento sustancial en la capacidad de transmisión a través de una única fibra, resultando esencial en un contexto donde la demanda de datos continúa creciendo de manera exponencial.
  Al incorporar el concepto de fibras multinúcleo en el diseño de las redes EON, se puede lograr una mayor adaptabilidad a las cambiantes necesidades del tráfico y una optimización más profunda de los recursos disponibles.
%

\begin{figure}[H]
    \centering
    \includegraphics[width=1\textwidth]{capitulos/img/MCF_NUCLEOS.png}
    \caption{Ejemplo de Fibras Multinúcleo (MCF) con 7, 12 y 19 núcleos}
    \label{fig:MCF_NUCLEOS}
\end{figure}
%

Conforme a lo documentado en la literatura [], inicialmente podría considerarse que la utilización de redes EON con mayor cantidad de núcleos proporciona ventajas sustanciales debido a la amplia disponibilidad de recursos espectrales y espaciales. No obstante, se ha identificado que en las redes MCF, el principal desafío radica en la interferencia denominada diafonía (crosstalk), la cual se genera cuando una fracción de la potencia óptica de un núcleo se propaga hacia los núcleos contiguos.
 Este fenómeno ocasiona una interferencia significativa en los circuitos activos y complejiza considerablemente la asignación de las ranuras espectrales (FS). Investigaciones previas [] [] han indicado que para viabilizar la implementación de redes EON multinúcleo, resulta fundamental desarrollar fibras que minimicen la diafonía entre núcleos adyacentes.
%

 En la Figura [], se presenta una configuración de MCF con 7 núcleos dispuestos en un patrón hexagonal. En esta arquitectura, el núcleo central (núcleo N° 6) se encuentra rodeado por 6 núcleos adyacentes, resultando en una mayor incidencia de diafonía sobre este núcleo.
 En contraste, los núcleos periféricos (núcleos N° 0, 1, 2, 3, 4 y 5) poseen 3 núcleos adyacentes cada uno. La Figura [] exhibe una MCF con 12 núcleos organizados en una disposición anular. En esta configuración, cada núcleo presenta exactamente 2 núcleos adyacentes, lo que deriva en que todos los núcleos experimenten un nivel equivalente de diafonía.
 %

 Finalmente, la Figura [] ilustra una MCF con 19 núcleos. En este tipo de fibras, los núcleos pueden presentar hasta 6 núcleos adyacentes por núcleo, generando una mayor incidencia de diafonía.
%

\begin{figure}[H]
    \centering
    \includegraphics[width=1\textwidth]{capitulos/img/CORE_PITCH.png}
    \caption{Core Pitch entre dos núcleos adyacentes en un MCF de 7 núcleos}
    \label{fig:CORE_PITCH}
\end{figure}

\section{Diafonía en Redes Ópticas Elásticas Multinúcleo}
La diafonía constituye un fenómeno indeseado que se manifiesta en las redes de fibra óptica cuando la señal transmitida en una fibra se acopla hacia otra fibra contigua. En el contexto de las redes EON fundamentadas en MCF, la diafonía se define como la interferencia entre conexiones ópticas establecidas en núcleos adyacentes que emplean las mismas ranuras espectrales (FS) en un enlace común.
 Este tipo de interferencia se denomina diafonía entre núcleos o Inter-Core Crosstalk (XT). La interferencia ocasionada por el XT puede degradar la calidad de la señal en las FS afectadas, lo que implica que la señal en estas ranuras puede experimentar distorsiones, incrementando la probabilidad de errores en la transmisión de datos.
%
 
 Consecuentemente, impacta directamente en la capacidad de la red al «inhabilitar» estas FS para la transferencia de datos debido a la diafonía, generando espacios no utilizables entre ellas.
%

 En síntesis, esto deriva en una reducción de la cantidad total de FS disponibles para la transmisión de datos, limitando la capacidad operativa de la red. Con menor disponibilidad de FS, se reducen los canales para transmitir datos, lo que puede restringir la capacidad total de transmisión de la infraestructura de red.
%

 En las redes SDM-EON, el XT entre dos núcleos de una MCF depende significativamente de la distancia entre dicho par de núcleos, denominada core pitch ($\Lambda_{i,j}$). A mayor core pitch, menor será el impacto del XT entre estos dos núcleos [].
 No obstante, resulta importante destacar que, a medida que se incrementa el core pitch, la capacidad de la fibra óptica disminuye. Es decir, al aumentar la distancia física entre dos núcleos, se reduce el espacio disponible en la fibra óptica para albergar núcleos adicionales.
 Por consiguiente, resulta fundamental establecer un equilibrio entre un core pitch reducido para incrementar la capacidad y uno suficientemente amplio para minimizar los efectos del XT. Este balance posibilita optimizar el desempeño y la eficiencia de la red.
 %

\begin{figure}[H]
    \centering
    \includegraphics[width=1\textwidth]{capitulos/img/XT_MCF.png}
    \caption{XT en un Fibra MCF de 3 núcleos []}
    \label{fig:XT_MCF}
\end{figure}
%

Para demostrar el comportamiento del crosstalk (XT) entre núcleos en fibras multinúcleo, se analiza el ejemplo presentado en la figura [], la cual ilustra una fibra óptica multinúcleo (MCF) compuesta por tres núcleos organizados en configuración lineal.
En esta topología, resulta evidente que el núcleo central (núcleo 2) experimenta una interferencia significativa por XT intercore. Este fenómeno se atribuye a que ambos núcleos contiguos (núcleos 1 y 3) mantienen conexiones establecidas en intervalos espectrales similares a aquellos ya ocupados en el núcleo 2.
A modo de ejemplo, los segmentos espectrales FS1 y FS2 resultan inutilizables en el núcleo 2 debido a la interferencia generada por las asignaciones activas en los núcleos laterales.
Esta misma situación se replica en los intervalos FS4, FS5 y FS6 del núcleo mencionado. Consecuentemente, en arquitecturas SDM-EON basadas en fibras multinúcleo, resulta imperativo evaluar el impacto del XT intercore durante el proceso de asignación de recursos espectrales, a fin de mitigar degradaciones en la calidad de transmisión.

Es relevante destacar que el crosstalk intercore no afecta exclusivamente a los segmentos espectrales con demandas activas, sino también a aquellos recursos disponibles en núcleos adyacentes. Un análisis detallado de la figura [] revela que incluso los intervalos espectrales sin conexiones asignadas experimentan interferencia procedente de transmisiones en núcleos contiguos.
 Esto se ejemplifica con los segmentos FS8 y FS9 del núcleo 1, que sufren degradación por el XT generado desde las mismas posiciones espectrales en el núcleo 2. En determinadas circunstancias, esta interferencia puede superar el umbral de crosstalk admisible, imposibilitando la utilización de dichos recursos para el establecimiento de futuras conexiones en arquitecturas multinúcleo.

 %\section{Análisis Bibliográfico}     
% El trabajo presentado por M. Quagliotti \cite{quagliotti2017spectrum} propone un algoritmo RSA que busca mantener bajo el índice de fragmentación mediante el uso de una heurística básica durante la asignación del espectro, también brinda una amplia y útil descripción de métricas para evaluar la fragmentación del espectro.

% Para recopilar el conjunto de métricas de fragmentación utilizadas, realizaron un extenso estudio de la literatura científica, las cuales son: Utilization Entropy (UE), Shannon Entropy (SHF), External Fragmentation (EF), Access Blocking Probability (ABP), Compactness (SC) y High-slot Mark (HSM).

% Cada métrica de fragmentación proporciona su propia y peculiar medida de ocupación del espectro, relacionadas a la accesibilidad de recursos para el caso de EF, y ABP, grado de desorden y entropía en UE y SHF y compacidad del espectro ocupado en SC.

% Para nuestra investigación utilizamos algunas de las métricas presentadas en el mencionado articulo, las cuales nos sirven como características esenciales en la construcción de nuestro modelo de predicción de probabilidades de bloqueo.  

% Seguidamente presentamos un análisis bibliográfico de trabajos presentes en el estado del arte que abordaron el mismo problema.

% Jaume Comellas, Laura Vicario, y Gabriel Junyent \cite{comellas2018periodic} nos presentan un análisis de la desfragmentación periódica en redes EON con un tráfico dinámico, evaluando los diferentes efectos de los parámetros de desfragmentación en el rendimiento de la red.

% El enfoque presentado en el trabajo consiste en realizar las desfragmentaciones en periodos fijos de N unidades de tiempo, con el principal objetivo de encontrar valores adecuados de N, ya que periodos de desfragmetación muy pequeños implican tener el espectro tan compactado como sea posible, pero a expensas de ejecutar el algoritmo con demasiada frecuencia, lo cual añade complejidad al proceso. Por otro lado, para valores muy altos de N, los efectos de la desfragmentación son insignificantes.

% Este tipo de desfragmentaciones en periodos fijos es utilizado ampliamente en distintas investigaciones tales cómo \cite{davalos2019spectrum}, \cite{luo2012partial}, entre otros. En nuestra investigación utilizamos esta técnica a fin de comparar los resultados con el disparador que proponemos.

% Otra manera de afrontar el enfoque proactivo del proceso de desfragmentación es realizarlo mediante algún tipo de disparador de tal manera que la misma sea ejecutada solo en periodos de tiempo donde es realmente necesaria, a continuación, veremos algunos artículos los cuales usaron esta estrategia.

% La investigación realizada por Yutaka Takita y colegas \cite{takita2016wavelength} propone un mecanismo de disparo para el proceso de desfragmentación basado en el valor del \textit{High-slot Mark} (HM) el cual indica el número máximo de una ranura ocupada en la red. Utilizan esta métrica ya que lo consideran como una medida válida para evaluar la eficiencia en la utilización de los recursos. El proceso de desfragmentación se dispara de manera aleatoria cuando el valor del HM es mayor a un valor de \( HM_{max} \) definido previamente, en el artículo mencionado utilizan 30 como valor para \( HM_{max} \). 

% Para el proceso de disparo en nuestro método propuesto se utilizan un conjunto de características o parámetros que indican el estado actual de la red, parte de ellas al igual que el \textit{High-slot Mark} son también métricas que indican la fragmentación del espectro. En el capitulo 4 se explican en profundidad estas características donde una de ellas es el llamado \textit{Maximum Slot Index} (MSI) el cual tiene una definición equivalente al del \textit{High-slot Mark}. 

% Otra propuesta para el disparo es presentada por Ricardo V. Fávero y colegas \cite{favero2015new}. En su método combinan el enfoque reactivo y el enfoque proactivo para determinar el periodo en el que será ejecutado el proceso de desfragmentación, en la figura \ref{fig:favero} se puede observar el diagrama que ilustra el algoritmo propuesto.

% Inicialmente la variable \textit{d} que utilizan para representar el estado de fragmentación se coloca en 0, el proceso de desfragmentación (DS) se ejecuta al cumplirse alguna de las siguientes condiciones.

% Si se intenta establecer una demanda y no se encuentra un camino disponible para la misma, se verifica la variable \textit{d}, si esta se encuentra en 0 se ejecuta el proceso \textit{DS}, si no se logra establecer la demanda aun después del proceso de desfragmentación la misma se bloquea y la variable \textit{d} se cambia a 1.

% La otra posibilidad de ejecución es cuando se intenta liberar una demanda, se ejecuta el proceso \textit{DS} sí \textit{d} = 1 y si la cantidad de \textit{lighpaths} liberados (\textit{r}) es igual a la variable predefinida previamente \textit{R}. 

% Como se puede ver el método propuesto considera principalmente las conexiones liberadas por lo que puede considerarse como periódica.


% \begin{figure}[h!]
%     \centering
%     \includegraphics{capitulos/img/disparoFavero}
%     \caption{Algoritmo propuesto por Favero y colegas \cite{favero2015new}}
%     \label{fig:favero}
% \end{figure}

%Mingyang Zanhg y colegas plantean en su investigación \cite{zhang2013bandwidth} un disparo para el proceso de desfragmentación basado en la cantidad de conexiones liberadas. Básicamente consiste en ejecutar la desfragmentación cuándo la cantidad de conexiones liberadas es igual al parámetro fijo \textit{TH} (\textit{Threshold}). En su investigación utilizaron 300 como valor de \textit{TH}.

%El trabajo presentado por Jie Zhang y colegas \cite{zhang2012priority}, proponen un disparo basado en el concepto de \textit{Spectrum Compacteness} o Compacidad del Espectro (SC) el cual es una métrica de fragmentación.

%Para determinar el momento del disparo para el proceso de desfragmentación tienen en cuenta los siguientes pasos:  
% \begin{enumerate}[label=\arabic*)]
%     \item Seleccionar un valor apropiado de \textit{Spectrum Compactness} (SC) para actuar como umbral (T) para el disparo de la desfragmentación.
%     \item Actualizar el valor de SC después de liberar conexiones o al establecer una nueva conexión. 
%     \item Comparar los valores de SC y T; si SC < T disparar la desfragmentación y pasar al siguiente paso, sino volver al paso 2.
%     \item Actualizar el ultimo valor de SC después de la desfragmentación y volver al paso 3.
% \end{enumerate}

% Por último el trabajo propuesto por Zhang y colegas \cite{zhang2014dynamic} presentan un análisis del problema de la desfragmentación de redes EON dividido en cuatro sub-problemas, el tercero de ellos es ¿Cuándo reconfigurar?.

% Para el tercer subproblema plantean un algoritmo de disparo el cual tiene en cuenta la probabilidad de bloqueo instantánea (B) en un periodo \(\Delta \)t \((B(\Delta t))\) y la utilización del ancho de banda. 
% De esta forma buscan realizan la comparación entre \(B(\Delta t)\) con \(B_{th}\) (umbral de probabilidad de bloqueo para el disparo del proceso de desfragmentación) solo cuando la red se encuentra con un crecimiento en la utilización del ancho de banda. 
\newpage


\chapter{Aprendizaje Automático}
%

El Aprendizaje Automático, conocido en inglés como \textit{Machine Learning}, representa una de las áreas más dinámicas y prometedoras dentro del campo de la inteligencia artificial contemporánea. se fundamenta en el desarrollo de algoritmos y modelos computacionales capaces de identificar patrones complejos en conjuntos de datos, con el propósito de generar predicciones o tomar decisiones informadas sin necesidad de instrucciones programáticas explícitas para cada escenario específico.[]
%

La esencia del aprendizaje automático radica en su capacidad para mejorar el desempeño de manera iterativa mediante la experiencia acumulada. Mitchell proporciona una definición operacional particularmente esclarecedora: un sistema computacional manifiesta capacidad de aprendizaje cuando su rendimiento en una tarea determinada T, cuantificado mediante una métrica de desempeño P, experimenta una mejora mensurable como consecuencia de la exposición a una experiencia E.[] Esta conceptualización establece tres componentes fundamentales que articulan cualquier sistema de aprendizaje automático: la tarea objetivo, la experiencia de aprendizaje y el criterio de evaluación.
%

Para ilustrar estos conceptos de manera concreta, se puede examinar el caso de los sistemas de filtrado de correo electrónico no deseado. Un filtro de spam ejemplifica de forma paradigmática los principios del aprendizaje automático. El sistema desarrolla progresivamente la capacidad de discriminar entre mensajes legítimos y correo no solicitado mediante el análisis de ejemplos previamente etiquetados por usuarios. Estos conjuntos de datos, denominados conjuntos de entrenamiento, contienen tanto instancias positivas (correos identificados como spam) como negativas (mensajes legítimos), permitiendo al algoritmo extraer características distintivas de cada categoría.
%

En este contexto específico, la tarea T consiste en la clasificación binaria de nuevos mensajes electrónicos, la experiencia E está constituida por el proceso de entrenamiento con los datos etiquetados, y la métrica de desempeño P puede definirse como la tasa de precisión o \textit{Accuracy}, que cuantifica la proporción de mensajes correctamente clasificados en relación con el total de predicciones realizadas.
%

\section{Clasificación de sistemas o tipos de aprendizaje automático}
%

La diversidad de aplicaciones y contextos en los que se implementan sistemas de aprendizaje automático ha propiciado el desarrollo de múltiples paradigmas metodológicos. La clasificación más fundamental de estos enfoques se establece en función del tipo y grado de supervisión disponible durante la fase de entrenamiento. A continuación, se muestran las tres categorías principales:
%

\begin{itemize}
    \item \textbf{Aprendizaje supervisado:} Este paradigma constituye el enfoque más ampliamente implementado en aplicaciones prácticas. Se caracteriza por la disponibilidad de un conjunto de entrenamiento que incluye pares de entrada-salida, donde cada instancia de entrada está asociada con su correspondiente etiqueta o \textit{label}, que representa la solución correcta. El objetivo del algoritmo consiste en inferir una función de mapeo que establezca la correspondencia óptima entre el espacio de características de entrada y el conjunto de salidas deseadas, de manera que pueda generalizar efectivamente a instancias no observadas previamente.
    
    El aprendizaje supervisado se subdivide en dos categorías fundamentales según la naturaleza de la variable objetivo:

    \begin{itemize}
        \item \textit{Clasificación:} Se trata de brindar ejemplos de entrenamiento donde cada instancia está asociada con una o múltiples clases predefinidas, a modo de que se pueda realizar el entrenamiento y clasificar nuevas entradas dentro de alguna de las clases existentes. Aplicaciones típicas incluyen el reconocimiento de imágenes, detección de fraudes y análisis de sentimientos.
        
        \item \textit{Predicción:} A diferencia de la clasificación, éste consiste en la predicción de una variable objetivo de naturaleza continua o numérica. El sistema recibe datos de entrenamiento compuestos por vectores de características junto con sus valores objetivos correspondientes, permitiendo al modelo aprender la relación funcional entre sí. Esta capacidad predictiva se aplica posteriormente para estimar valores numéricos de nuevas instancias basándose exclusivamente en sus características de entrada. Aplicaciones típicas comprenden la predicción de precios, estimación de demanda y proyecciones temporales.
    \end{itemize}
    
    \item \textbf{Aprendizaje no supervisado:} Este paradigma aborda escenarios donde los datos de entrenamiento carecen de soluciones deseadas. La ausencia de supervisión directa plantea un desafío metodológico fundamentalmente diferente: el algoritmo debe descubrir estructuras intrínsecas, patrones latentes o relaciones subyacentes en los datos sin guía externa. Las técnicas de aprendizaje no supervisado resultan particularmente valiosas para tareas exploratorias, tales como la segmentación de clientes, detección de anomalías, reducción de dimensionalidad y descubrimiento de asociaciones en grandes volúmenes de datos. Este enfoque refleja una aproximación más cercana a cómo los sistemas biológicos pueden aprender mediante la observación y organización autónoma de información sensorial.
    
    \item \textbf{Aprendizaje por refuerzo:} Este paradigma se distingue por su naturaleza secuencial e interactiva. En lugar de aprender a partir de un conjunto estático de ejemplos, el aprendizaje por refuerzo se fundamenta en la interacción continua de un agente con un entorno dinámico. El proceso de aprendizaje se articula mediante señales de retroalimentación en forma de recompensas (positivas o negativas) que el agente recibe como consecuencia de sus acciones. El objetivo fundamental consiste en desarrollar una política de comportamiento que maximice la recompensa acumulada a largo plazo.
     Este bucle de retroalimentación continua entre acción, observación y recompensa permite al sistema refinar progresivamente su estrategia mediante exploración y explotación del espacio de estados. Las aplicaciones emblemáticas incluyen sistemas de control robótico, estrategias de juegos, optimización de recursos y vehículos autónomos.
\end{itemize}
%

Cada uno de estos paradigmas presenta ventajas distintivas y limitaciones inherentes, determinando su idoneidad para contextos específicos. La selección del enfoque apropiado constituye una decisión metodológica crucial que debe considerar tanto la naturaleza del problema como las características de los datos disponibles.
%

\section{Gradient Boosting}

El Gradient Boosting representa uno de los algoritmos de aprendizaje automático supervisado más potentes y efectivos en la actualidad, especialmente para problemas de clasificación y regresión. Este método se fundamenta en el paradigma de aprendizaje por ensamble, donde múltiples modelos predictivos débiles se combinan secuencialmente para construir un predictor robusto de alto rendimiento.

A diferencia de las redes neuronales artificiales que se inspiran en la arquitectura biológica del cerebro, el Gradient Boosting se sustenta en principios de optimización matemática y aprendizaje secuencial. La filosofía subyacente consiste en entrenar modelos de manera iterativa, donde cada nuevo modelo se especializa en corregir los errores residuales cometidos por los modelos previos, generando así una mejora progresiva del rendimiento global del sistema.

\subsection{Fundamentos del Aprendizaje por Ensamble}

El aprendizaje por ensamble o \textit{Ensemble Learning} constituye una estrategia metodológica que combina las predicciones de múltiples modelos base para obtener un resultado final superior al que produciría cualquier modelo individual. Este enfoque se fundamenta en dos principios estadísticos complementarios:

\begin{itemize}
    \item \textbf{Reducción de varianza:} Mediante la agregación de predicciones de modelos diversos, se reduce la sensibilidad del sistema a fluctuaciones en los datos de entrenamiento, incrementando la estabilidad de las predicciones.
    
    \item \textbf{Reducción de sesgo:} La combinación secuencial de modelos permite corregir sistemáticamente errores persistentes, mejorando la capacidad del sistema para capturar relaciones complejas en los datos.
\end{itemize}

Existen dos estrategias principales en el aprendizaje por ensamble. El \textit{Bagging} o agregación bootstrap, entrena múltiples modelos de manera independiente y paralela sobre diferentes subconjuntos de datos, combinando posteriormente sus predicciones mediante votación o promediación. El \textit{Boosting}, por otro lado, entrena modelos de forma secuencial, donde cada nuevo modelo se enfoca en corregir los errores de sus predecesores, estableciendo una relación de dependencia entre los aprendices.

El Gradient Boosting pertenece a esta segunda categoría, distinguiéndose por su fundamentación matemática rigurosa y su eficacia demostrada en múltiples dominios de aplicación.

\subsection{Principios del Gradient Boosting}

El algoritmo de Gradient Boosting construye el modelo predictivo mediante la adición secuencial de funciones, típicamente árboles de decisión de profundidad limitada, que optimizan iterativamente una función objetivo. El proceso puede conceptualizarse como un descenso del gradiente en el espacio funcional, donde en cada iteración se añade una nueva función que aproxima el gradiente negativo de la pérdida respecto a las predicciones actuales.

Formalmente, dado un conjunto de entrenamiento $\{(x_i, y_i)\}_{i=1}^{n}$ donde $x_i$ representa el vector de características y $y_i$ la variable objetivo, el Gradient Boosting construye un modelo aditivo de la forma:

\begin{equation}
F_M(x) = \sum_{m=0}^{M} \gamma_m h_m(x)
\end{equation}

donde $h_m(x)$ representa el $m$-ésimo aprendiz débil, típicamente un árbol de decisión, $\gamma_m$ es el coeficiente de peso asociado, y $M$ denota el número total de iteraciones o árboles en el ensamble.

El algoritmo opera mediante el siguiente procedimiento iterativo:

\begin{enumerate}
    \item \textbf{Inicialización:} Se establece un modelo inicial $F_0(x)$, comúnmente una constante que minimiza la función de pérdida sobre el conjunto de entrenamiento completo.
    
    \item \textbf{Iteración secuencial:} Para cada iteración $m = 1, 2, ..., M$:
    \begin{itemize}
        \item Se calculan los pseudo-residuos, representando el gradiente negativo de la función de pérdida respecto a las predicciones actuales:
        \begin{equation}
        r_{im} = -\left[\frac{\partial L(y_i, F(x_i))}{\partial F(x_i)}\right]_{F(x)=F_{m-1}(x)}
        \end{equation}
        
        \item Se entrena un nuevo aprendiz débil $h_m(x)$ para predecir estos residuos, ajustándose a los patrones de error del modelo acumulado.
        
        \item Se determina el coeficiente óptimo $\gamma_m$ que minimiza la pérdida al incorporar el nuevo aprendiz:
        \begin{equation}
        \gamma_m = \arg\min_{\gamma} \sum_{i=1}^{n} L(y_i, F_{m-1}(x_i) + \gamma h_m(x_i))
        \end{equation}
        
        \item Se actualiza el modelo mediante la adición ponderada del nuevo componente:
        \begin{equation}
        F_m(x) = F_{m-1}(x) + \nu \cdot \gamma_m h_m(x)
        \end{equation}
        donde $\nu$ representa la tasa de aprendizaje o \textit{learning rate}, un hiperparámetro que controla la contribución de cada árbol al modelo final.
    \end{itemize}
\end{enumerate}

Este proceso iterativo continúa hasta alcanzar el número especificado de árboles $M$ o hasta satisfacer un criterio de convergencia establecido.

\subsection{GradientBoostingClassifier}

El \textit{GradientBoostingClassifier} constituye la implementación específica del algoritmo Gradient Boosting para problemas de clasificación. Esta variante emplea funciones de pérdida apropiadas para variables categóricas y adapta el procedimiento de optimización para generar probabilidades de pertenencia a clases.

Para clasificación binaria, se utiliza típicamente la función de pérdida logística o \textit{log loss}:

\begin{equation}
L(y, F(x)) = \log(1 + e^{-2yF(x)})
\end{equation}

donde $y \in \{-1, 1\}$ representa la clase verdadera y $F(x)$ la predicción del modelo. La probabilidad de pertenencia a la clase positiva se obtiene mediante la transformación logística:

\begin{equation}
P(y=1|x) = \frac{1}{1 + e^{-2F(x)}}
\end{equation}

Para problemas de clasificación multiclase, se extiende el enfoque mediante la estrategia \textit{one-versus-all}, entrenando un modelo separado por cada clase y combinando las predicciones mediante normalización softmax para obtener distribuciones de probabilidad válidas.

\subsection{Hiperparámetros y Regularización}

El rendimiento del \textit{GradientBoostingClassifier} depende críticamente de la configuración apropiada de sus hiperparámetros, los cuales regulan la complejidad del modelo y previenen el sobreajuste:

\begin{itemize}
    \item \textbf{Número de estimadores (n\_estimators):} Define la cantidad de árboles en el ensamble. Valores elevados incrementan la capacidad expresiva pero aumentan el riesgo de sobreajuste y el costo computacional. Típicamente se emplean valores entre 100 y 1000.
    
    \item \textbf{Tasa de aprendizaje (learning\_rate):} Controla la contribución de cada árbol al modelo final. Valores pequeños (0.01-0.1) requieren más árboles pero generalmente producen mejor generalización. Existe una relación de compromiso entre este parámetro y el número de estimadores.
    
    \item \textbf{Profundidad máxima (max\_depth):} Limita la profundidad de cada árbol individual. Árboles superficiales (3-5 niveles) actúan como aprendices débiles efectivos, mientras que árboles profundos incrementan la complejidad y el riesgo de sobreajuste.
    
    \item \textbf{Mínimo de muestras por división (min\_samples\_split):} Especifica el número mínimo de muestras requeridas para dividir un nodo interno. Valores mayores previenen la creación de divisiones excesivamente específicas.
    
    \item \textbf{Mínimo de muestras por hoja (min\_samples\_leaf):} Define el número mínimo de muestras en los nodos terminales. Este parámetro suaviza el modelo en regiones de baja densidad de datos.
    
    \item \textbf{Submuestreo (subsample):} Fracción de muestras utilizada para entrenar cada árbol. Valores menores a 1.0 introducen aleatorización estocástica, mejorando la diversidad del ensamble y reduciendo el sobreajuste.
\end{itemize}

La selección óptima de estos hiperparámetros requiere típicamente validación cruzada y búsqueda sistemática en el espacio de configuraciones mediante técnicas como \textit{Grid Search} o \textit{Random Search}.

\subsection{Ventajas y Limitaciones}

El \textit{GradientBoostingClassifier} presenta características distintivas que determinan su idoneidad para diferentes contextos:

\textbf{Ventajas:}
\begin{itemize}
    \item Capacidad para modelar relaciones no lineales complejas sin requerir transformaciones explícitas de características
    \item Robustez ante variables de diferentes escalas, eliminando la necesidad de normalización
    \item Manejo natural de variables mixtas (numéricas y categóricas)
    \item Resistencia a \textit{outliers} mediante funciones de pérdida apropiadas
    \item Interpretabilidad mediante análisis de importancia de características
    \item Rendimiento competitivo en conjuntos de datos tabulares estructurados
\end{itemize}

\textbf{Limitaciones:}
\begin{itemize}
    \item Susceptibilidad al sobreajuste con configuraciones inadecuadas de hiperparámetros
    \item Entrenamiento secuencial que limita la paralelización eficiente
    \item Mayor costo computacional comparado con algoritmos más simples
    \item Sensibilidad al desbalance de clases, requiriendo estrategias de ponderación
    \item Rendimiento subóptimo en datos de muy alta dimensionalidad comparado con métodos especializados
\end{itemize}

\section{Aplicación de Machine Learning en Redes Ópticas Elásticas Multinúcleo}
%

En esta sección se presenta un estudio bibliográfico del estado del arte de técnicas de \textit{Machine Learning} aplicadas a problemas en redes ópticas elásticas multinúcleo (MC-EON, \textit{Multi-Core Elastic Optical Networks}).
%

Panchali Datta Choudhury y Tanmay De presentan un análisis comprehensivo del uso de técnicas de \textit{Machine Learning} en redes ópticas elásticas \cite{[]}, fundamento que se extiende a las arquitecturas multinúcleo. Las MC-EON introducen complejidades adicionales respecto a las redes ópticas elásticas convencionales, particularmente en la gestión de múltiples núcleos dentro de una misma fibra y los fenómenos de interferencia entre núcleos (inter-core crosstalk), aspectos que requieren estrategias de optimización más sofisticadas donde el \textit{Machine Learning} demuestra particular utilidad.
%

Las principales áreas de aplicación de estas técnicas en el contexto de MC-EON incluyen:
%

\begin{itemize}
    \item \textbf{Evaluación y predicción de calidad de servicio}
    
    La investigación presentada en \cite{[]} propone un modelo de asignación de ancho de banda en EON considerando los requisitos de calidad de servicio o \textit{Quality of Service} (QoS). Se emplea aprendizaje por refuerzo o \textit{Reinforcement Learning}, donde la función de recompensa se fundamenta en el cumplimiento de los requisitos de QoS. En el contexto de MC-EON, esta aproximación adquiere mayor relevancia debido a la necesidad de garantizar QoS considerando simultáneamente la asignación de recursos en múltiples núcleos y la gestión de interferencias entre ellos.
    
    \item \textbf{Supervivencia de red}
    
    El trabajo presentado en \cite{[]} explora la optimización de redes considerando su capacidad de supervivencia mediante aprendizaje por refuerzo profundo. Se implementa una arquitectura de dos agentes: uno proporciona el esquema de trabajo principal y otro gestiona el esquema de protección. Esta combinación, junto con un mecanismo de recompensas orientado a maximizar la rentabilidad, genera políticas de enrutamiento, asignación de espectro y selección de modulación que garantizan supervivencia. En MC-EON, estos mecanismos de protección resultan especialmente críticos dado el mayor número de recursos físicos susceptibles a fallos.
    
    \item \textbf{Predicción de tráfico}
    
    Aibin \cite{[]} presenta un enfoque para predicción de tráfico en redes de centros de datos en la nube o \textit{Cloud Data Center Networks} utilizando búsqueda de árbol de Monte Carlo. Para cada solicitud, esta técnica identifica el centro de datos más apropiado y el conjunto óptimo de rutas candidatas mediante la construcción de un árbol disperso y selección estocástica. Esta metodología es aplicable a MC-EON para predecir patrones de tráfico y optimizar la asignación de núcleos.
    
    \item \textbf{Enrutamiento, modulación y asignación del espectro}
    
    Chen et al. \cite{[]} proponen un modelo de aprendizaje por refuerzo profundo para desarrollar un sistema autónomo de RMSA en redes ópticas elásticas. Emplean redes neuronales convolucionales, denominadas \textit{Q Networks}, para aprender políticas RMSA considerando conectividad, utilización espectral y demandas de tráfico. En MC-EON, este enfoque se extiende al problema RMSCA (\textit{Routing, Modulation, Spectrum and Core Assignment}), donde adicionalmente se debe seleccionar el núcleo óptimo y considerar las restricciones de crosstalk entre núcleos.
    
\end{itemize}
%

Durante este trabajo dentro de los puntos mas destacados nos podemos encontrar con algunos trabajos cuyo principal objetivo y enfoque es la solución al problema de la fragmentación de la red, con alguno de ellos para otro tipo de redes, como el presentado en \cite{trindade2020machine}, el cual encuentra su principal enfoque en \textit{Space Division Multiplexing Elastic Optical Networks} o SDM-EON , implementando redes neuronales, específicamente la red neuronal de Elman, para la predicción de tráfico de manera de disminuir la fragmentación y la diafonía o \textit{Cross-talk}.
 También se cuenta con el trabajo presentado por Enciso y Silva, el cual propone un algoritmo para decidir el mejor momento para disparar la desfragmentación de la red. Tomando ese trabajo como base, en este trabajo se proponen modelos para predecir los mejores momentos para para ejecutar la desfragmentación para Redes Ópticas Elásticas Multinúcleo (MC-EON). De manera en que se contemplan los problemas anteriores, agregando la complejidad que conlleva el uso de múltiples núcleos en una misma fibra.
%%

Para redes MC-EON también contamos con algoritmos de desfragmentación basados en Machine Learning, como el presentado en \cite{xiong2019machine}, donde los autores proponen un enfoque de aprendizaje no supervisado que no requiere conocimientos previos de la red. El algoritmo identifica aquellos \textit{lightpaths} que pueden ser agrupados en base a ciertas características, para luego mapear esos grupos a los núcleos y reordenar el espectro sin necesidad de realizar re-ruteos.
%

\subsection{Gradient Boosting en MC-EON}

En el contexto de las redes ópticas elásticas multinúcleo (MC-EON), los algoritmos de Gradient Boosting han demostrado particular eficacia para tareas de clasificación y predicción relacionadas con la gestión dinámica de recursos y la optimización del rendimiento de la red.

Las principales áreas de aplicación de \textit{GradientBoostingClassifier} en MC-EON incluyen:

\begin{itemize}
    \item \textbf{Predicción de bloqueo de solicitudes:} El \textit{GradientBoostingClassifier} puede entrenarse para predecir la probabilidad de que una solicitud de conexión sea bloqueada dadas las condiciones actuales de la red, considerando factores como la fragmentación espectral, la disponibilidad de núcleos y los niveles de diafonía entre núcleos.
    
    \item \textbf{Clasificación de eventos de desfragmentación:} El modelo puede determinar el momento óptimo para ejecutar algoritmos de desfragmentación, clasificando el estado de la red en categorías que indican la necesidad o conveniencia de reorganizar las asignaciones espectrales. Esta capacidad resulta particularmente valiosa para decisiones en tiempo real que deben balancear el costo operacional de la desfragmentación contra los beneficios en términos de reducción de bloqueos futuros.
    
    \item \textbf{Selección de estrategias RMSCA:} Mediante el aprendizaje de patrones históricos, el clasificador puede seleccionar entre diferentes estrategias de \textit{Routing, Modulation, Spectrum and Core Assignment} según las características de la demanda y el estado de la red. La capacidad del modelo para capturar interacciones no lineales entre múltiples variables permite adaptar dinámicamente la estrategia de asignación a condiciones cambiantes de tráfico.
    
    \item \textbf{Detección de degradación de QoS:} El algoritmo puede identificar situaciones donde la calidad de servicio está en riesgo de degradarse, permitiendo acciones preventivas antes de que ocurran violaciones de los acuerdos de nivel de servicio. La interpretabilidad del modelo mediante análisis de importancia de características facilita la identificación de los factores más críticos que afectan la calidad de servicio.
\end{itemize}

La naturaleza tabular de los datos operacionales en MC-EON (métricas de utilización, estadísticas de tráfico, indicadores de fragmentación) hace particularmente adecuado el uso de Gradient Boosting, cuyo rendimiento en este tipo de datos frecuentemente supera a aproximaciones basadas en redes neuronales profundas. Adicionalmente, la capacidad del modelo para proporcionar estimaciones de importancia de características facilita la comprensión de los factores más relevantes en las decisiones de gestión de la red, aspecto crítico para la validación y aceptación de sistemas autónomos en entornos de producción.

El tiempo de entrenamiento relativamente reducido del \textit{GradientBoostingClassifier} comparado con redes neuronales profundas, junto con su robustez ante desbalances moderados en las clases y su capacidad para manejar características de diferentes tipos y escalas sin preprocesamiento extensivo, lo convierten en una opción pragmática para implementaciones en sistemas de gestión de MC-EON donde la eficiencia computacional y la confiabilidad son requisitos fundamentales.
\chapter{ Método Propuesto }
Este trabajo propone una solución para la selección del momento de disparo de procesos proactivos de desfragmentación en redes EON, basado en técnicas de \textit{Machine Learning} para los cuáles predicen la probabilidad de bloqueo de una demanda \textit{unicast} en un momento determinado \textit{t}.

Para esta predicción se utiliza un modelo de regresión basado en redes neuronales, utilizando un conjunto de datos de simulaciones de tráfico \textit{unicast} en diversas topologías de redes EON, tomando parámetros o características relacionadas a la fragmentación y la utilización de la red como datos de entrada y produciendo un valor estimado de la probabilidad de bloqueo. Se fija además un valor límite de probabilidad para la realización del proceso de desfragmentación.

\section{Características}

En el área de \textit{Machine Learning}, se conoce cómo ``características'' a los parámetros o datos de entrada del modelo de aprendizaje.

Las características seleccionadas fueron aquellas que se encuentran relacionadas al uso y la fragmentación de la red, así como también al bloqueo de las demandas. Se tomaron las principales métricas usadas para la determinación del estado de fragmentación, de acuerdo con \cite{quagliotti2017spectrum}, además de otras relacionadas a la utilización de la red. 

Estas características son las siguientes: 

\begin{itemize}
    \item Entropía de utilización\cite{wang2012utilization}: La entropía de utilización es una métrica de fragmentación de enlaces de la red.  Un valor bajo de entropía indica que el ancho de banda de los enlaces de fibra óptica está siendo usado de forma ordenada, con menos bloques de FS utilizados o no utilizados, y con un nivel de fragmentación menor.
    La entropía de un enlace está definida como:
    \begin{equation}
        Ent_{link} = \sum_{i=1}^{N-1} FS_{i} \oplus FS_{i + 1}
    \end{equation}
    donde N es la cantidad de FS en el enlace, \(FS_{i}\) representa al FS de índice i dentro del enlace, y tiene valor 1 si el FS está ocupado, y valor 0 en caso contrario, y se realiza una operación XOR sobre FS contiguos del enlace.
    Para obtener la entropía de la red se calcula el promedio de la entropía en cada enlace:
    \begin{equation}
        Ent_{red} = \sum_{i=1}^{\left | E \right |} \frac{Ent_{link - i}}{{\left | E \right |}}
    \end{equation}
    \item Entropía de Shannon (SHF)\cite{wright2015minimum}: Es una métrica de fragmentación de enlaces, que es una variación de la anterior característica, definida por.
    \begin{equation}
        SHF_{link} = \sum_{i=1}^{B} \frac{S_{i}^{free}}{N}~ln\frac{N}{S_{i}^{free}}
    \end{equation}
    Donde \(S^{free}\) representa la cantidad de FS libres en el enlace y \textit{B} la cantiadad de bloques de FS libres. Para calcular el SHF de la red se calcula el promedio de los valores calculados en todos los enlaces.
    \begin{equation}
        SHF_{red} = \sum_{i=1}^{\left | E \right |} \frac{SHF_{link - i}}{{\left | E \right |}}
    \end{equation}
    \item \textit{Bandwidth Fragmentation Ratio} o Relación de Fragmentación de ancho de banda o  (BFR)\cite{zhang2013bandwidth}: Representa el índice de fragmentación de los recursos de la red. El BFR de un link se define como:
    \begin{equation}
        BFR_{link} = 1 - \frac{MaxBlock()}{S^{free}}
    \end{equation}
    Donde \textit{MaxBlock()} es el tamaño del mayor bloque de FS libres y \(S^{free}\) es la sumatoria total de FS libres en el enlace \textit{link}.
    El BFR de la red podemos calcular de la siguiente manera.
    \begin{equation}
       BFR_{red} = \sum_{i=1}^{\left | E \right |} \frac{BFR_{link - i}}{{\left | E \right |}}
    \end{equation}
    \item \textit{Maximum Slot Index} o Mayor Índice de FS utilizado (MSI): Este valor indica el valor del índice de FS más alto que está siendo utilizado dentro de un enlace.
    Para calcular el MSI de la red, se halla el índice máximo usado en todos los enlaces de la red y se calcula el promedio:
    \begin{equation}
        MSI_{red} = \sum_{i=1}^{\left | E \right |} \frac{MSI_{link - i}}{{\left | E \right |}}
    \end{equation}

    Donde  \(MSI_{link} - i\) es el mayor índice utilizado en el enlace i.
    \item Consecutividad del espectro (CE)\cite{wang2012spectrum}: Este valor refleja la alineación de los FS en un camino en particular, para obtener el valor para un camino en particular aplicamos la siguiente fórmula.
    \begin{equation}
        CE = \frac{Joins}{Bloques} \times \frac{S_{free}}{  N  }
    \end{equation}
    Donde \textit{Joins} se calcula como la cantidad total de bloques de dos FS libres adyacentes distintos dentro del enlace, \textit{Bloques} es la cantidad de bloques de FS libres en el enlace y \(S_{free}\) es la cantidad de FS libres en el enlace.
    Y para calcular la consecutividad en una red se calculan previamente todos los caminos de dos enlaces en la red, y luego se calcula el valor de la consecutividad para todas esas rutas y al final se halla el promedio.
    \begin{equation}
        CE{red} = \sum_{i=1}^{\left | K \right |} \frac{CE_{link - i}}{{\left | K \right |}}
    \end{equation}
    donde \(CE_{link - i}\) representa la consecutividad de la ruta de dos enlaces i calculada previamente, y K es la cantidad de rutas de dos enlaces que existen en la red. 
    \item Utilización de la Red: Se define como el cociente entre la sumatoria de todas las FS ocupados con la cantidad total de FS.
    \begin{equation}
        Uso_{link} =  \frac{sum(i)}{N}
    \end{equation}
    Donde \(sum(i)\) es la cantidad de FS utilizados en el enlace \(i\) y \(N\) es la cantidad total de FS en el enlace. 
    
    \begin{equation}
        Uso_{red} = \sum_{i=1}^{\left | E \right |} \frac{Uso_{link -i}}{\left | E \right |}
    \end{equation}
    
    \item Acumulación de FS bloqueados: Valor que muestra la sumatoria de las FS requeridos por demandas bloqueadas en las D demandas anteriores al periodo de tiempo actual.
    \begin{equation}
        FSB = \sum_{i=1}^{D} S_{i}^{block}
    \end{equation}
    Donde \(S_{i}^{block}\) es la cantidad de FS solicitados por la demanda bloqueada i.
\end{itemize}

\begin{figure}[!h]
    \centering
    \includegraphics[width=1\textwidth]{capitulos/img/ejemploMetricasCompleto4.png}
    \caption{Ejemplo de métricas de fragmentación}
    \label{fig:ejemploMetricas}
\end{figure}

En la figura \ref{fig:ejemploMetricas} se puede ver un ejemplo de una topología con 4 conexiones activas, el estado de sus enlaces y el cálculo de cada una de las métricas de fragmentación explicadas anteriormente. 

\section{Obtención de datos para el entrenamiento}
Se utilizó un simulador de redes EON \cite{davalos2019spectrum} para la generación del conjunto de datos a ser utilizados para el entrenamiento de la red neuronal.

Para esto se utilizaron dos topologías de red: USNET y NSFNET, en donde por cada topología se realizaron 50 simulaciones con volumen de  tráfico variable en la misma simulación, para lograr que la cantidad de conexiones activas no permanezca constante durante la simulación.  La variación de tráfico usada fue la que se ve en la figura \ref{fig:traficos}-a, el proceso se realizó durante 1210 unidades de tiempo para cada simulación, generando un total de 121.000 registros.

Una vez generados los datos, los mismos fueron preprocesados, de forma a obtener los valores de la probabilidad de bloqueo que deseamos estimar.

El preprocesamiento consiste en el cálculo del índice de bloqueo en base a la ecuación \ref{eq:ecuacion_ib}, donde para un tiempo t, utilizando una ventana de 10 unidades de tiempo hacia delante, para cada instante podemos obtener una mirada hacia delante de posibles bloqueos, \(FSB_{i}\) es la sumatoria de \textit{FS} bloqueados en el tiempo \textit{i} y \(FSD_{i}\) es la sumatoria de \textit{FS} demandados en el tiempo \textit{i}.

\begin{equation} \label{eq:ecuacion_ib}
        PB_{t} = \frac{\sum_{i=t}^{t+T}FSB_{i}}{\sum_{i=t}^{t+T}FSD_{i}}
\end{equation}

Además, una vez separados los datos y debido a la diferencia de rangos de valores, los mismos son normalizados, usando la siguiente fórmula.
\begin{equation}
    n = x - \frac{train_{mean}}{train_{std}}
\end{equation}
Donde x es el valor que queremos normalizar, \(train_{mean}\) la media de valores y \(train_{std}\) es la desviación estándar. 

\begin{figure}
    \centering
    \includegraphics[width=1\textwidth]{capitulos/img/trafico_1_a.png}
    \includegraphics[width=1\textwidth]{capitulos/img/trafico_2_b.png}
    \includegraphics[width=1\textwidth]{capitulos/img/trafico_3_c.png}
    \caption{Volumen de tráfico variado utilizado}
    \label{fig:traficos}
\end{figure}

\section{Herramientas Utilizadas}

En esta sección se presentan las herramientas utilizadas para la implementación del método propuesto en este trabajo, seleccionadas luego de realizar numerosas ejecuciones de prueba para obtener los valores de los parámetros del modelo, asi como la prueba de concepto en sí.

Para todo el proceso de \textit{Machine Learning} se utilizó la plataforma de código abierto llamada \textit{\textbf{TensorFlow}} \cite{tensowFlow} en el lenguaje de programación \textit{\textbf{Python}}. En la creación y entrenamiento de redes neuronales se utilizó la API de alto nivel incluida en la plataforma \textit{\textbf{Tensorflow}} denominada \textit{\textbf{Keras}}, la cual permite la creación de modelos de aprendizaje automático de forma rápida y sencilla.

Elegimos Tensorflow como herramienta principal debido a que es una plataforma de código abierto orientado a \textit{Machine Learning}. Cuenta con un ecosistema completo de herramientas y librerías que facilitan la creación y desarrollo de aplicaciones de aprendizaje automático, además cuenta con una extensa y muy completa documentación.

\section{Modelado}

Para entrenar los datos recolectados se utilizó un modelo con una capa de entrada de 7 neuronas, dos capas ocultas densamente conectadas, cada una con 64 y 32 neuronas respectivamente, y una capa de salida que devuelve un único valor continuo.
Para todas las capas la función de activación utilizada fue la RELU (Ver sección 3.2.2).

\section{Entrenamiento}
Para el entrenamiento del modelo, se creó un conjunto de datos de entrenamiento utilizando el 70\% de los datos recolectados de forma aleatoria.  

El 20\% de los datos de entrenamiento se utilizó como el conjunto de validación. Una técnica utilizada en el procedimiento es el  llamado parada temprana o \textit{Early Stopping}, el cual mediante el monitoreo del rendimiento del entrenamiento nos permite detenernos una vez que el error de validación aumente de forma sostenida, de forma así evitar un sobre entrenamiento.

El modelo se entrena como máximo por 1000 épocas, el cual se detiene automáticamente cuando el valor del error de validación deja de mejorar. La figura \ref{fig:errorEpocas} muestra la evolución del error al pasar las épocas.

Los parámetros comparados son el error de entrenamiento o \textit{Train Error}, el cuál es el error obtenido durante la fase de entrenamiento del modelo y el error de validación o \textit{Val Error}, que se obtiene en la fase de validación.

Por cada época el \textit{Val Error} es comparado con el  \textit{Train Error},esto hasta que se determina que ya no existe mejora, sino que el error de validación se mantiene o aumenta su valor con respecto al punto de parada.


\begin{figure}[H]
    \centering
    \includegraphics[width=9cm]{capitulos/img/graficoErrorEs.png}
    \caption{Evolución del error a través de épocas}
    \label{fig:errorEpocas}
\end{figure}

\section{Pruebas de predicción}
Para comprobar la efectividad de nuestro modelo se procedió a tomar el 30\% de datos restantes que no fueron incluidos en el entrenamiento y realizar predicciones, como ya conocemos el valor de la probabilidad de bloqueo que se busca predecir podemos calcular el error absoluto medio (MAE) y el error cuadrático medio (MSE). La tabla \ref{table:resultadosPrediccionTabla} muestra los resultados obtenidos, siendo éstos muy satisfactorios.

\begin{table}[H]
\centering
    \caption{Tabla de resultados en pruebas de predicción}
    \begin{tabular}{|l|l|}
        \hline
        MAE & 0.0239 \\ \hline
        MSE & 0.002  \\ \hline
    \end{tabular}
    \label{table:resultadosPrediccionTabla}
\end{table}
\newpage
\begin{figure}[H]
    \centering
    \begin{tabular}{@{}c@{}}
        \includegraphics[width=1\textwidth]{capitulos/img/trafico_1.png}\\
        \small (a) Tráfico de volumen variable
    \end{tabular}
    \begin{tabular}{@{}c@{}}
        \includegraphics[width=1\textwidth]{capitulos/img/grafico_prediccion_usnet.PNG}\\
        \small (b) Gráfico de valores predichos por la red neuronal
    \end{tabular}
    \caption{Utilización de la red y predicciones}
    \label{fig:utilizacionPredicciones}
\end{figure}
En la Figura \ref{fig:utilizacionPredicciones} - a se observa la variación de la utilización de la red, al realizarse una simulación con volumen de tráfico variable sin realizar procesos de desfragmentación. En la parte b de la misma figura se observa la curva de valores predichos con las técnicas presentadas junto a los bloqueos dados en la misma simulación. Puede observarse cómo este valor acompaña la variación de la utilización de la red, y también toma valores más altos en periodos de tiempo en que la frecuencia de bloqueos se hace más alta. 

\chapter{Pruebas y Resultados Obtenidos}

Las pruebas experimentales se realizaron sobre el simulador de redes ópticas elásticas multinúcleo desarrollado en \cite{davalos2019spectrum}, utilizando la topología USNET. El objetivo fundamental de esta evaluación consiste en validar la efectividad del método propuesto basado en aprendizaje automático para la predicción proactiva de fragmentación y el disparo adaptativo de procesos de desfragmentación en redes MC-EON.

\section{Configuración Experimental}

\subsection{Topología y Parámetros de Red}

Las simulaciones se ejecutaron sobre la topología USNET, configurada con los siguientes parámetros operacionales:

\begin{itemize}
    \item \textbf{Topología}: USNET (24 nodos, 43 enlaces bidireccionales)
    \item \textbf{Algoritmo de asignación}: Múltiples Cores con restricciones de crosstalk
    \item \textbf{Nivel de crosstalk}: $1.0 \times 10^{-10}$ (umbral crítico para interferencia entre núcleos)
    \item \textbf{Tiempo de simulación}: 20,000 unidades de tiempo
    \item \textbf{Tipo de tráfico}: Unicast con generación aleatoria de demandas
    \item \textbf{Variación de carga}: Patrón montaña de 10 niveles con transiciones suaves
\end{itemize}

\subsection{Escenarios de Carga Evaluados}

Para evaluar el comportamiento del sistema bajo diferentes condiciones de estrés, se diseñaron tres escenarios de carga con características distintivas. Cada escenario representa un rango operacional diferente de la red, desde condiciones de baja utilización hasta estados de alta congestión. Los escenarios se configuraron mediante los niveles de carga discretos presentados en el capítulo anterior, con las siguientes distribuciones:

\textbf{Escenario 1 - Carga Baja:}
\begin{itemize}
    \item Rango de Erlangs: 800 - 3,000
    \item Característica: Red con amplia disponibilidad de recursos espectrales
    \item Probabilidad esperada de bloqueo: Baja ($< 1\%$)
\end{itemize}

\textbf{Escenario 2 - Carga Media:}
\begin{itemize}
    \item Rango de Erlangs: 1,200 - 3,400
    \item Característica: Utilización moderada con fragmentación progresiva
    \item Probabilidad esperada de bloqueo: Media (1\% - 2\%)
\end{itemize}

\textbf{Escenario 3 - Carga Alta:}
\begin{itemize}
    \item Rango de Erlangs: 2,000 - 4,000
    \item Característica: Alta congestión con fragmentación severa
    \item Probabilidad esperada de bloqueo: Alta ($> 3\%$)
\end{itemize}

La Tabla~\ref{tab:escenarios_carga} resume la distribución temporal de cada nivel de carga en los tres escenarios evaluados.

\begin{table}[htbp]
\centering
\caption{Configuración de escenarios de carga evaluados}
\label{tab:escenarios_carga}
\begin{tabular}{lccc}
\hline
\textbf{Nivel} & \textbf{Escenario 1 (Erlangs)} & \textbf{Escenario 2 (Erlangs)} & \textbf{Escenario 3 (Erlangs)} \\
\hline
NIVEL\_1 & 800 & 1,200 & 2,000 \\
NIVEL\_2 & 1,000 & 1,400 & 2,200 \\
NIVEL\_3 & 1,200 & 1,600 & 2,400 \\
NIVEL\_4 & 1,400 & 1,900 & 2,600 \\
NIVEL\_5 & 1,600 & 2,200 & 2,800 \\
NIVEL\_6 & 1,900 & 2,500 & 3,000 \\
NIVEL\_7 & 2,300 & 2,600 & 3,200 \\
NIVEL\_8 & 2,600 & 3,100 & 3,400 \\
NIVEL\_10 & 3,000 & 3,400 & 4,000 \\
\hline
\end{tabular}
\end{table}

Cada escenario se ejecutó con patrón de carga tipo montaña, presentando aproximadamente 15 transiciones entre niveles a lo largo de las 20,000 unidades de tiempo. Este patrón permite evaluar el comportamiento adaptativo de los algoritmos ante variaciones realistas de tráfico.

\section{Métodos Comparados}

Para evaluar el desempeño del método propuesto, se implementaron tres estrategias de desfragmentación que representan diferentes paradigmas de gestión de recursos en redes ópticas elásticas:

\subsection{Método Propuesto (MP): Desfragmentación Adaptativa con ML}

El método propuesto implementa una estrategia de tres niveles basada en la predicción del índice de fragmentación BFR en horizonte $t+1000$ mediante el modelo Gradient Boosting entrenado. La estrategia adaptativa opera según los siguientes criterios:

\begin{itemize}
    \item \textbf{Período de warm-up}: 1,000 unidades de tiempo iniciales sin desfragmentación para permitir estabilización de la red
    
    \item \textbf{Nivel 1 (BFR predicho $< 0.20$)}: 
    \begin{itemize}
        \item Acción: NO desfragmentar
        \item Justificación: La red se encuentra en estado saludable, la desfragmentación generaría costos innecesarios
    \end{itemize}
    
    \item \textbf{Nivel 2 ($0.20 \leq$ BFR predicho $< 0.46$)}:
    \begin{itemize}
        \item Acción: Desfragmentación preventiva
        \item Intervalo posterior: 1,500 unidades de tiempo
        \item Justificación: Fragmentación moderada detectada, intervención preventiva con frecuencia reducida
    \end{itemize}
    
    \item \textbf{Nivel 3 (BFR predicho $\geq 0.46$)}:
    \begin{itemize}
        \item Acción: Desfragmentación reactiva
        \item Intervalo posterior: 800 unidades de tiempo
        \item Justificación: Fragmentación crítica anticipada, requiere intervenciones frecuentes
    \end{itemize}
\end{itemize}

Esta estrategia permite ajustar dinámicamente la frecuencia de desfragmentaciones según el estado predicho de la red, anticipando situaciones críticas con 1,000 demandas de antelación y evitando intervenciones innecesarias en estados saludables.

\subsection{Método de Referencia 1 (MR1): Desfragmentación Periódica por Tiempo Fijo}

Este método representa el enfoque tradicional más simple, ejecutando desfragmentaciones a intervalos temporales constantes independientemente del estado de la red:

\begin{itemize}
    \item \textbf{Intervalo fijo}: 1,000 unidades de tiempo
    \item \textbf{Característica}: Estrategia proactiva sin adaptación al estado de la red
    \item \textbf{Ventaja}: Simplicidad de implementación, comportamiento predecible
    \item \textbf{Limitación}: No considera el estado real de fragmentación, puede generar intervenciones innecesarias o insuficientes
\end{itemize}

\subsection{Método de Referencia 2 (MR2): Desfragmentación por Umbral de BFR}

Este método implementa una estrategia reactiva basada en el monitoreo continuo del índice de fragmentación actual:

\begin{itemize}
    \item \textbf{Criterio de disparo}: BFR actual $\geq 0.46$
    \item \textbf{Característica}: Estrategia reactiva basada en mediciones en tiempo real
    \item \textbf{Ventaja}: Responde directamente al estado de fragmentación observado
    \item \textbf{Limitación}: No anticipa situaciones críticas, actúa cuando la fragmentación ya es severa
\end{itemize}

La comparación entre estos tres métodos permite evaluar el valor agregado de la predicción mediante Machine Learning frente a estrategias tradicionales proactivas y reactivas.

\section{Objetivos de Optimización}

El problema de desfragmentación en redes MC-EON presenta un carácter multiobjetivo inherente, donde la optimización de un aspecto puede deteriorar otros. En este contexto, se consideran dos objetivos globales medidos al final de cada simulación, cuya minimización simultánea representa el desafío fundamental:

\subsection{Objetivo 1: Cantidad de Bloqueos (BL)}

\begin{equation}
BL = \sum_{i=1}^{N_{demandas}} \mathbf{1}_{\mathrm{bloqueada}}(i)
\end{equation}

donde $\mathbf{1}_{\mathrm{bloqueada}}(i)$ es la función indicadora que vale 1 si la demanda $i$ fue bloqueada y 0 en caso contrario. Este objetivo cuantifica el impacto negativo de la fragmentación sobre la capacidad de la red para aceptar nuevas conexiones. La probabilidad de bloqueo global se calcula como:

\begin{equation}
P_{bloqueo} = \frac{BL}{N_{demandas}} \times 100\%
\end{equation}

\subsection{Objetivo 2: Cantidad de Reconfiguraciones (RC)}

\begin{equation}
RC = \sum_{j=1}^{N_{desfrag}} |C_j|
\end{equation}

donde $N_{desfrag}$ representa el número de procesos de desfragmentación ejecutados y $|C_j|$ denota la cantidad de conexiones reconfiguradas durante el proceso $j$. Este objetivo refleja el costo operacional de la desfragmentación, considerando que cada reconfiguración implica:

\begin{itemize}
    \item Interrupción temporal del servicio
    \item Consumo de recursos computacionales
    \item Posible degradación transitoria de QoS
    \item Overhead de señalización en el plano de control
\end{itemize}

\subsection{Métricas de Evaluación Multiobjetivo}

Dado que BL y RC representan objetivos conflictivos (mayor frecuencia de desfragmentación reduce BL pero incrementa RC), se emplean métricas específicas para optimización multiobjetivo:

\subsubsection{Soluciones en el Frente de Pareto (SFP)}

Una solución $s_1$ domina a otra solución $s_2$ (denotado $s_1 \prec s_2$) si y solo si:

\begin{equation}
BL(s_1) \leq BL(s_2) \land RC(s_1) \leq RC(s_2) \land (BL(s_1) < BL(s_2) \lor RC(s_1) < RC(s_2))
\end{equation}

El conjunto de soluciones no dominadas constituye el Frente de Pareto. La métrica SFP cuantifica el número de configuraciones de cada método que pertenecen a este frente óptimo.

\subsubsection{Cobertura de Pareto (CP)}

Para comparar pares de métodos $A$ y $B$, se define la métrica de cobertura:

\begin{equation}
C(A,B) = \frac{|\{s_B \in S_B : \exists s_A \in S_A, s_A \prec s_B\}|}{|S_B|}
\end{equation}

donde $S_A$ y $S_B$ son los conjuntos de soluciones de los métodos $A$ y $B$ respectivamente. Esta métrica indica qué proporción de las soluciones del método $B$ son dominadas por al menos una solución del método $A$.

\section{Resultados Experimentales}

\subsection{Escenario 1: Carga Baja (800 - 3,000 Erlangs)}

Este escenario representa condiciones operacionales favorables donde la red dispone de recursos espectrales abundantes. Los resultados obtenidos para cada método se presentan en la Tabla~\ref{tab:resultados_escenario1}.

\begin{table}[htbp]
\centering
\caption{Resultados comparativos - Escenario 1 (Carga Baja)}
\label{tab:resultados_escenario1}
\begin{tabular}{lccc}
\hline
\textbf{Métrica} & \textbf{MP (ML)} & \textbf{MR1 (Periódico)} & \textbf{MR2 (Umbral BFR)} \\
\hline
\textbf{Demandas totales} & 99,720 & 100,090 & 99,601 \\
\textbf{Bloqueos (BL)} & \textbf{685} & 730 & 717 \\
\textbf{Prob. bloqueo} & \textbf{0.687\%} & 0.729\% & 0.720\% \\
\textbf{Desfragmentaciones} & 13 & 19 & 11 \\
\textbf{Desfrag. preventivas} & 13 (100\%) & 19 (100\%) & - \\
\textbf{Desfrag. reactivas} & 0 (0\%) & - & 11 (100\%) \\
\hline
\end{tabular}
\end{table}

\textbf{Análisis del comportamiento por niveles de carga:}

En este escenario de baja congestión, los tres métodos presentan desempeño similar en términos de probabilidad de bloqueo, con valores inferiores al 1\%. El método propuesto (MP) logra una ligera ventaja con 0.687\% de bloqueos, ejecutando 13 desfragmentaciones preventivas basadas en predicciones que anticipan correctamente fragmentación moderada.

El análisis detallado por nivel de carga revela que:

\begin{itemize}
    \item Los niveles NIVEL\_1 a NIVEL\_4 (800-1,400 Erlangs) presentan bloqueos nulos para todos los métodos, indicando amplia disponibilidad espectral.
    \item A partir de NIVEL\_5 (1,600 Erlangs) comienzan a aparecer bloqueos esporádicos (0.03\%-0.05\%).
    \item El nivel crítico NIVEL\_10 (3,000 Erlangs) concentra el 65\% de los bloqueos totales, con probabilidades entre 1.77\%-1.97\%.
\end{itemize}

Notablemente, el método propuesto ejecutó todas sus desfragmentaciones en modo preventivo, evidenciando que el modelo de predicción identificó correctamente que el horizonte $t+1000$ permanecería por debajo del umbral crítico (BFR $< 0.46$) durante la mayor parte de la simulación.

\subsection{Escenario 2: Carga Media (1,200 - 3,400 Erlangs)}

Este escenario intermedio representa condiciones operacionales típicas donde la red experimenta fragmentación progresiva. La Tabla~\ref{tab:resultados_escenario2} presenta los resultados comparativos.

\begin{table}[htbp]
\centering
\caption{Resultados comparativos - Escenario 2 (Carga Media)}
\label{tab:resultados_escenario2}
\begin{tabular}{lccc}
\hline
\textbf{Métrica} & \textbf{MP (ML)} & \textbf{MR1 (Periódico)} & \textbf{MR2 (Umbral BFR)} \\
\hline
\textbf{Demandas totales} & 100,023 & 100,184 & 100,066 \\
\textbf{Bloqueos (BL)} & \textbf{1,473} & 1,561 & 1,397 \\
\textbf{Prob. bloqueo} & 1.473\% & 1.558\% & \textbf{1.397\%} \\
\textbf{Desfragmentaciones} & 16 & 19 & 15 \\
\textbf{Desfrag. preventivas} & 9 (56.3\%) & 19 (100\%) & 8 (53.3\%) \\
\textbf{Desfrag. reactivas} & 7 (43.8\%) & - & 7 (46.7\%) \\
\hline
\end{tabular}
\end{table}

\textbf{Comportamiento adaptativo del método propuesto:}

En este escenario de carga media, el método ML demuestra su capacidad adaptativa alternando entre intervenciones preventivas (56.3\%) y reactivas (43.8\%). El modelo de predicción identificó correctamente:

\begin{itemize}
    \item 9 situaciones donde BFR predicho se mantendría en rango moderado (0.20-0.46), aplicando desfragmentación preventiva con intervalos largos (1,500t)
    \item 7 situaciones donde BFR predicho superaría el umbral crítico (0.46), aplicando desfragmentación reactiva con intervalos cortos (800t)
\end{itemize}

El análisis por niveles de carga muestra:

\begin{itemize}
    \item NIVEL\_1 a NIVEL\_3 (1,200-1,600 Erlangs): Bloqueos prácticamente nulos (0\%-0.04\%)
    \item NIVEL\_4 a NIVEL\_6 (1,900-2,500 Erlangs): Aparición gradual de fragmentación (0.05\%-0.78\%)
    \item NIVEL\_7 a NIVEL\_8 (2,600-3,100 Erlangs): Fragmentación moderada (1.23\%-2.95\%)
    \item NIVEL\_10 (3,400 Erlangs): Concentración del 59\% de bloqueos totales (3.34\%)
\end{itemize}

Aunque MR2 logra ligeramente menos bloqueos en este escenario particular (1.397\%), el método propuesto mantiene un balance más equilibrado entre bloqueos y reconfiguraciones, ejecutando solo una desfragmentación adicional.

\subsection{Escenario 3: Carga Alta (2,000 - 4,000 Erlangs)}

Este escenario crítico evalúa el comportamiento bajo condiciones de alta congestión y fragmentación severa. Los resultados se presentan en la Tabla~\ref{tab:resultados_escenario3}.

\begin{table}[htbp]
\centering
\caption{Resultados comparativos - Escenario 3 (Carga Alta)}
\label{tab:resultados_escenario3}
\begin{tabular}{lccc}
\hline
\textbf{Métrica} & \textbf{MP (ML)} & \textbf{MR1 (Periódico)} & \textbf{MR2 (Umbral BFR)} \\
\hline
\textbf{Demandas totales} & 100,174 & 99,896 & 100,066 \\
\textbf{Bloqueos (BL)} & \textbf{2,986} & 3,262 & 2,873 \\
\textbf{Prob. bloqueo} & 2.981\% & 3.265\% & \textbf{2.871\%} \\
\textbf{Desfragmentaciones} & 18 & 19 & 18 \\
\textbf{Desfrag. preventivas} & 7 (38.9\%) & 19 (100\%) & 7 (38.9\%) \\
\textbf{Desfrag. reactivas} & 11 (61.1\%) & - & 11 (61.1\%) \\
\textbf{Consultas t+1000} & 18 & - & - \\
\textbf{Tasa éxito predicción} & 100\% & - & - \\
\hline
\end{tabular}
\end{table}

\textbf{Análisis del comportamiento en carga alta:}

Este escenario representa el caso más demandante, con probabilidades de bloqueo superiores al 2.8\% en todos los métodos. El método propuesto exhibe características distintivas:

\begin{itemize}
    \item \textbf{Adaptación reactiva predominante}: 61.1\% de desfragmentaciones ejecutadas en modo reactivo (intervalo 800t), reflejando la detección precisa de fragmentación crítica por el predictor
    \item \textbf{Anticipación efectiva}: Las 18 consultas al predictor ML en horizonte $t+1000$ fueron exitosas (100\%), evidenciando la robustez del modelo Gradient Boosting entrenado
    \item \textbf{Balance bloqueos-reconfiguraciones}: Con solo 18 desfragmentaciones (vs. 19 de MR1), logra 2.981\% de bloqueos, cercano al mejor resultado (2.871\% de MR2)
\end{itemize}

Distribución de bloqueos por nivel de carga:

\begin{itemize}
    \item NIVEL\_1 (2,000 Erlangs): Bloqueos mínimos (0.02\%-0.05\%)
    \item NIVEL\_2 a NIVEL\_5 (2,200-2,800 Erlangs): Incremento gradual (0.35\%-1.51\%)
    \item NIVEL\_6 a NIVEL\_8 (3,000-3,400 Erlangs): Fragmentación severa (2.05\%-5.32\%)
    \item NIVEL\_10 (4,000 Erlangs): Estado crítico, 50\% de bloqueos totales (5.77\%-6.66\%)
\end{itemize}

El comportamiento del método ML en este escenario valida su capacidad de adaptación: ante fragmentación crítica predicha, intensifica las desfragmentaciones (intervalos de 800t), mientras que en períodos de recuperación transitoria aplica estrategia preventiva (intervalos de 1,500t).

\subsection{Variante con Doble Predictor (3 Niveles)}

Adicionalmente, se evaluó una variante del método propuesto que emplea dos modelos predictores simultáneos para implementar estrategia de tres niveles:

\begin{itemize}
    \item \textbf{Predictor 1}: Umbral BFR = 0.20 (detección de fragmentación emergente)
    \item \textbf{Predictor 2}: Umbral BFR = 0.46 (detección de fragmentación crítica)
\end{itemize}

Los resultados de esta variante para los tres escenarios se presentan en la Tabla~\ref{tab:resultados_doble_predictor}.

\begin{table}[htbp]
\centering
\caption{Resultados con estrategia de doble predictor (3 niveles)}
\label{tab:resultados_doble_predictor}
\begin{tabular}{lcccc}
\hline
\textbf{Escenario} & \textbf{Bloqueos} & \textbf{Prob. (\%)} & \textbf{Desfrag.} & \textbf{Distribución} \\
\hline
\multirow{2}{*}{\textbf{Carga Baja}} & \multirow{2}{*}{717} & \multirow{2}{*}{0.720} & \multirow{2}{*}{11} & Nivel 1 (sin acción): 15.4\% \\
 & & & & Nivel 2 (largo): 84.6\% \\
 & & & & Nivel 3 (corto): 0.0\% \\
\hline
\multirow{2}{*}{\textbf{Carga Media}} & \multirow{2}{*}{1,397} & \multirow{2}{*}{1.397} & \multirow{2}{*}{15} & Nivel 1 (sin acción): 6.3\% \\
 & & & & Nivel 2 (largo): 50.0\% \\
 & & & & Nivel 3 (corto): 43.8\% \\
\hline
\multirow{2}{*}{\textbf{Carga Alta}} & \multirow{2}{*}{2,873} & \multirow{2}{*}{2.871} & \multirow{2}{*}{18} & Nivel 1 (sin acción): 0.0\% \\
 & & & & Nivel 2 (largo): 38.9\% \\
 & & & & Nivel 3 (corto): 61.1\% \\
\hline
\end{tabular}
\end{table}

Esta variante demuestra capacidad para identificar estados de red saludable (Nivel 1) donde la desfragmentación es innecesaria, logrando resultados competitivos con menos intervenciones totales en escenarios de carga baja y media.

\section{Análisis Comparativo Multiobjetivo}

\subsection{Soluciones en el Frente de Pareto}

Para cada escenario de carga, se identificaron las soluciones no dominadas considerando simultáneamente minimización de bloqueos (BL) y reconfiguraciones (RC). La Tabla~\ref{tab:frente_pareto} presenta la distribución de soluciones en el Frente de Pareto por método.

\begin{table}[htbp]
\centering
\caption{Distribución de soluciones en el Frente de Pareto}
\label{tab:frente_pareto}
\begin{tabular}{lcccc}
\hline
\textbf{Escenario} & \textbf{MP (ML)} & \textbf{MR1 (Periódico)} & \textbf{MR2 (Umbral)} & \textbf{Total} \\
\hline
Carga Baja & \textbf{3} & 2 & 2 & 7 \\
Carga Media & \textbf{3} & 1 & 2 & 6 \\
Carga Alta & \textbf{3} & 2 & 3 & 8 \\
\hline
\textbf{Total General} & \textbf{9 (42.9\%)} & 5 (23.8\%) & 7 (33.3\%) & 21 \\
\hline
\end{tabular}
\end{table}

\textbf{Interpretación de resultados:}

El método propuesto (MP) contribuye con 9 soluciones no dominadas de las 21 identificadas (42.9\%), superando significativamente a ambos métodos de referencia. Esta predominancia se mantiene consistente en los tres escenarios de carga, evidenciando:

\begin{itemize}
    \item \textbf{Robustez}: El método ML genera soluciones Pareto-óptimas independientemente del nivel de congestión
    \item \textbf{Diversidad}: Las configuraciones adaptativas (preventiva/reactiva) exploran eficientemente el espacio de soluciones
    \item \textbf{Superioridad multiobjetivo}: El balance bloqueos-reconfiguraciones supera a estrategias estáticas (periódica) y puramente reactivas (umbral)
\end{itemize}

\subsection{Cobertura de Pareto}

Para cuantificar la dominancia relativa entre métodos, se calculó la métrica de cobertura $C(A,B)$ para cada par de estrategias. La Tabla~\ref{tab:cobertura_pareto} presenta los resultados agregados.

\begin{table}[htbp]
\centering
\caption{Cobertura de Pareto entre métodos (agregado 3 escenarios)}
\label{tab:cobertura_pareto}
\begin{tabular}{llccl}
\hline
\textbf{Método A} & \textbf{Método B} & \textbf{C(A,B)} & \textbf{C(B,A)} & \textbf{Conclusión} \\
\hline
MP & MR1 (Periódico) & \textbf{0.567} & 0.222 & \makecell[l]{MP domina 56.7\% de MR1 \\ MR1 domina 22.2\% de MP} \\
\hline
MP & MR2 (Umbral) & \textbf{0.476} & 0.333 & \makecell[l]{MP domina 47.6\% de MR2 \\ MR2 domina 33.3\% de MP} \\
\hline
MR1 & MR2 & 0.333 & 0.400 & \makecell[l]{MR1 domina 33.3\% de MR2 \\ MR2 domina 40.0\% de MR1} \\
\hline
\end{tabular}
\end{table}

\textbf{Análisis de cobertura:}

Los resultados de cobertura revelan patrones distintivos:

\begin{enumerate}
    \item \textbf{MP vs. MR1 (Periódico)}:
    \begin{itemize}
        \item El método ML domina 56.7\% de las soluciones del método periódico
        \item Esta ventaja significativa (diferencia de 34.5 puntos porcentuales) evidencia la superioridad de la adaptación dinámica sobre intervalos fijos
        \item Las pocas soluciones donde MR1 domina a MP corresponden a configuraciones específicas en carga baja donde la simplicidad del enfoque periódico resulta suficiente
    \end{itemize}
    
    \item \textbf{MP vs. MR2 (Umbral)}:
    \begin{itemize}
        \item El método ML domina 47.6\% de las soluciones reactivas por umbral
        \item La diferencia menor respecto a MR1 (14.3 puntos) indica que el enfoque reactivo es más competitivo que el puramente proactivo
        \item Sin embargo, MP mantiene ventaja neta de 14.3 puntos, demostrando valor de la anticipación mediante predicción
    \end{itemize}
    
    \item \textbf{MR1 vs. MR2}:
    \begin{itemize}
        \item Cobertura equilibrada (33.3\% vs. 40.0\%), sin claro dominante
        \item Confirma que ambos enfoques tradicionales presentan limitaciones complementarias
    \end{itemize}
\end{enumerate}

\subsection{Eficiencia en Uso de Recursos}

La Figura~\ref{fig:eficiencia_recursos} (conceptual) ilustraría el trade-off entre probabilidad de bloqueo y número de desfragmentaciones para los tres métodos en cada escenario. El método propuesto ocuparía consistentemente regiones del espacio de soluciones que logran:

\begin{itemize}
    \item Menor probabilidad de bloqueo con igual o menor número de desfragmentaciones (dominancia pura)
    \item Probabilidades de bloqueo competitivas con significativamente menos desfragmentaciones (eficiencia superior)
\end{itemize}

\section{Validación del Horizonte de Predicción}

El horizonte temporal $t+1000$ fue seleccionado mediante análisis de predictibilidad previo al entrenamiento del modelo. Los resultados experimentales validan esta elección:

\begin{table}[htbp]
\centering
\caption{Validación del horizonte de predicción $t+1000$}
\label{tab:validacion_horizonte}
\begin{tabular}{lccc}
\hline
\textbf{Métrica} & \textbf{Escenario 1} & \textbf{Escenario 2} & \textbf{Escenario 3} \\
\hline
Consultas predictor & 13 & 16 & 18 \\
Consultas exitosas & 13 & 16 & 18 \\
Errores predictor & 0 & 0 & 0 \\
\textbf{Tasa éxito} & \textbf{100.0\%} & \textbf{100.0\%} & \textbf{100.0\%} \\
\hline
\end{tabular}
\end{table}

La tasa de éxito del 100\% en las 47 consultas agregadas (13+16+18) valida que:

\begin{enumerate}
    \item El modelo Gradient Boosting captura efectivamente patrones de evolución de fragmentación
    \item El horizonte $t+1000$ proporciona anticipación suficiente sin exceder el límite de predictibilidad
    \item Las características de entrada (BFR, SHF, SC, GM, ASFR3D, UD) contienen información predictiva robusta
\end{enumerate}

\section{Discusión de Resultados}

\subsection{Superioridad del Método Propuesto}

Los resultados experimentales confirman la hipótesis de que modelos de aprendizaje automático pueden predecir momentos óptimos para ejecutar desfragmentación en redes MC-EON. Las evidencias específicas incluyen:

\begin{enumerate}
    \item \textbf{Desempeño multiobjetivo superior}: El método ML genera 42.9\% de las soluciones Pareto-óptimas, superando a MR1 (23.8\%) y MR2 (33.3\%)
    
    \item \textbf{Adaptación dinámica efectiva}: La distribución preventiva/reactiva se ajusta automáticamente al nivel de carga:
    \begin{itemize}
        \item Carga baja: 100\% preventivo (máxima eficiencia)
        \item Carga media: 56.3\% preventivo / 43.8\% reactivo (transición)
        \item Carga alta: 38.9\% preventivo / 61.1\% reactivo (máxima reactividad)
    \end{itemize}
    
    \item \textbf{Reducción de bloqueos}: En escenario de alta carga, MP logra 8.5\% menos bloqueos que MR1 con número similar de desfragmentaciones
    
    \item \textbf{Eficiencia de predicción}: 100\% de éxito en 47 predicciones $t+1000$ valida robustez del modelo
\end{enumerate}

\subsection{Ventajas de la Estrategia Adaptativa}

La capacidad de ajustar dinámicamente intervalos de desfragmentación según predicciones representa la innovación fundamental del método propuesto:

\begin{itemize}
    \item \textbf{Evita intervenciones innecesarias}: En carga baja, identifica estados saludables (BFR $< 0.20$) y suprime desfragmentaciones, reduciendo overhead operacional
    
    \item \textbf{Anticipa congestión}: La predicción en horizonte $t+1000$ permite activar modo reactivo (intervalo 800t) \emph{antes} de que BFR alcance niveles críticos, minimizando bloqueos
    
    \item \textbf{Optimiza uso de recursos}: Aplica intervenciones preventivas espaciadas (1,500t) en fragmentación moderada, balanceando eficacia y costo
\end{itemize}

\subsection{Comparación con Métodos Tradicionales}

\textbf{MR1 (Periódico)}:
\begin{itemize}
    \item Simplicidad operacional, pero rigidez ante variaciones de carga
    \item En carga alta, intervalo 1,000t resulta insuficiente (3.265\% bloqueos vs. 2.981\% de MP)
    \item En carga baja, genera intervenciones excesivas (19 desfrag. vs. 13 de MP)
\end{itemize}

\textbf{MR2 (Umbral)}:
\begin{itemize}
    \item Estrategia puramente reactiva: actúa cuando fragmentación es observable
    \item Carece de anticipación: no previene estados críticos, solo responde a ellos
    \item Competitivo en carga media-baja, pero en alta carga la ausencia de anticipación limita efectividad
\end{itemize}

\chapter{ Conclusiones y Trabajos Futuros }

En las redes ópticas elásticas, la constante asignación y liberación de recursos en forma dinámica puede dar lugar al problema conocido como "fragmentación del ancho de banda". Este problema es crítico ya que la presencia de bloques aislados de ancho de banda dentro del dominio del espectro deja a los mismos inutilizables ante futuras solicitudes de conexiones, debido a que los mismos no se encuentran alineados y contiguos,

Un enfoque utilizado para combatir la fragmentación son los procesos de desfragmentación, que consisten en el retiro y posterior re-ruteo de un sub-conjunto de conexiones existentes, con el objetivo de consolidar los espacios disponibles en grandes bloques contiguos y continuos que puedan ser utilizados para futuras solicitudes de conexiones.

El problema analizado en este trabajo es el de ¿Cuándo Reconfigurar?, es decir, buscar el momento adecuado para disparar el proceso de desfragmentación, ya que desfragmentaciones muy frecuentes o muy distantes en el tiempo pueden hacer que estos no sean muy eficientes.

Este trabajo presenta un enfoque con desfragmentación para tráfico dinámico de redes ópticas elásticas por medio de un disparador inteligente utilizando técnicas de \textit{Machine Learning}. En su implementación se utilizó un enfoque de aprendizaje supervisado, con un modelo de redes neuronales artificiales para la predicción de futuros bloqueos y utilizando algunas características para medir el estado de fragmentación de la red y el uso de la misma.

El método propuesto de disparo recibe el estado actual de la red o ``características'' para cada instante de tiempo, con estas características y el entrenamiento previo del modelo, obtenemos una predicción de la probabilidad de futuros bloqueos, para una ventana de 10 unidades de tiempo hacia delante.

Para evaluar la eficiencia del método propuesto de disparo se consideraron tres escenarios diferentes, con un volumen de tráfico variable, utilizando las topologías NSFNET y USNET. Los objetivos a optimizar fueron: 
\begin{itemize}
    \item La cantidad de bloqueos obtenidos (BL)
    \item La cantidad de reconfiguraciones al final de cada instancia de prueba (RC)
\end{itemize}

\section{Conclusiones Experimentales}
Se realizaron pruebas experimentales a fin de comparar nuestro método de disparo contra otros dos presentes en la literatura científica. El método de desfragmentanción periódica es una estrategia ampliamente utilizada, la cual consiste en realizar el proceso de desfragmentación cada cierto periodo fijo de tiempo y el disparo por medio de métricas, la cual considera en realizar el proceso de desfragmetación en base al valor actual de la métrica, para las pruebas de este método se utilizó la métrica de fragmentación BFR.

Para comparar los resultados obtenidos en relación a los objetivos citados anteriormente (BL y RC), se utilizaron dos métricas de desempeño para optimización multi-objetivo.

\begin{enumerate}
    \item Número de soluciones en el Frente Pareto (SFP).
    \item Cobertura Pareto (CP).
\end{enumerate}

Como resultado de la comparación de los métodos en base a los objetivos expuestos previamente, se concluye que el método propuesto es mejor ya que consigue en la mayoría de los escenarios mejores resultados,
minimizando los valores obtenidos para BL y RC. Considerando la métrica SFP se obtiene que constituye el 52.9\% de soluciones no dominadas y en el caso de CP se logró en un 67\% del total de comparaciones resultados favorables a nuestro método.

\section{Aportes}
Los aportes del presente trabajo son:
\begin{enumerate}[label=\arabic*)]
    \item Un análisis bibliográfico sobre el problema relacionado al periodo de tiempo en el que el proceso de desfragmentación será ejecutado.
    \item Una investigación y recopilación de métricas de fragmentación las cuales son utilizadas como características necesarias para la predicción de la probabilidad de bloqueo por parte del modelo entrenado.
    \item Como principal aporte se diseñó un algoritmo que realiza el preprocesamiento de datos, entrenamiento del modelo y predicción de probabilidades de bloqueo utilizando técnicas de \textit{Machine Learning}.
    \item Pruebas experimentales utilizando en conjunto un simulador de redes EON y el modelo entrenado a fin de evaluar la eficiencia de nuestro método propuesto. Se realizaron comparaciones contra otros dos mecanismos de disparo, disparo periódico en tiempos fijos y disparo basado en la métrica BFR, teniendo resultados favorables para nuestro método en relación a la minimización de la cantidad de bloqueos y reconfiguraciones.
\end{enumerate}

\section{Trabajos Futuros}
\begin{itemize}
    \item Aplicar el modelo de disparo inteligente del proceso de desfragmentación propuesto en este trabajo a redes ópticas elásticas que utilizan otras tecnologías o técnicas, como las redes EON con multiplexación por división de espacios o \textit{Space División Multiplexing} (SDM).
    
    EL SDM es utilizado en redes con múltiples núcleos, por lo que sería interesante realizar un análisis de la eficiencia del modelo en este tipo de redes.
    
    \item Proponer algoritmos de disparo del proceso de desfragmentación utilizando métodos estadísticos, tal como la regresión logística binaria (RLB), la cual se utiliza cuando se desea conocer la relación entre una variable dependiente binaria y una o más variables independientes o explicativas, las cuales pueden ser cuantitativas y/o cualitativas.
    
    El objetivo de la RLB es obtener una estimación ajustada de la probabilidad de ocurrencia de un evento a partir de una o más variables independientes.
    
    \item Otro enfoque posible es utilizar programación genética, el cual consiste en una metodología basada en algoritmos evolutivos e inspirada en la evolución biológica para desarrollar programas que realicen ciertas tareas, por ejemplo, realizar disparo del proceso de desfragmentación en el mejor momento.
    
    Es una técnica de aprendizaje automático utilizada para optimizar una población de programas de acuerdo a una función de ajuste o \textit{fitness function} que evalúala capacidad de cada programa de realizar la tarea.
\end{itemize}
	
	
	


\appendix   % inician los apendices de tu tesis
% los cap'itulos que incluyas a partir de aqu'i aparecen
% como ap'endices
\include{./anexo-1}
% estos comandos generan la bilbiograf'ia

\printbibliography

\end{document}
