\newpage
\chapter*{Lista de Símbolos\hfill}
\addcontentsline{toc}{chapter}{Lista de Símbolos}
\begin{tabbing}

\color{white}Zero \=\color{white}One \=\color{white}Twoooooo \=\color{white}Thre\\

\textbf{Parámetros de red y espectro}\\
\textit{N} \>\>\> Cantidad de \textit{FS} en un enlace.\\
\textit{\(FS_{i}\)} \>\>\> \textit{Frecuency Slot} de índice \textit{i}.\\
\textit{\(FS_{i+1}\)} \>\>\> \textit{Frecuency Slot} de índice \(\textit{i} + 1\).\\
|\textit{E}| \>\>\> Cantidad de enlaces de la red.\\
\textit{C} \>\>\> Número de núcleos por enlace.\\
\textit{L} \>\>\> Longitud física del enlace óptico.\\
\textit{\(S_{free}\)} \>\>\> Cantidad de \textit{FS} libres en un enlace o núcleo.\\
\textit{\(S^{occupied}\)} \>\>\> Cantidad de \textit{FS} ocupados.\\
\textit{\(S^{small}\)} \>\>\> Suma de \textit{FS} libres en bloques menores a 5 slots.\\
\textit{Bloques} \>\>\> Cantidad de bloques de ranuras libres en un enlace. \\
\textit{B} \>\>\> Cantidad de bloques libres.\\
\textit{G} \>\>\> Cantidad total de gaps.\\
\textit{\(g_{i}\)} \>\>\> Tamaño del gap libre \textit{i}.\\
\textit{\(s_{min}\)} \>\>\> Índice del primer \textit{FS} ocupado.\\
\textit{\(s_{max}\)} \>\>\> Índice del último \textit{FS} ocupado.\\
\\
\textbf{Parámetros de fibra multinúcleo}\\
\textit{\(\Lambda_{i,j}\)} \>\>\> Core pitch - Distancia entre los núcleos \textit{i} y \textit{j}.\\
\textit{\(N_{i}\)} \>\>\> Cantidad de núcleos adyacentes al núcleo \textit{i}.\\
\textit{k} \>\>\> Coeficiente de acoplamiento.\\
\textit{r} \>\>\> Radio de curvatura.\\
\textit{\(\beta\)} \>\>\> Constante de propagación.\\
\textit{\(h_{i,j}\)} \>\>\> XT por unidad de longitud entre núcleos \textit{i} y \textit{j}.\\
\textit{\(XT_{i}\)} \>\>\> Crosstalk total que impacta al núcleo \textit{i}.\\
\textit{\(XT_{TH}\)} \>\>\> Umbral de crosstalk admisible.\\
\\
\textbf{Métricas de fragmentación}\\
\textit{\(BFR_{link}\)} \>\>\> Relación de Fragmentación de ancho de banda de un enlace.\\
\textit{\(BFR_{link-i}\)} \>\>\> Relación de Fragmentación de ancho de banda del enlace \textit{i}. \\
\textit{\(BFR_{red}\)} \>\>\> Relación de Fragmentación de ancho de banda de la red. \\
\textit{\(BFR_F\)} \>\>\> BFR futuro predicho.\\
\textit{\(BFR_{F_{MIN}}\)} \>\>\> Umbral mínimo de BFR.\\
\textit{\(BFR_{F_{MAX}}\)} \>\>\> Umbral crítico de BFR.\\
\textit{MaxBlock()} \>\>\> Tamaño del mayor bloque de \textit{FS} libres.\\
\\
\textit{\(SHF_{link}\)} \>\>\> Entropía de Shannon del enlace.\\
\textit{\(SHF_{link-i}\)} \>\>\> Entropía de Shannon del enlace \textit{i}. \\
\textit{\(SHF_{red}\)} \>\>\> Entropía de Shannon de la red. \\
\\
\textit{MSI} \>\>\> Índice de slot máximo utilizado.\\
\textit{\(MSI_{link-i}\)} \>\>\> Índice de slot máximo utilizado del enlace \textit{i}. \\
\textit{\(MSI_{red}\)} \>\>\> Índice de slot máximo utilizado de la red. \\
\\
\textit{\(n_{1}, n_{2}\)} \>\>\> Parámetros de tamaños típicos de demanda (Golden Metric).\\
\textit{avg} \>\>\> Promedio en cálculo de Golden Metric.\\
\textit{\(\epsilon\)} \>\>\> Valor pequeño para evitar división por cero.\\
\textit{a, b} \>\>\> Variables acumulativas en Golden Metric.\\
\textit{\(F_{spatial}\)} \>\>\> Factor de peso espacial.\\
\textit{\(D_{active}\)} \>\>\> Número de demandas activas en la red.\\
\\
\textbf{Utilización de red}\\
\textit{Uso} \>\>\> Porcentaje de utilización de la red.\\
\textit{\(U_{core}\)} \>\>\> Utilización de un núcleo.\\
\textit{\(U_{max}\)} \>\>\> Utilización máxima entre todos los núcleos.\\
\textit{\(U_{min}\)} \>\>\> Utilización mínima entre todos los núcleos.\\
\\
\textbf{Métricas de desempeño}\\
\textit{PB} \>\>\> Probabilidad de Bloqueo.\\
\textit{\(PB_{th}\)} \>\>\> Umbral para disparar el proceso de desfragmentación.\\
\textit{\(P_{bloqueo}\)} \>\>\> Probabilidad de bloqueo global.\\
\textit{BL} \>\>\> Cantidad de Bloqueos.\\
\textit{\(B_t\)} \>\>\> Bloqueos en tiempo \textit{t}.\\
\textit{RC} \>\>\> Cantidad de Reconfiguraciones.\\
\textit{\(N_{demandas}\)} \>\>\> Número total de demandas.\\
\textit{\(N_{desfrag}\)} \>\>\> Número de procesos de desfragmentación.\\
\textit{\(|C_j|\)} \>\>\> Cantidad de conexiones reconfiguradas en proceso \textit{j}.\\
\textit{\(\mathbf{1}_{bloqueada}(i)\)} \>\>\> Función indicadora de bloqueo para demanda \textit{i}.\\
\textit{SFP} \>\>\> Número de soluciones en el Frente Pareto.\\
\textit{CP} \>\>\> Cobertura Pareto.\\
\textit{\(TBI_t\)} \>\>\> Tasa de Bloqueo Instantánea en tiempo \textit{t}.\\
\textit{\(D_{processed,t}\)} \>\>\> Demandas procesadas hasta tiempo \textit{t}.\\
\\
\textbf{Parámetros de simulación y tiempo}\\
\textit{t} \>\>\> Tiempo de simulación.\\
\textit{\(\widehat{BFR}_{t+1000}\)} \>\>\> BFR predicho a 1000 demandas futuras.\\
\\
\textbf{Parámetros de aprendizaje automático}\\
\textit{M} \>\>\> Número total de iteraciones (Gradient Boosting).\\
\textit{\(h_{m}(x)\)} \>\>\> m-ésimo aprendiz débil.\\
\textit{\(\gamma_{m}\)} \>\>\> Coeficiente de peso asociado.\\
\textit{\(\nu\)} \>\>\> Tasa de aprendizaje.\\
\textit{\(p_i\)} \>\>\> Probabilidad predicha para la muestra \textit{i}.\\
\textit{\(X_{scaled}\)} \>\>\> Características escaladas.\\
\textit{IQR} \>\>\> Rango intercuartílico.\\
\\
\textbf{Métodos comparados}\\
\textit{MP} \>\>\> Método Propuesto.\\
\textit{MR1} \>\>\> Método de Referencia 1 (Desfragmentación Periódica).\\
\textit{MR2} \>\>\> Método de Referencia 2 (Desfragmentación por Umbral).\\
\textit{SD} \>\>\> Sin Desfragmentación.\\

\end{tabbing}
