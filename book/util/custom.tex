%%%
%%% Conjuto de funciones utilitarias
%%% autor: Maximiliano Báez
%%% fecha: 25/08/2014
%%%

\usepackage{tablefootnote} %para utilizar \footnote{}
\usepackage{amssymb}
\renewcommand{\thefootnote}{\arabic{footnote}}

%%Para la construcción de la lista de simbolos
% Macro for 'List of Symbols', 'List of Notations' etc...
\def\listofsymbols{
    \newpage
\chapter*{Lista de Símbolos\hfill}
\addcontentsline{toc}{chapter}{Lista de Símbolos}
\begin{tabbing}

%\textit{HM} \indent \textit{High-slot Mark}\\
%\textit{MS} \indent \textit{Maximon Slot Index} - Mayor Índice de Ranura\\
\color{white}Zero \=\color{white}One \=\color{white}Twoooooo \=\color{white}Thre\\

\textit{HM} \>\>\>\textit{High-slot Mark}.\\
\textit{\(HM_{max}\)} \>\>\> HM máximo.\\

\textit{\(Ent_{link}\)} \>\>\> Entropía del enlace.\\
\textit{\(Ent_{link-i}\)} \>\>\> Entropía del enlace \textit{i}. \\
\textit{\(Ent_{red}\)} \>\>\> Entropía de la red. \\
\textit{N} \>\>\> Cantidad de \textit{FS} en un enlace.\\
\textit{\(FS_{i}\)} \>\>\> \textit{Frecuency Slot} de índice \textit{i}.\\
\textit{\(FS_{i+1}\)} \>\>\> \textit{Frecuency Slot} de índice \(\textit{i} + 1\).\\
|\textit{E}| \>\>\> Cantidad de enlaces de la red.\\

\textit{\(SHF_{link}\)} \>\>\> Entropía de Shannon del enlace.\\
\textit{\(SHF_{link-i}\)} \>\>\> Entropía de Shannon del enlace \textit{i}. \\
\textit{\(SHF_{red}\)} \>\>\> Entropía de Shannon de la red. \\
\textit{\(S_{free}\)} \>\>\> Cantidad de \textit{FS} libres en un enlace.\\

\textit{\(BFR_{link}\)} \>\>\> Relación de Fragmentación de ancho de banda de un enlace.\\
\textit{\(BFR_{link-i}\)} \>\>\> Relación de Fragmentación de ancho de banda del enlace \textit{i}. \\
\textit{\(BFR_{red}\)} \>\>\> Relación de Fragmentación de ancho de banda de la red. \\
\textit{MaxBlock()} \>\>\> Tamaño del mayor bloque de \textit{FS} bloquados.\\

\textit{MSI} \>\>\> Índice de slot máximo utilizado.\\
\textit{\(MSI_{link-i}\)} \>\>\> Índice de slot máximo utilizado del enlace \textit{i}. \\
\textit{\(MSI_{red}\)} \>\>\> Índice de slot máximo utilizado de la red. \\

\textit{\(CE_{link}\)} \>\>\> Consecutividad del espectro.\\
\textit{\(CE_{red}\)} \>\>\> Consecutividad del espectro de la red. \\
\textit{Joins} \>\>\> Cantidad total de bloques de dos ranuras libres adyacentes\\ 
                \>\>\>  distintos dentro de un enlace.\\
\textit{Bloques} \>\>\> Cantidad de bloques de ranuras libres en un enlace. \\
\textit{K} \>\>\> Cantidad de rutas de dos enlaces en la red.\\

\textit{Uso} \>\>\> Porcentaje de utilización de la red.\\
\textit{sum(i)} \>\>\> Cantidad de \textit{FS} utilizadas en el enlace \textit{i}.\\

\textit{FSB} \>\>\> Acumulación de \textit{FS} bloqueados.\\
\textit{\(S_{i}^{block}\)} \>\>\> Cantidad de \textit{FS} solicitadas por la demanda bloqueada \textit{i}.\\
\textit{D} \>\>\> Cantidad de demandas.\\
\textit{T} \>\>\> Ventada de tiempo seleccionada para el calculo de \textit{IB}.\\

\textit{\(PB_{t}\)} \>\>\> Índice de bloqueo para el tiempo \textit{t}.\\
\textit{\(FSD_{i}\)} \>\>\> Cantidad de \textit{FS} demandadas en el tiempo \textit{t}.\\ 

\textit{\(train_{mean}\)} \>\>\> Media de valores.\\
\textit{\(train_{stf}\)} \>\>\> Desviación estándar.\\

\textit{MAE} \>\>\> Error Absoluto Medio.\\
\textit{MSE} \>\>\> Error Cuadrático Medio.\\

\textit{PB} \>\>\> Probabilidad de Bloqueo.\\
\textit{\(PB_{th}\)} \>\>\> Umbral para disparar el proceso de desfragmentación.\\

\textit{BL} \>\>\> Cantidad de Bloqueos.\\
\textit{RC} \>\>\> Cantidad de Reconfiguraciones.\\
\textit{SFP} \>\>\> Número de soluciones en el Frente Pareto.\\
\textit{CP} \>\>\> Cobertura Pareto.\\
\textit{MP} \>\>\> Método Propuesto.
% \begin{tabular}{p{0.1\linewidth}p{0.9\linewidth}}
% \textit{HM} & \textit{High-slot Mark} \\
% \textit{MS} & \textit{Maximun Slot Index} - Mayor Índice de Ranura \\
% \end{tabular}
\end{tabbing}

    \clearpage{}
}
\def\newsymbol #1: #2#3{$#1$ \> \parbox{5in}{#2 \dotfill \pageref{#3}}\\}
\def\addsymbol#1{\label{#1}}

% Para las imagenes en grilla
% custom commands
\newcommand{\foreign}[1]{{\it #1}}
\DeclareMathOperator*{\argmax}{arg\,max}
%\algsetup{}
\algsetup{
    indent=4em,
    linenosize=\small,
    linenodelimiter=.
}

\usepackage{amsmath}

%% se utilizan para referenciar figuras, tablas, secciones y algoritmos
\newcommand{\figref}[1]{Figura \ref{#1}}
\newcommand{\tabref}[1]{Tabla \ref{#1}}
\newcommand{\secref}[1]{sección \ref{#1}}
\newcommand{\algref}[1]{Algoritmo \ref{#1}}


%Traducción al español del paquete algorithmic%
\floatname{algorithm}{Algoritmo}
\renewcommand{\listalgorithmname}{Lista de algoritmos}
\renewcommand{\algorithmicrequire}{\textbf{Entrada:}}
\renewcommand{\algorithmicensure}{\textbf{Salida:}}
\renewcommand{\algorithmicend}{\textbf{fin}}
\renewcommand{\algorithmicif}{\textbf{si}}
\renewcommand{\algorithmicthen}{\textbf{entonces}}
\renewcommand{\algorithmicelse}{\textbf{si no}}
\renewcommand{\algorithmicelsif}{\algorithmicelse,\ \algorithmicif}
\renewcommand{\algorithmicendif}{\algorithmicend\ \algorithmicif}
\renewcommand{\algorithmicfor}{\textbf{para}}
\renewcommand{\algorithmicforall}{\textbf{para todo}}
\renewcommand{\algorithmicdo}{\textbf{hacer}}
\renewcommand{\algorithmicendfor}{\algorithmicend\ \algorithmicfor}
\renewcommand{\algorithmicwhile}{\textbf{mientras}}
\renewcommand{\algorithmicendwhile}{\algorithmicend\ \algorithmicwhile}
\renewcommand{\algorithmicloop}{\textbf{repetir}}
\renewcommand{\algorithmicendloop}{\algorithmicend\ \algorithmicloop}
\renewcommand{\algorithmicrepeat}{\textbf{repetir}}
\renewcommand{\algorithmicuntil}{\textbf{hasta que}}
\renewcommand{\algorithmicprint}{\textbf{imprimir}}
\renewcommand{\algorithmicreturn}{\textbf{retorna}}
\renewcommand{\algorithmictrue}{\textbf{cierto }}
\renewcommand{\algorithmicfalse}{\textbf{falso }}
\renewcommand{\algorithmiccomment}{\textbf{comentario : }}
