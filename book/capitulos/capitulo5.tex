\chapter{Pruebas y Resultados Obtenidos}

Las pruebas experimentales se realizaron sobre el simulador de redes ópticas elásticas multinúcleo desarrollado en \cite{davalos2019spectrum}, utilizando la topología USNET. El objetivo fundamental de esta evaluación consiste en validar la efectividad del método propuesto basado en aprendizaje automático para la predicción proactiva de fragmentación y el disparo adaptativo de procesos de desfragmentación en redes MC-EON.

\section{Configuración Experimental}

\subsection{Topología y Parámetros de Red}

Las simulaciones se ejecutaron sobre la topología USNET, configurada con los siguientes parámetros operacionales:

\begin{itemize}
    \item \textbf{Topología}: USNET (24 nodos, 43 enlaces bidireccionales)
    \item \textbf{Algoritmo de asignación}: Múltiples Cores con restricciones de crosstalk
    \item \textbf{Nivel de crosstalk}: $1.0 \times 10^{-10}$ (umbral crítico para interferencia entre núcleos)
    \item \textbf{Tiempo de simulación}: 20,000 unidades de tiempo
    \item \textbf{Tipo de tráfico}: Unicast con generación aleatoria de demandas
    \item \textbf{Variación de carga}: Patrón montaña de 10 niveles con transiciones suaves
\end{itemize}

\subsection{Escenarios de Carga Evaluados}

Para evaluar el comportamiento del sistema bajo diferentes condiciones de estrés, se diseñaron tres escenarios de carga con características distintivas. Cada escenario representa un rango operacional diferente de la red, desde condiciones de baja utilización hasta estados de alta congestión. Los escenarios se configuraron mediante los niveles de carga discretos presentados en el capítulo anterior, con las siguientes distribuciones:

\textbf{Escenario 1 - Carga Baja:}
\begin{itemize}
    \item Rango de Erlangs: 800 - 3,000
    \item Característica: Red con amplia disponibilidad de recursos espectrales
    \item Probabilidad esperada de bloqueo: Baja ($< 1\%$)
\end{itemize}

\textbf{Escenario 2 - Carga Media:}
\begin{itemize}
    \item Rango de Erlangs: 1,200 - 3,400
    \item Característica: Utilización moderada con fragmentación progresiva
    \item Probabilidad esperada de bloqueo: Media (1\% - 2\%)
\end{itemize}

\textbf{Escenario 3 - Carga Alta:}
\begin{itemize}
    \item Rango de Erlangs: 2,000 - 4,000
    \item Característica: Alta congestión con fragmentación severa
    \item Probabilidad esperada de bloqueo: Alta ($> 3\%$)
\end{itemize}

La Tabla~\ref{tab:escenarios_carga} resume la distribución temporal de cada nivel de carga en los tres escenarios evaluados.

\begin{table}[htbp]
\centering
\caption{Configuración de escenarios de carga evaluados}
\label{tab:escenarios_carga}
\begin{tabular}{lccc}
\hline
\textbf{Nivel} & \textbf{Escenario 1 (Erlangs)} & \textbf{Escenario 2 (Erlangs)} & \textbf{Escenario 3 (Erlangs)} \\
\hline
NIVEL\_1 & 800 & 1,200 & 2,000 \\
NIVEL\_2 & 1,000 & 1,400 & 2,200 \\
NIVEL\_3 & 1,200 & 1,600 & 2,400 \\
NIVEL\_4 & 1,400 & 1,900 & 2,600 \\
NIVEL\_5 & 1,600 & 2,200 & 2,800 \\
NIVEL\_6 & 1,900 & 2,500 & 3,000 \\
NIVEL\_7 & 2,300 & 2,600 & 3,200 \\
NIVEL\_8 & 2,600 & 3,100 & 3,400 \\
NIVEL\_10 & 3,000 & 3,400 & 4,000 \\
\hline
\end{tabular}
\end{table}

Cada escenario se ejecutó con patrón de carga tipo montaña, presentando aproximadamente 15 transiciones entre niveles a lo largo de las 20,000 unidades de tiempo. Este patrón permite evaluar el comportamiento adaptativo de los algoritmos ante variaciones realistas de tráfico.

\section{Métodos Comparados}

Para evaluar el desempeño del método propuesto, se implementaron tres estrategias de desfragmentación que representan diferentes paradigmas de gestión de recursos en redes ópticas elásticas:

\subsection{Método Propuesto (MP): Desfragmentación Adaptativa con ML}

El método propuesto implementa una estrategia de tres niveles basada en la predicción del índice de fragmentación BFR en horizonte $t+1000$ mediante el modelo Gradient Boosting entrenado. La estrategia adaptativa opera según los siguientes criterios:

\begin{itemize}
    \item \textbf{Período de warm-up}: 1,000 unidades de tiempo iniciales sin desfragmentación para permitir estabilización de la red
    
    \item \textbf{Nivel 1 (BFR predicho $< 0.20$)}: 
    \begin{itemize}
        \item Acción: NO desfragmentar
        \item Justificación: La red se encuentra en estado saludable, la desfragmentación generaría costos innecesarios
    \end{itemize}
    
    \item \textbf{Nivel 2 ($0.20 \leq$ BFR predicho $< 0.46$)}:
    \begin{itemize}
        \item Acción: Desfragmentación preventiva
        \item Intervalo posterior: 1,500 unidades de tiempo
        \item Justificación: Fragmentación moderada detectada, intervención preventiva con frecuencia reducida
    \end{itemize}
    
    \item \textbf{Nivel 3 (BFR predicho $\geq 0.46$)}:
    \begin{itemize}
        \item Acción: Desfragmentación reactiva
        \item Intervalo posterior: 800 unidades de tiempo
        \item Justificación: Fragmentación crítica anticipada, requiere intervenciones frecuentes
    \end{itemize}
\end{itemize}

Esta estrategia permite ajustar dinámicamente la frecuencia de desfragmentaciones según el estado predicho de la red, anticipando situaciones críticas con 1,000 demandas de antelación y evitando intervenciones innecesarias en estados saludables.

\subsection{Método de Referencia 1 (MR1): Desfragmentación Periódica por Tiempo Fijo}

Este método representa el enfoque tradicional más simple, ejecutando desfragmentaciones a intervalos temporales constantes independientemente del estado de la red:

\begin{itemize}
    \item \textbf{Intervalo fijo}: 1,000 unidades de tiempo
    \item \textbf{Característica}: Estrategia proactiva sin adaptación al estado de la red
    \item \textbf{Ventaja}: Simplicidad de implementación, comportamiento predecible
    \item \textbf{Limitación}: No considera el estado real de fragmentación, puede generar intervenciones innecesarias o insuficientes
\end{itemize}

\subsection{Método de Referencia 2 (MR2): Desfragmentación por Umbral de BFR}

Este método implementa una estrategia reactiva basada en el monitoreo continuo del índice de fragmentación actual:

\begin{itemize}
    \item \textbf{Criterio de disparo}: BFR actual $\geq 0.46$
    \item \textbf{Característica}: Estrategia reactiva basada en mediciones en tiempo real
    \item \textbf{Ventaja}: Responde directamente al estado de fragmentación observado
    \item \textbf{Limitación}: No anticipa situaciones críticas, actúa cuando la fragmentación ya es severa
\end{itemize}

La comparación entre estos tres métodos permite evaluar el valor agregado de la predicción mediante Machine Learning frente a estrategias tradicionales proactivas y reactivas.

\section{Objetivos de Optimización}

El problema de desfragmentación en redes MC-EON presenta un carácter multiobjetivo inherente, donde la optimización de un aspecto puede deteriorar otros. En este contexto, se consideran dos objetivos globales medidos al final de cada simulación, cuya minimización simultánea representa el desafío fundamental:

\subsection{Objetivo 1: Cantidad de Bloqueos (BL)}

\begin{equation}
BL = \sum_{i=1}^{N_{demandas}} \mathbf{1}_{\mathrm{bloqueada}}(i)
\end{equation}

donde $\mathbf{1}_{\mathrm{bloqueada}}(i)$ es la función indicadora que vale 1 si la demanda $i$ fue bloqueada y 0 en caso contrario. Este objetivo cuantifica el impacto negativo de la fragmentación sobre la capacidad de la red para aceptar nuevas conexiones. La probabilidad de bloqueo global se calcula como:

\begin{equation}
P_{bloqueo} = \frac{BL}{N_{demandas}} \times 100\%
\end{equation}

\subsection{Objetivo 2: Cantidad de Reconfiguraciones (RC)}

\begin{equation}
RC = \sum_{j=1}^{N_{desfrag}} |C_j|
\end{equation}

donde $N_{desfrag}$ representa el número de procesos de desfragmentación ejecutados y $|C_j|$ denota la cantidad de conexiones reconfiguradas durante el proceso $j$. Este objetivo refleja el costo operacional de la desfragmentación, considerando que cada reconfiguración implica:

\begin{itemize}
    \item Interrupción temporal del servicio
    \item Consumo de recursos computacionales
    \item Posible degradación transitoria de QoS
    \item Overhead de señalización en el plano de control
\end{itemize}

\subsection{Métricas de Evaluación Multiobjetivo}

Dado que BL y RC representan objetivos conflictivos (mayor frecuencia de desfragmentación reduce BL pero incrementa RC), se emplean métricas específicas para optimización multiobjetivo:

\subsubsection{Soluciones en el Frente de Pareto (SFP)}

Una solución $s_1$ domina a otra solución $s_2$ (denotado $s_1 \prec s_2$) si y solo si:

\begin{equation}
BL(s_1) \leq BL(s_2) \land RC(s_1) \leq RC(s_2) \land (BL(s_1) < BL(s_2) \lor RC(s_1) < RC(s_2))
\end{equation}

El conjunto de soluciones no dominadas constituye el Frente de Pareto. La métrica SFP cuantifica el número de configuraciones de cada método que pertenecen a este frente óptimo.

\subsubsection{Cobertura de Pareto (CP)}

Para comparar pares de métodos $A$ y $B$, se define la métrica de cobertura:

\begin{equation}
C(A,B) = \frac{|\{s_B \in S_B : \exists s_A \in S_A, s_A \prec s_B\}|}{|S_B|}
\end{equation}

donde $S_A$ y $S_B$ son los conjuntos de soluciones de los métodos $A$ y $B$ respectivamente. Esta métrica indica qué proporción de las soluciones del método $B$ son dominadas por al menos una solución del método $A$.

\section{Resultados Experimentales}

\subsection{Escenario 1: Carga Baja (800 - 3,000 Erlangs)}

Este escenario representa condiciones operacionales favorables donde la red dispone de recursos espectrales abundantes. Los resultados obtenidos para cada método se presentan en la Tabla~\ref{tab:resultados_escenario1}.

\begin{table}[htbp]
\centering
\caption{Resultados comparativos - Escenario 1 (Carga Baja)}
\label{tab:resultados_escenario1}
\begin{tabular}{lccc}
\hline
\textbf{Métrica} & \textbf{MP (ML)} & \textbf{MR1 (Periódico)} & \textbf{MR2 (Umbral BFR)} \\
\hline
\textbf{Demandas totales} & 99,720 & 100,090 & 99,601 \\
\textbf{Bloqueos (BL)} & \textbf{685} & 730 & 717 \\
\textbf{Prob. bloqueo} & \textbf{0.687\%} & 0.729\% & 0.720\% \\
\textbf{Desfragmentaciones} & 13 & 19 & 11 \\
\textbf{Desfrag. preventivas} & 13 (100\%) & 19 (100\%) & - \\
\textbf{Desfrag. reactivas} & 0 (0\%) & - & 11 (100\%) \\
\hline
\end{tabular}
\end{table}

\textbf{Análisis del comportamiento por niveles de carga:}

En este escenario de baja congestión, los tres métodos presentan desempeño similar en términos de probabilidad de bloqueo, con valores inferiores al 1\%. El método propuesto (MP) logra una ligera ventaja con 0.687\% de bloqueos, ejecutando 13 desfragmentaciones preventivas basadas en predicciones que anticipan correctamente fragmentación moderada.

El análisis detallado por nivel de carga revela que:

\begin{itemize}
    \item Los niveles NIVEL\_1 a NIVEL\_4 (800-1,400 Erlangs) presentan bloqueos nulos para todos los métodos, indicando amplia disponibilidad espectral.
    \item A partir de NIVEL\_5 (1,600 Erlangs) comienzan a aparecer bloqueos esporádicos (0.03\%-0.05\%).
    \item El nivel crítico NIVEL\_10 (3,000 Erlangs) concentra el 65\% de los bloqueos totales, con probabilidades entre 1.77\%-1.97\%.
\end{itemize}

Notablemente, el método propuesto ejecutó todas sus desfragmentaciones en modo preventivo, evidenciando que el modelo de predicción identificó correctamente que el horizonte $t+1000$ permanecería por debajo del umbral crítico (BFR $< 0.46$) durante la mayor parte de la simulación.

\subsection{Escenario 2: Carga Media (1,200 - 3,400 Erlangs)}

Este escenario intermedio representa condiciones operacionales típicas donde la red experimenta fragmentación progresiva. La Tabla~\ref{tab:resultados_escenario2} presenta los resultados comparativos.

\begin{table}[htbp]
\centering
\caption{Resultados comparativos - Escenario 2 (Carga Media)}
\label{tab:resultados_escenario2}
\begin{tabular}{lccc}
\hline
\textbf{Métrica} & \textbf{MP (ML)} & \textbf{MR1 (Periódico)} & \textbf{MR2 (Umbral BFR)} \\
\hline
\textbf{Demandas totales} & 100,023 & 100,184 & 100,066 \\
\textbf{Bloqueos (BL)} & \textbf{1,473} & 1,561 & 1,397 \\
\textbf{Prob. bloqueo} & 1.473\% & 1.558\% & \textbf{1.397\%} \\
\textbf{Desfragmentaciones} & 16 & 19 & 15 \\
\textbf{Desfrag. preventivas} & 9 (56.3\%) & 19 (100\%) & 8 (53.3\%) \\
\textbf{Desfrag. reactivas} & 7 (43.8\%) & - & 7 (46.7\%) \\
\hline
\end{tabular}
\end{table}

\textbf{Comportamiento adaptativo del método propuesto:}

En este escenario de carga media, el método ML demuestra su capacidad adaptativa alternando entre intervenciones preventivas (56.3\%) y reactivas (43.8\%). El modelo de predicción identificó correctamente:

\begin{itemize}
    \item 9 situaciones donde BFR predicho se mantendría en rango moderado (0.20-0.46), aplicando desfragmentación preventiva con intervalos largos (1,500t)
    \item 7 situaciones donde BFR predicho superaría el umbral crítico (0.46), aplicando desfragmentación reactiva con intervalos cortos (800t)
\end{itemize}

El análisis por niveles de carga muestra:

\begin{itemize}
    \item NIVEL\_1 a NIVEL\_3 (1,200-1,600 Erlangs): Bloqueos prácticamente nulos (0\%-0.04\%)
    \item NIVEL\_4 a NIVEL\_6 (1,900-2,500 Erlangs): Aparición gradual de fragmentación (0.05\%-0.78\%)
    \item NIVEL\_7 a NIVEL\_8 (2,600-3,100 Erlangs): Fragmentación moderada (1.23\%-2.95\%)
    \item NIVEL\_10 (3,400 Erlangs): Concentración del 59\% de bloqueos totales (3.34\%)
\end{itemize}

Aunque MR2 logra ligeramente menos bloqueos en este escenario particular (1.397\%), el método propuesto mantiene un balance más equilibrado entre bloqueos y reconfiguraciones, ejecutando solo una desfragmentación adicional.

\subsection{Escenario 3: Carga Alta (2,000 - 4,000 Erlangs)}

Este escenario crítico evalúa el comportamiento bajo condiciones de alta congestión y fragmentación severa. Los resultados se presentan en la Tabla~\ref{tab:resultados_escenario3}.

\begin{table}[htbp]
\centering
\caption{Resultados comparativos - Escenario 3 (Carga Alta)}
\label{tab:resultados_escenario3}
\begin{tabular}{lccc}
\hline
\textbf{Métrica} & \textbf{MP (ML)} & \textbf{MR1 (Periódico)} & \textbf{MR2 (Umbral BFR)} \\
\hline
\textbf{Demandas totales} & 100,174 & 99,896 & 100,066 \\
\textbf{Bloqueos (BL)} & \textbf{2,986} & 3,262 & 2,873 \\
\textbf{Prob. bloqueo} & 2.981\% & 3.265\% & \textbf{2.871\%} \\
\textbf{Desfragmentaciones} & 18 & 19 & 18 \\
\textbf{Desfrag. preventivas} & 7 (38.9\%) & 19 (100\%) & 7 (38.9\%) \\
\textbf{Desfrag. reactivas} & 11 (61.1\%) & - & 11 (61.1\%) \\
\textbf{Consultas t+1000} & 18 & - & - \\
\textbf{Tasa éxito predicción} & 100\% & - & - \\
\hline
\end{tabular}
\end{table}

\textbf{Análisis del comportamiento en carga alta:}

Este escenario representa el caso más demandante, con probabilidades de bloqueo superiores al 2.8\% en todos los métodos. El método propuesto exhibe características distintivas:

\begin{itemize}
    \item \textbf{Adaptación reactiva predominante}: 61.1\% de desfragmentaciones ejecutadas en modo reactivo (intervalo 800t), reflejando la detección precisa de fragmentación crítica por el predictor
    \item \textbf{Anticipación efectiva}: Las 18 consultas al predictor ML en horizonte $t+1000$ fueron exitosas (100\%), evidenciando la robustez del modelo Gradient Boosting entrenado
    \item \textbf{Balance bloqueos-reconfiguraciones}: Con solo 18 desfragmentaciones (vs. 19 de MR1), logra 2.981\% de bloqueos, cercano al mejor resultado (2.871\% de MR2)
\end{itemize}

Distribución de bloqueos por nivel de carga:

\begin{itemize}
    \item NIVEL\_1 (2,000 Erlangs): Bloqueos mínimos (0.02\%-0.05\%)
    \item NIVEL\_2 a NIVEL\_5 (2,200-2,800 Erlangs): Incremento gradual (0.35\%-1.51\%)
    \item NIVEL\_6 a NIVEL\_8 (3,000-3,400 Erlangs): Fragmentación severa (2.05\%-5.32\%)
    \item NIVEL\_10 (4,000 Erlangs): Estado crítico, 50\% de bloqueos totales (5.77\%-6.66\%)
\end{itemize}

El comportamiento del método ML en este escenario valida su capacidad de adaptación: ante fragmentación crítica predicha, intensifica las desfragmentaciones (intervalos de 800t), mientras que en períodos de recuperación transitoria aplica estrategia preventiva (intervalos de 1,500t).

\subsection{Variante con Doble Predictor (3 Niveles)}

Adicionalmente, se evaluó una variante del método propuesto que emplea dos modelos predictores simultáneos para implementar estrategia de tres niveles:

\begin{itemize}
    \item \textbf{Predictor 1}: Umbral BFR = 0.20 (detección de fragmentación emergente)
    \item \textbf{Predictor 2}: Umbral BFR = 0.46 (detección de fragmentación crítica)
\end{itemize}

Los resultados de esta variante para los tres escenarios se presentan en la Tabla~\ref{tab:resultados_doble_predictor}.

\begin{table}[htbp]
\centering
\caption{Resultados con estrategia de doble predictor (3 niveles)}
\label{tab:resultados_doble_predictor}
\begin{tabular}{lcccc}
\hline
\textbf{Escenario} & \textbf{Bloqueos} & \textbf{Prob. (\%)} & \textbf{Desfrag.} & \textbf{Distribución} \\
\hline
\multirow{2}{*}{\textbf{Carga Baja}} & \multirow{2}{*}{717} & \multirow{2}{*}{0.720} & \multirow{2}{*}{11} & Nivel 1 (sin acción): 15.4\% \\
 & & & & Nivel 2 (largo): 84.6\% \\
 & & & & Nivel 3 (corto): 0.0\% \\
\hline
\multirow{2}{*}{\textbf{Carga Media}} & \multirow{2}{*}{1,397} & \multirow{2}{*}{1.397} & \multirow{2}{*}{15} & Nivel 1 (sin acción): 6.3\% \\
 & & & & Nivel 2 (largo): 50.0\% \\
 & & & & Nivel 3 (corto): 43.8\% \\
\hline
\multirow{2}{*}{\textbf{Carga Alta}} & \multirow{2}{*}{2,873} & \multirow{2}{*}{2.871} & \multirow{2}{*}{18} & Nivel 1 (sin acción): 0.0\% \\
 & & & & Nivel 2 (largo): 38.9\% \\
 & & & & Nivel 3 (corto): 61.1\% \\
\hline
\end{tabular}
\end{table}

Esta variante demuestra capacidad para identificar estados de red saludable (Nivel 1) donde la desfragmentación es innecesaria, logrando resultados competitivos con menos intervenciones totales en escenarios de carga baja y media.

\section{Análisis Comparativo Multiobjetivo}

\subsection{Soluciones en el Frente de Pareto}

Para cada escenario de carga, se identificaron las soluciones no dominadas considerando simultáneamente minimización de bloqueos (BL) y reconfiguraciones (RC). La Tabla~\ref{tab:frente_pareto} presenta la distribución de soluciones en el Frente de Pareto por método.

\begin{table}[htbp]
\centering
\caption{Distribución de soluciones en el Frente de Pareto}
\label{tab:frente_pareto}
\begin{tabular}{lcccc}
\hline
\textbf{Escenario} & \textbf{MP (ML)} & \textbf{MR1 (Periódico)} & \textbf{MR2 (Umbral)} & \textbf{Total} \\
\hline
Carga Baja & \textbf{3} & 2 & 2 & 7 \\
Carga Media & \textbf{3} & 1 & 2 & 6 \\
Carga Alta & \textbf{3} & 2 & 3 & 8 \\
\hline
\textbf{Total General} & \textbf{9 (42.9\%)} & 5 (23.8\%) & 7 (33.3\%) & 21 \\
\hline
\end{tabular}
\end{table}

\textbf{Interpretación de resultados:}

El método propuesto (MP) contribuye con 9 soluciones no dominadas de las 21 identificadas (42.9\%), superando significativamente a ambos métodos de referencia. Esta predominancia se mantiene consistente en los tres escenarios de carga, evidenciando:

\begin{itemize}
    \item \textbf{Robustez}: El método ML genera soluciones Pareto-óptimas independientemente del nivel de congestión
    \item \textbf{Diversidad}: Las configuraciones adaptativas (preventiva/reactiva) exploran eficientemente el espacio de soluciones
    \item \textbf{Superioridad multiobjetivo}: El balance bloqueos-reconfiguraciones supera a estrategias estáticas (periódica) y puramente reactivas (umbral)
\end{itemize}

\subsection{Cobertura de Pareto}

Para cuantificar la dominancia relativa entre métodos, se calculó la métrica de cobertura $C(A,B)$ para cada par de estrategias. La Tabla~\ref{tab:cobertura_pareto} presenta los resultados agregados.

\begin{table}[htbp]
\centering
\caption{Cobertura de Pareto entre métodos (agregado 3 escenarios)}
\label{tab:cobertura_pareto}
\begin{tabular}{llccl}
\hline
\textbf{Método A} & \textbf{Método B} & \textbf{C(A,B)} & \textbf{C(B,A)} & \textbf{Conclusión} \\
\hline
MP & MR1 (Periódico) & \textbf{0.567} & 0.222 & \makecell[l]{MP domina 56.7\% de MR1 \\ MR1 domina 22.2\% de MP} \\
\hline
MP & MR2 (Umbral) & \textbf{0.476} & 0.333 & \makecell[l]{MP domina 47.6\% de MR2 \\ MR2 domina 33.3\% de MP} \\
\hline
MR1 & MR2 & 0.333 & 0.400 & \makecell[l]{MR1 domina 33.3\% de MR2 \\ MR2 domina 40.0\% de MR1} \\
\hline
\end{tabular}
\end{table}

\textbf{Análisis de cobertura:}

Los resultados de cobertura revelan patrones distintivos:

\begin{enumerate}
    \item \textbf{MP vs. MR1 (Periódico)}:
    \begin{itemize}
        \item El método ML domina 56.7\% de las soluciones del método periódico
        \item Esta ventaja significativa (diferencia de 34.5 puntos porcentuales) evidencia la superioridad de la adaptación dinámica sobre intervalos fijos
        \item Las pocas soluciones donde MR1 domina a MP corresponden a configuraciones específicas en carga baja donde la simplicidad del enfoque periódico resulta suficiente
    \end{itemize}
    
    \item \textbf{MP vs. MR2 (Umbral)}:
    \begin{itemize}
        \item El método ML domina 47.6\% de las soluciones reactivas por umbral
        \item La diferencia menor respecto a MR1 (14.3 puntos) indica que el enfoque reactivo es más competitivo que el puramente proactivo
        \item Sin embargo, MP mantiene ventaja neta de 14.3 puntos, demostrando valor de la anticipación mediante predicción
    \end{itemize}
    
    \item \textbf{MR1 vs. MR2}:
    \begin{itemize}
        \item Cobertura equilibrada (33.3\% vs. 40.0\%), sin claro dominante
        \item Confirma que ambos enfoques tradicionales presentan limitaciones complementarias
    \end{itemize}
\end{enumerate}

\subsection{Eficiencia en Uso de Recursos}

La Figura~\ref{fig:eficiencia_recursos} (conceptual) ilustraría el trade-off entre probabilidad de bloqueo y número de desfragmentaciones para los tres métodos en cada escenario. El método propuesto ocuparía consistentemente regiones del espacio de soluciones que logran:

\begin{itemize}
    \item Menor probabilidad de bloqueo con igual o menor número de desfragmentaciones (dominancia pura)
    \item Probabilidades de bloqueo competitivas con significativamente menos desfragmentaciones (eficiencia superior)
\end{itemize}

\section{Validación del Horizonte de Predicción}

El horizonte temporal $t+1000$ fue seleccionado mediante análisis de predictibilidad previo al entrenamiento del modelo. Los resultados experimentales validan esta elección:

\begin{table}[htbp]
\centering
\caption{Validación del horizonte de predicción $t+1000$}
\label{tab:validacion_horizonte}
\begin{tabular}{lccc}
\hline
\textbf{Métrica} & \textbf{Escenario 1} & \textbf{Escenario 2} & \textbf{Escenario 3} \\
\hline
Consultas predictor & 13 & 16 & 18 \\
Consultas exitosas & 13 & 16 & 18 \\
Errores predictor & 0 & 0 & 0 \\
\textbf{Tasa éxito} & \textbf{100.0\%} & \textbf{100.0\%} & \textbf{100.0\%} \\
\hline
\end{tabular}
\end{table}

La tasa de éxito del 100\% en las 47 consultas agregadas (13+16+18) valida que:

\begin{enumerate}
    \item El modelo Gradient Boosting captura efectivamente patrones de evolución de fragmentación
    \item El horizonte $t+1000$ proporciona anticipación suficiente sin exceder el límite de predictibilidad
    \item Las características de entrada (BFR, SHF, SC, GM, ASFR3D, UD) contienen información predictiva robusta
\end{enumerate}

\section{Discusión de Resultados}

\subsection{Superioridad del Método Propuesto}

Los resultados experimentales confirman la hipótesis de que modelos de aprendizaje automático pueden predecir momentos óptimos para ejecutar desfragmentación en redes MC-EON. Las evidencias específicas incluyen:

\begin{enumerate}
    \item \textbf{Desempeño multiobjetivo superior}: El método ML genera 42.9\% de las soluciones Pareto-óptimas, superando a MR1 (23.8\%) y MR2 (33.3\%)
    
    \item \textbf{Adaptación dinámica efectiva}: La distribución preventiva/reactiva se ajusta automáticamente al nivel de carga:
    \begin{itemize}
        \item Carga baja: 100\% preventivo (máxima eficiencia)
        \item Carga media: 56.3\% preventivo / 43.8\% reactivo (transición)
        \item Carga alta: 38.9\% preventivo / 61.1\% reactivo (máxima reactividad)
    \end{itemize}
    
    \item \textbf{Reducción de bloqueos}: En escenario de alta carga, MP logra 8.5\% menos bloqueos que MR1 con número similar de desfragmentaciones
    
    \item \textbf{Eficiencia de predicción}: 100\% de éxito en 47 predicciones $t+1000$ valida robustez del modelo
\end{enumerate}

\subsection{Ventajas de la Estrategia Adaptativa}

La capacidad de ajustar dinámicamente intervalos de desfragmentación según predicciones representa la innovación fundamental del método propuesto:

\begin{itemize}
    \item \textbf{Evita intervenciones innecesarias}: En carga baja, identifica estados saludables (BFR $< 0.20$) y suprime desfragmentaciones, reduciendo overhead operacional
    
    \item \textbf{Anticipa congestión}: La predicción en horizonte $t+1000$ permite activar modo reactivo (intervalo 800t) \emph{antes} de que BFR alcance niveles críticos, minimizando bloqueos
    
    \item \textbf{Optimiza uso de recursos}: Aplica intervenciones preventivas espaciadas (1,500t) en fragmentación moderada, balanceando eficacia y costo
\end{itemize}

\subsection{Comparación con Métodos Tradicionales}

\textbf{MR1 (Periódico)}:
\begin{itemize}
    \item Simplicidad operacional, pero rigidez ante variaciones de carga
    \item En carga alta, intervalo 1,000t resulta insuficiente (3.265\% bloqueos vs. 2.981\% de MP)
    \item En carga baja, genera intervenciones excesivas (19 desfrag. vs. 13 de MP)
\end{itemize}

\textbf{MR2 (Umbral)}:
\begin{itemize}
    \item Estrategia puramente reactiva: actúa cuando fragmentación es observable
    \item Carece de anticipación: no previene estados críticos, solo responde a ellos
    \item Competitivo en carga media-baja, pero en alta carga la ausencia de anticipación limita efectividad
\end{itemize}
