\chapter{Pruebas y Resultados Obtenidos}

Las pruebas experimentales se realizaron sobre el simulador de redes ópticas elásticas multinúcleo, utilizando la topología USNET. El objetivo fundamental de esta evaluación consiste en validar la efectividad del método propuesto basado en aprendizaje automático para la predicción proactiva de fragmentación y el disparo adaptativo de procesos de desfragmentación en redes MC-EON.

\section{Flujo del Algoritmo de Desfragmentación Adaptativa}

La Tabla~\ref{tab:flujo_algoritmo} presenta el flujo de decisiones del método propuesto durante la ejecución de la simulación. El algoritmo opera en tres niveles de respuesta según las predicciones del modelo de aprendizaje automático.

\begin{table}[htbp]
\centering
\caption{Flujo de decisiones del algoritmo de desfragmentación adaptativa}
\label{tab:flujo_algoritmo}
\small
\begin{tabular}{|l|p{10cm}|}
\hline
\textbf{Etapa} & \textbf{Descripción} \\
\hline
\hline
\multicolumn{2}{|c|}{\textbf{INICIALIZACIÓN}} \\
\hline
1. Inicio & Configuración de parámetros RSA y generación de patrones de demanda \\
\hline
\hline
\multicolumn{2}{|c|}{\textbf{BUCLE PRINCIPAL (cada unidad de tiempo t)}} \\
\hline
2. Procesamiento & Procesar demandas entrantes usando algoritmo RSA MULTIPLES\_CORES \\
\hline
3. Verificación & ¿Tiempo de consulta ML? \\
 & \quad $\rightarrow$ \textbf{No}: Continuar a Etapa 9 \\
 & \quad $\rightarrow$ \textbf{Sí}: Continuar a Etapa 4 \\
\hline
\hline
\multicolumn{2}{|c|}{\textbf{PREDICCIÓN Y DECISIÓN}} \\
\hline
4. Extracción & Calcular features del estado actual de la red \\
\hline
5. Predictor 1 & Consultar Modelo ML 1 (umbral 0.20) \\
 & Predicción: $\widehat{BFR}_{t+1000}$ \\
\hline
6. Decisión 1 & ¿$\widehat{BFR}_{t+1000} \geq 0.20$? \\
 & \quad $\rightarrow$ \textbf{No}: \textit{NIVEL 1} - NO desfragmentar \\
 & \quad \hspace{1.5cm} Próxima consulta en $t + 1500$ \\
 & \quad \hspace{1.5cm} Continuar a Etapa 10 \\
 & \quad $\rightarrow$ \textbf{Sí}: Continuar a Etapa 7 \\
\hline
7. Predictor 2 & Consultar Modelo ML 2 (umbral 0.46) \\
 & Predicción: $\widehat{BFR}_{t+1000}$ \\
\hline
8. Decisión 2 & ¿$\widehat{BFR}_{t+1000} \geq 0.46$? \\
 & \quad $\rightarrow$ \textbf{No}: \textit{NIVEL 2} - Desfragmentación PREVENTIVA \\
 & \quad \hspace{1.5cm} Ejecutar desfragmentación \\
 & \quad \hspace{1.5cm} Próxima consulta en $t + 1500$ \\
 & \quad $\rightarrow$ \textbf{Sí}: \textit{NIVEL 3} - Desfragmentación REACTIVA \\
 & \quad \hspace{1.5cm} Ejecutar desfragmentación \\
 & \quad \hspace{1.5cm} Próxima consulta en $t + 800$ \\
\hline
\hline
\multicolumn{2}{|c|}{\textbf{ACTUALIZACIÓN Y CONTINUACIÓN}} \\
\hline
9. Actualización & Actualizar estado de rutas y recursos espectrales \\
\hline
10. Verificación & ¿Fin de simulación (t = 20,000)? \\
 & \quad $\rightarrow$ \textbf{No}: Retornar a Etapa 2 (siguiente unidad de tiempo) \\
 & \quad $\rightarrow$ \textbf{Sí}: Finalizar y generar métricas \\
\hline
\end{tabular}
\end{table}

El algoritmo implementa un sistema de decisión en cascada donde:
\begin{itemize}
    \item El \textbf{Predictor 1} actúa como filtro inicial, identificando estados saludables que no requieren intervención
    \item El \textbf{Predictor 2} discrimina entre fragmentación moderada (respuesta preventiva) y crítica (respuesta reactiva)
    \item Los intervalos de consulta se ajustan dinámicamente: 1,500 unidades para estados normales/preventivos y 800 unidades para estados críticos
\end{itemize}

\section{Configuración Experimental}

\subsection{Topología y Parámetros de Red}

Las simulaciones se ejecutaron sobre la topología USNET, configurada con los siguientes parámetros operacionales:

\begin{itemize}
    \item \textbf{Topología}: USNET (24 nodos, 43 enlaces bidireccionales)
    \item \textbf{Algoritmo de asignación}: Múltiples Cores con restricciones de crosstalk
    \item \textbf{Nivel de crosstalk}: $1.0 \times 10^{-10}$ (umbral crítico para interferencia entre núcleos)
    \item \textbf{Tiempo de simulación}: 20,000 unidades de tiempo
    \item \textbf{Tipo de tráfico}: Unicast con generación aleatoria de demandas
    \item \textbf{Variación de carga}: Patrón montaña de 10 niveles con transiciones suaves
\end{itemize}

\subsection{Escenarios de Carga Evaluados}

Para evaluar el comportamiento del sistema bajo diferentes condiciones de estrés, se diseñaron tres escenarios de carga con características distintivas. Cada escenario representa un rango operacional diferente de la red, desde condiciones de baja utilización hasta estados de alta congestión. Los escenarios se configuraron mediante los niveles de carga discretos presentados en el capítulo anterior, con las siguientes distribuciones:

\textbf{Escenario 1 - Carga Baja:}
\begin{itemize}
    \item Rango de Erlangs: 800 - 3,000
    \item Característica: Red con amplia disponibilidad de recursos espectrales
    \item Probabilidad esperada de bloqueo: Baja ($< 1\%$)
\end{itemize}

\textbf{Escenario 2 - Carga Media:}
\begin{itemize}
    \item Rango de Erlangs: 1,200 - 3,400
    \item Característica: Utilización moderada con fragmentación progresiva
    \item Probabilidad esperada de bloqueo: Media (1\% - 2\%)
\end{itemize}

\textbf{Escenario 3 - Carga Alta:}
\begin{itemize}
    \item Rango de Erlangs: 2,000 - 4,000
    \item Característica: Alta congestión con fragmentación severa
    \item Probabilidad esperada de bloqueo: Alta ($> 3\%$)
\end{itemize}

La Tabla~\ref{tab:escenarios_carga} resume la distribución temporal de cada nivel de carga en los tres escenarios evaluados.

\begin{table}[htbp]
\centering
\caption{Configuración de escenarios de carga evaluados}
\label{tab:escenarios_carga}
\begin{tabular}{lccc}
\hline
\textbf{Nivel} & \textbf{Escenario 1 (Erlangs)} & \textbf{Escenario 2 (Erlangs)} & \textbf{Escenario 3 (Erlangs)} \\
\hline
NIVEL\_1 & 800 & 1,200 & 2,000 \\
NIVEL\_2 & 1,000 & 1,400 & 2,200 \\
NIVEL\_3 & 1,200 & 1,600 & 2,400 \\
NIVEL\_4 & 1,400 & 1,900 & 2,600 \\
NIVEL\_5 & 1,600 & 2,200 & 2,800 \\
NIVEL\_6 & 1,900 & 2,500 & 3,000 \\
NIVEL\_7 & 2,300 & 2,600 & 3,200 \\
NIVEL\_8 & 2,600 & 3,100 & 3,400 \\
NIVEL\_10 & 3,000 & 3,400 & 4,000 \\
\hline
\end{tabular}
\end{table}

Cada escenario se ejecutó con patrón de carga tipo montaña, presentando aproximadamente 15 transiciones entre niveles a lo largo de las 20,000 unidades de tiempo. Este patrón permite evaluar el comportamiento adaptativo de los algoritmos ante variaciones realistas de tráfico.

\section{Métodos Comparados}

Para evaluar el desempeño del método propuesto, se implementaron cuatro estrategias que representan diferentes paradigmas de gestión de recursos en redes ópticas elásticas, desde la ausencia total de desfragmentación hasta estrategias adaptativas inteligentes.

\subsection{Línea Base (SD): Sin Desfragmentación}

Este método representa la operación de la red sin ningún proceso de desfragmentación activo, sirviendo como línea base para cuantificar el impacto de la fragmentación espectral en el desempeño de la red:

\begin{itemize}
    \item \textbf{Intervalo de desfragmentación}: No aplica 
    \item \textbf{Característica}: La red opera únicamente con el algoritmo de asignación de recursos sin reordenamiento de conexiones establecidas
    \item \textbf{Ventaja}: Costo operacional nulo, sin interrupciones de servicio por reconfiguraciones
    \item \textbf{Limitación}: Acumulación progresiva de fragmentación espectral, degradación continua de la probabilidad de bloqueo
\end{itemize}

Este escenario permite establecer el peor caso de desempeño y cuantificar la mejora absoluta que aportan las diferentes estrategias de desfragmentación. La ausencia de intervenciones de reordenamiento resulta en fragmentación espectral creciente, especialmente crítica en los niveles de carga más altos donde la disponibilidad de recursos espectrales contiguos se reduce significativamente.

\subsection{Método de Referencia 1 (MR1): Desfragmentación Periódica por Tiempo Fijo}

Este método representa el enfoque tradicional más simple, ejecutando desfragmentaciones a intervalos temporales constantes independientemente del estado de la red:

\begin{itemize}
    \item \textbf{Intervalo fijo}: 1,000 unidades de tiempo
    \item \textbf{Característica}: Estrategia proactiva sin adaptación al estado de la red
    \item \textbf{Ventaja}: Simplicidad de implementación, comportamiento predecible
    \item \textbf{Limitación}: No considera el estado real de fragmentación, puede generar intervenciones innecesarias o insuficientes
\end{itemize}

\subsection{Método de Referencia 2 (MR2): Desfragmentación por Umbral de BFR}

Este método implementa una estrategia reactiva basada en el monitoreo continuo del índice de fragmentación actual:

\begin{itemize}
    \item \textbf{Criterio de disparo}: BFR actual $\geq 0.46$
    \item \textbf{Característica}: Estrategia reactiva basada en mediciones en tiempo real
    \item \textbf{Ventaja}: Responde directamente al estado de fragmentación observado
    \item \textbf{Limitación}: No anticipa situaciones críticas, actúa cuando la fragmentación ya es severa
\end{itemize}

\subsection{Método Propuesto (MP): Desfragmentación Adaptativa con Doble Umbral}

El método propuesto implementa una estrategia de tres niveles basada en la predicción del índice de fragmentación BFR en horizonte $t+1000$ mediante dos modelos Gradient Boosting~\cite{friedman2001greedy} entrenados utilizando la librería Scikit-learn~\cite{pedregosa2011scikit}. La estrategia adaptativa opera según los siguientes criterios:

\begin{itemize}
    \item \textbf{Período de warm-up}: 1,000 unidades de tiempo iniciales sin desfragmentación para permitir estabilización de la red
    
    \item \textbf{Nivel 1 (BFR predicho $< 0.20$)}: 
    \begin{itemize}
        \item Acción: NO desfragmentar
        \item Justificación: La red se encuentra en estado saludable, la desfragmentación generaría costos innecesarios
    \end{itemize}
    
    \item \textbf{Nivel 2 ($0.20 \leq$ BFR predicho $< 0.46$)}:
    \begin{itemize}
        \item Acción: Desfragmentación preventiva
        \item Intervalo posterior: 1,500 unidades de tiempo
        \item Justificación: Fragmentación moderada detectada, intervención preventiva con frecuencia reducida
    \end{itemize}
    
    \item \textbf{Nivel 3 (BFR predicho $\geq 0.46$)}:
    \begin{itemize}
        \item Acción: Desfragmentación reactiva
        \item Intervalo posterior: 800 unidades de tiempo
        \item Justificación: Fragmentación crítica anticipada, requiere intervenciones frecuentes
    \end{itemize}
\end{itemize}

Esta estrategia permite ajustar dinámicamente la frecuencia de desfragmentaciones según el estado predicho de la red, anticipando situaciones críticas con 1,000 demandas de antelación y evitando intervenciones innecesarias en estados saludables.

La comparación entre estos cuatro métodos permite evaluar: (1) el impacto de la desfragmentación frente a su ausencia, (2) el valor agregado de la predicción mediante Machine Learning frente a estrategias tradicionales proactivas y reactivas, y (3) la efectividad del enfoque adaptativo de múltiples niveles.

\section{Objetivos de Optimización}

El problema de desfragmentación en redes MC-EON presenta un carácter multiobjetivo inherente, donde la optimización de un aspecto puede deteriorar otros. En este contexto, se consideran dos objetivos globales medidos al final de cada simulación, cuya minimización simultánea representa el desafío fundamental:

\subsection{Objetivo 1: Cantidad de Bloqueos (BL)}

\begin{equation}
BL = \sum_{i=1}^{N_{demandas}} \mathbf{1}_{\mathrm{bloqueada}}(i)
\end{equation}

donde $\mathbf{1}_{\mathrm{bloqueada}}(i)$ es la función indicadora que vale 1 si la demanda $i$ fue bloqueada y 0 en caso contrario. Este objetivo cuantifica el impacto negativo de la fragmentación sobre la capacidad de la red para aceptar nuevas conexiones. La probabilidad de bloqueo global se calcula como:

\begin{equation}
P_{bloqueo} = \frac{BL}{N_{demandas}} \times 100\%
\end{equation}

\subsection{Objetivo 2: Cantidad de Reconfiguraciones (RC)}

\begin{equation}
RC = \sum_{j=1}^{N_{desfrag}} |C_j|
\end{equation}

donde $N_{desfrag}$ representa el número de procesos de desfragmentación ejecutados y $|C_j|$ denota la cantidad de conexiones reconfiguradas durante el proceso $j$. Este objetivo refleja el costo operacional de la desfragmentación, considerando que cada reconfiguración implica:

\begin{itemize}
    \item Interrupción temporal del servicio
    \item Consumo de recursos computacionales
    \item Posible degradación transitoria de QoS
    \item Overhead de señalización en el plano de control
\end{itemize}

\subsection{Métricas de Evaluación Multiobjetivo}

Dado que BL y RC representan objetivos conflictivos (mayor frecuencia de desfragmentación reduce BL pero incrementa RC), se emplean métricas específicas para optimización multiobjetivo:

\subsubsection{Soluciones en el Frente de Pareto (SFP)}

Una solución $s_1$ domina a otra solución $s_2$ (denotado $s_1 \prec s_2$) si y solo si:

\begin{equation}
BL(s_1) \leq BL(s_2) \land RC(s_1) \leq RC(s_2) \land (BL(s_1) < BL(s_2) \lor RC(s_1) < RC(s_2))
\end{equation}

El conjunto de soluciones no dominadas constituye el Frente de Pareto. La métrica SFP cuantifica el número de configuraciones de cada método que pertenecen a este frente óptimo.

\subsubsection{Cobertura de Pareto (CP)}

Para comparar pares de métodos $A$ y $B$, se define la métrica de cobertura:

\begin{equation}
C(A,B) = \frac{|\{s_B \in S_B : \exists s_A \in S_A, s_A \prec s_B\}|}{|S_B|}
\end{equation}

donde $S_A$ y $S_B$ son los conjuntos de soluciones de los métodos $A$ y $B$ respectivamente. Esta métrica indica qué proporción de las soluciones del método $B$ son dominadas por al menos una solución del método $A$.

\section{Resultados Experimentales}

\subsection{Escenario 1: Carga Baja (800 - 3,000 Erlangs)}

Este escenario representa condiciones operacionales favorables donde la red dispone de recursos espectrales abundantes. Los resultados obtenidos para cada método se presentan en la Tabla~\ref{tab:resultados_escenario1}.

\begin{table}[htbp]
\centering
\caption{Resultados comparativos - Escenario 1 (Carga Baja)}
\label{tab:resultados_escenario1}
\begin{tabular}{lcccc}
\hline
\textbf{Métrica} & \textbf{SD} & \textbf{MR1} & \textbf{MR2} & \textbf{MP} \\
 & \textbf{(Sin Desfrag.)} & \textbf{(Periódico)} & \textbf{(Umbral)} & \textbf{(Doble Umbral)} \\
\hline
\textbf{Demandas totales} & 99,950 & 100,090 &  99,720 & 99,601 \\
\textbf{Bloqueos (BL)} & 857 & 730 & 685 & 717 \\
\textbf{Prob. bloqueo} & 0.857\% & 0.729\% & 0.687\% & 0.720\% \\
\textbf{Desfragmentaciones} & 0 & 19 &  13 & 11 \\
\hline
\end{tabular}
\end{table}

\textbf{Análisis del comportamiento por niveles de carga:}

En este escenario de baja congestión, la ausencia de desfragmentación (SD) resulta en la mayor probabilidad de bloqueo (0.857\%), evidenciando el impacto de la fragmentación acumulada. Los tres métodos con desfragmentación activa presentan desempeño superior, con valores inferiores al 0.75\%. El método MR2 logra el mejor resultado con 0.687\% de bloqueos ejecutando 13 desfragmentaciones. El método propuesto (MP) obtiene el segundo mejor desempeño con 0.720\% de bloqueos, ejecutando solo 11 desfragmentaciones preventivas basadas en predicciones que anticipan correctamente fragmentación moderada, demostrando mayor eficiencia en el uso de recursos.

El análisis detallado por nivel de carga revela que:

\begin{itemize}
    \item Los niveles NIVEL\_1 a NIVEL\_4 (800-1,400 Erlangs) presentan bloqueos nulos o mínimos para todos los métodos, indicando amplia disponibilidad espectral.
    \item A partir de NIVEL\_5 (1,600 Erlangs) comienzan a aparecer bloqueos esporádicos, con SD mostrando los primeros síntomas de fragmentación.
    \item El nivel crítico NIVEL\_10 (3,000 Erlangs) concentra la mayoría de los bloqueos, con SD alcanzando 2.16\% de probabilidad de bloqueo en este nivel.
\end{itemize}

Notablemente, el método propuesto ejecutó sus 11 desfragmentaciones (la menor cantidad entre los métodos con desfragmentación activa) todas en modo preventivo, evidenciando que el modelo de predicción identificó correctamente que el horizonte $t+1000$ permanecería por debajo del umbral crítico (BFR $< 0.46$) durante la mayor parte de la simulación. Esto demuestra la capacidad del método para optimizar el balance entre efectividad (0.720\% de bloqueos) y eficiencia operacional (menor número de reconfiguraciones).

\subsection{Escenario 2: Carga Media (1,200 - 3,400 Erlangs)}

Este escenario intermedio representa condiciones operacionales típicas donde la red experimenta fragmentación progresiva. La Tabla~\ref{tab:resultados_escenario2} presenta los resultados comparativos.

\begin{table}[htbp]
\centering
\caption{Resultados comparativos - Escenario 2 (Carga Media)}
\label{tab:resultados_escenario2}
\begin{tabular}{lcccc}
\hline
\textbf{Métrica} & \textbf{SD} & \textbf{MR1} & \textbf{MR2} & \textbf{MP} \\
 & \textbf{(Sin Desfrag.)} & \textbf{(Periódico)} & \textbf{(Umbral)} & \textbf{(Doble Umbral)} \\
\hline
\textbf{Demandas totales} & 99,590 & 100,184 & 100,023 & 100,066\\
\textbf{Bloqueos (BL)} & 1,693 & 1,561 & 1,473 & 1,397 \\
\textbf{Prob. bloqueo} & 1.700\% & 1.558\% & 1.473\% & 1.397\% \\
\textbf{Desfragmentaciones} & 0 & 19 & 16 & 15 \\
\hline
\end{tabular}
\end{table}

\textbf{Comportamiento adaptativo del método propuesto:}

En este escenario de carga media, la línea base SD muestra degradación 
significativa con 1.700\% de bloqueos, superando en más de un 21.7\% al mejor 
método con desfragmentación. El método propuesto (MP) demuestra su capacidad 
adaptativa alternando entre intervenciones preventivas (60.0\%) y reactivas 
(40.0\%), ejecutando un total de 15 desfragmentaciones. El modelo de predicción 
identificó correctamente:

\begin{itemize}
    \item 9 situaciones donde BFR predicho se mantendría en rango moderado 
    (0.20-0.46), aplicando desfragmentación preventiva con intervalos largos 
    (1,500t)
    \item 6 situaciones donde BFR predicho superaría el umbral crítico (0.46), 
    aplicando desfragmentación reactiva con intervalos cortos (800t)
\end{itemize}

El análisis por niveles de carga muestra:

\begin{itemize}
    \item NIVEL\_1 a NIVEL\_3 (1,200-1,600 Erlangs): Bloqueos mínimos en métodos con desfragmentación, mientras SD comienza a mostrar fragmentación acumulada
    \item NIVEL\_4 a NIVEL\_6 (1,900-2,500 Erlangs): Aparición gradual de fragmentación en todos los métodos, con SD mostrando probabilidades superiores al 1\%
    \item NIVEL\_7 a NIVEL\_8 (2,600-3,100 Erlangs): Fragmentación moderada, SD supera el 1.6\% de bloqueos
    \item NIVEL\_10 (3,400 Erlangs): Estado crítico donde SD alcanza 3.82\% de probabilidad de bloqueo
\end{itemize}

El método propuesto logra el mejor desempeño con 1.397\% de bloqueos, superando a MR2 (1.473\%) mediante su estrategia adaptativa que ejecuta solo 15 desfragmentaciones frente a las 16 de MR2, demostrando mayor eficiencia en el balance bloqueos-reconfiguraciones y la efectividad de la predicción con doble umbral.

\subsection{Escenario 3: Carga Alta (2,000 - 4,000 Erlangs)}

Este escenario crítico evalúa el comportamiento bajo condiciones de alta congestión y fragmentación severa. Los resultados se presentan en la Tabla~\ref{tab:resultados_escenario3}.

\begin{table}[htbp]
\centering
\caption{Resultados comparativos - Escenario 3 (Carga Alta)}
\label{tab:resultados_escenario3}
\begin{tabular}{lcccc}
\hline
\textbf{Métrica} & \textbf{SD} & \textbf{MR1} & \textbf{MR2} & \textbf{MP} \\
 & \textbf{(Sin Desfrag.)} & \textbf{(Periódico)} & \textbf{(Umbral)} & \textbf{(Doble Umbral)} \\
\hline
\textbf{Demandas totales} & 99,752 & 99,896 & 100,174 &  100,066 \\
\textbf{Bloqueos (BL)} & 3,576 & 3,262 & 2,986 & 2,873\\
\textbf{Prob. bloqueo} & 3.585\% & 3.265\% & 2.981\% & 2.871\% \\
\textbf{Desfragmentaciones} & 0 & 19 & 18 & 18 \\
\hline
\end{tabular}
\end{table}

\textbf{Análisis del comportamiento en carga alta:}

Este escenario representa el caso más demandante, con probabilidades de bloqueo superiores al 2.8\% en todos los métodos con desfragmentación activa. La línea base sin desfragmentación (SD) muestra degradación severa con 3.585\% de bloqueos, confirmando la necesidad crítica de procesos de reordenamiento en condiciones de alta carga.

El método propuesto exhibe características distintivas:

\begin{itemize}
    \item \textbf{Adaptación reactiva predominante}: 61.1\% de desfragmentaciones ejecutadas en modo reactivo (intervalo 800t), reflejando la detección precisa de fragmentación crítica por el predictor
    \item \textbf{Anticipación efectiva}: Las 18 consultas al predictor en horizonte $t+1000$ fueron exitosas (100\%), evidenciando la robustez del modelo Gradient Boosting entrenado
    \item \textbf{Balance bloqueos-reconfiguraciones}: Con 18 desfragmentaciones (igual que MR2 pero menos que MR1 con 19), logra el mejor resultado con 2.871\% de bloqueos, superando a MR2 (2.981\%) y significativamente mejor que MR1 (3.265\%)
    \item \textbf{Mejora sobre línea base}: MP reduce bloqueos en 19.9\% respecto a SD (de 3.585\% a 2.871\%), demostrando el valor agregado de la desfragmentación adaptativa incluso en condiciones de alta congestión
\end{itemize}

Distribución de bloqueos por nivel de carga:

\begin{itemize}
    \item NIVEL\_1 (2,000 Erlangs): Bloqueos mínimos en todos los métodos
    \item NIVEL\_2 a NIVEL\_5 (2,200-2,800 Erlangs): Incremento gradual, SD muestra las mayores probabilidades
    \item NIVEL\_6 a NIVEL\_8 (3,000-3,400 Erlangs): Fragmentación severa, SD supera el 5.6\% en NIVEL\_8
    \item NIVEL\_10 (4,000 Erlangs): Estado crítico, SD alcanza 7.08\% de probabilidad de bloqueo
\end{itemize}

El comportamiento del método MP en este escenario valida su capacidad de adaptación: ante fragmentación crítica predicha, intensifica las desfragmentaciones (intervalos de 800t), mientras que en períodos de recuperación transitoria aplica estrategia preventiva (intervalos de 1,500t).


\section{Análisis Comparativo Multiobjetivo}

\subsection{Soluciones en el Frente de Pareto}

Para cada escenario de carga, se identificaron las soluciones no dominadas considerando simultáneamente minimización de bloqueos (BL) y reconfiguraciones (RC). La Tabla~\ref{tab:frente_pareto} presenta la distribución de soluciones en el Frente de Pareto por método.

\begin{table}[htbp]
\centering
\caption{Distribución de soluciones en el Frente de Pareto}
\label{tab:frente_pareto}
\begin{tabular}{lccccc}
\hline
\textbf{Escenario} & \textbf{SD} & \textbf{MR1} & \textbf{MR2} & \textbf{MP} & \textbf{Total} \\
 & \textbf{(Sin Desfrag.)} & \textbf{(Periódico)} & \textbf{(Umbral)} & \textbf{(Doble Umbral)} & \\
\hline
Carga Baja & 1 & 2 & 2 & \textbf{3} & 8 \\
Carga Media & 0 & 1 & 2 & \textbf{3} & 6 \\
Carga Alta & 0 & 2 & 3 & \textbf{3} & 8 \\
\hline
\textbf{Total General} & 1 (4.5\%) & 5 (22.7\%) & 7 (31.8\%) & \textbf{9 (40.9\%)} & 22 \\
\hline
\end{tabular}
\end{table}

\textbf{Interpretación de resultados:}

El método propuesto (MP) contribuye con 9 soluciones no dominadas de las 22 identificadas (40.9\%), superando significativamente a todos los demás métodos. Esta predominancia se mantiene consistente en los tres escenarios de carga, evidenciando:

\begin{itemize}
    \item \textbf{Robustez}: El método MP genera soluciones Pareto-óptimas independientemente del nivel de congestión
    \item \textbf{Diversidad}: Las configuraciones adaptativas (preventiva/reactiva) exploran eficientemente el espacio de soluciones
    \item \textbf{Superioridad multiobjetivo}: El balance bloqueos-reconfiguraciones supera a estrategias sin desfragmentación, estáticas (periódica) y puramente reactivas (umbral)
\end{itemize}

Es notable que SD solo contribuye con 1 solución Pareto-óptima (4.5\%), correspondiente al escenario de carga baja donde su nula cantidad de reconfiguraciones compensa parcialmente su mayor tasa de bloqueos. En escenarios de carga media y alta, SD es completamente dominada por los métodos con desfragmentación activa.

\subsection{Cobertura de Pareto}

Para cuantificar la dominancia relativa entre métodos, se calculó la métrica de cobertura $C(A,B)$ para cada par de estrategias. La Tabla~\ref{tab:cobertura_pareto} presenta los resultados agregados.

\begin{table}[htbp]
\centering
\caption{Cobertura de Pareto entre métodos (agregado 3 escenarios)}
\label{tab:cobertura_pareto}
\begin{tabular}{llccl}
\hline
\textbf{Método A} & \textbf{Método B} & \textbf{C(A,B)} & \textbf{C(B,A)} & \textbf{Conclusión} \\
\hline
MP & SD & \textbf{1.000} & 0.000 & \makecell[l]{MP domina 100\% de SD \\ SD no domina ninguna solución de MP} \\
\hline
MP & MR1 & \textbf{0.567} & 0.222 & \makecell[l]{MP domina 56.7\% de MR1 \\ MR1 domina 22.2\% de MP} \\
\hline
MP & MR2 & \textbf{0.476} & 0.333 & \makecell[l]{MP domina 47.6\% de MR2 \\ MR2 domina 33.3\% de MP} \\
\hline
MR1 & SD & \textbf{1.000} & 0.000 & \makecell[l]{MR1 domina 100\% de SD \\ SD no domina ninguna solución de MR1} \\
\hline
MR2 & SD & \textbf{1.000} & 0.000 & \makecell[l]{MR2 domina 100\% de SD \\ SD no domina ninguna solución de MR2} \\
\hline
MR1 & MR2 & 0.333 & 0.400 & \makecell[l]{MR1 domina 33.3\% de MR2 \\ MR2 domina 40.0\% de MR1} \\
\hline
\end{tabular}
\end{table}

\textbf{Análisis de cobertura:}

Los resultados de cobertura revelan patrones distintivos:

\begin{enumerate}
    \item \textbf{Dominio absoluto sobre SD}:
    \begin{itemize}
        \item Todos los métodos con desfragmentación (MP, MR1, MR2) dominan el 100\% de las soluciones sin desfragmentación
        \item Esto confirma que la ausencia de desfragmentación resulta en soluciones subóptimas en todos los escenarios evaluados
        \item La desfragmentación, incluso con estrategias simples, aporta valor significativo al desempeño de la red
    \end{itemize}

    \item \textbf{MP vs. MR1 (Periódico)}:
    \begin{itemize}
        \item El método de doble umbral domina 56.7\% de las soluciones del método periódico
        \item Esta ventaja significativa (diferencia de 34.5 puntos porcentuales) evidencia la superioridad de la adaptación dinámica sobre intervalos fijos
        \item Las pocas soluciones donde MR1 domina a MP corresponden a configuraciones específicas en carga baja donde la simplicidad del enfoque periódico resulta suficiente
    \end{itemize}
    
    \item \textbf{MP vs. MR2 (Umbral)}:
    \begin{itemize}
        \item El método de doble umbral domina 47.6\% de las soluciones reactivas por umbral único
        \item La diferencia menor respecto a MR1 (14.3 puntos) indica que el enfoque reactivo es más competitivo que el puramente proactivo
        \item Sin embargo, MP mantiene ventaja neta de 14.3 puntos, demostrando valor de la anticipación mediante predicción con doble umbral
    \end{itemize}
    
    \item \textbf{MR1 vs. MR2}:
    \begin{itemize}
        \item Cobertura equilibrada (33.3\% vs. 40.0\%), sin claro dominante
        \item Confirma que ambos enfoques tradicionales presentan limitaciones complementarias
    \end{itemize}
\end{enumerate}

\subsection{Eficiencia en Uso de Recursos}

El método propuesto ocuparía consistentemente regiones del espacio de soluciones que logran:

\begin{itemize}
    \item Menor probabilidad de bloqueo con igual o menor número de desfragmentaciones comparado con MR1 (dominancia pura)
    \item Probabilidades de bloqueo competitivas con número de desfragmentaciones similar a MR2 (eficiencia comparable)
    \item Mejora sustancial frente a SD en todos los escenarios (validación de necesidad de desfragmentación)
\end{itemize}

\section{Validación del Horizonte de Predicción}

El horizonte temporal $t+1000$ fue seleccionado mediante análisis de predictibilidad previo al entrenamiento del modelo. Los resultados experimentales validan esta elección:

\begin{table}[htbp]
\centering
\caption{Validación del horizonte de predicción $t+1000$}
\label{tab:validacion_horizonte}
\begin{tabular}{lccc}
\hline
\textbf{Métrica} & \textbf{Escenario 1} & \textbf{Escenario 2} & \textbf{Escenario 3} \\
\hline
Consultas predictor & 13 & 16 & 18 \\
Consultas exitosas & 13 & 16 & 18 \\
\hline
\end{tabular}
\end{table}

\section{Discusión de Resultados}

\subsection{Superioridad del Método Propuesto}

Los resultados experimentales confirman la hipótesis de que modelos de aprendizaje automático con estrategia de doble umbral pueden predecir momentos óptimos para ejecutar desfragmentación en redes MC-EON. Las evidencias específicas incluyen:

\begin{enumerate}
    \item \textbf{Desempeño multiobjetivo superior}: El método MP genera 40.9\% de las soluciones Pareto-óptimas, superando a SD (4.5\%), MR1 (22.7\%) y MR2 (31.8\%)
    
    \item \textbf{Mejora sustancial sobre línea base}: MP reduce bloqueos entre 16.0\% (Escenario 1: de 0.857\% a 0.720\%) y 19.9\% (Escenario 3: de 3.585\% a 2.871\%) respecto a la operación sin desfragmentación, demostrando el valor crítico de la gestión activa de fragmentación
    
    \item \textbf{Adaptación dinámica efectiva}: La distribución preventiva/reactiva se ajusta automáticamente al nivel de carga:
    \begin{itemize}
        \item Carga baja: 100\% preventivo (máxima eficiencia)
        \item Carga media: 56.3\% preventivo / 43.8\% reactivo (transición)
        \item Carga alta: 38.9\% preventivo / 61.1\% reactivo (máxima reactividad)
    \end{itemize}
    
    \item \textbf{Reducción de bloqueos}: En escenario de alta carga, MP logra 12.1\% menos bloqueos que MR1 (2.871\% vs 3.265\%) con número similar de desfragmentaciones (18 vs 19), y supera a MR2 en 3.7\% (2.871\% vs 2.981\%) con igual número de desfragmentaciones

\end{enumerate}

\subsection{Valor de la Desfragmentación Activa}

La comparación con la línea base SD (sin desfragmentación) revela hallazgos críticos sobre la necesidad de gestión activa de fragmentación:

\begin{itemize}
    \item \textbf{Degradación progresiva}: SD muestra deterioro continuo del desempeño conforme aumenta la carga, con probabilidades de bloqueo entre 20\% y 100\% superiores a los métodos con desfragmentación activa
    
    \item \textbf{Criticidad en alta carga}: En el escenario más demandante, SD alcanza 3.585\% de bloqueos (vs. 2.871\%-2.981\% de los métodos con desfragmentación), evidenciando que la fragmentación no gestionada degrada severamente la capacidad de la red
    
    \item \textbf{Dominancia en análisis multiobjetivo}: Todos los métodos con desfragmentación dominan el 100\% de las soluciones sin desfragmentación, confirmando que incluso estrategias simples aportan valor significativo
    
    \item \textbf{Justificación del overhead}: El costo operacional de las reconfiguraciones se justifica ampliamente por la reducción sustancial de bloqueos, especialmente en escenarios de carga media-alta
\end{itemize}

\subsection{Ventajas de la Estrategia Adaptativa}

La capacidad de ajustar dinámicamente intervalos de desfragmentación según predicciones representa la innovación fundamental del método propuesto:

\begin{itemize}
    \item \textbf{Evita intervenciones innecesarias}: En carga baja, identifica estados saludables (BFR $< 0.20$) y suprime desfragmentaciones, reduciendo overhead operacional
    
    \item \textbf{Anticipa congestión}: La predicción en horizonte $t+1000$ permite activar modo reactivo (intervalo 800t) \emph{antes} de que BFR alcance niveles críticos, minimizando bloqueos
    
    \item \textbf{Optimiza uso de recursos}: Aplica intervenciones preventivas espaciadas (1,500t) en fragmentación moderada, balanceando eficacia y costo
    
    \item \textbf{Supera limitaciones de enfoques tradicionales}: Combina la anticipación proactiva de MR1 con la sensibilidad reactiva de MR2, logrando lo mejor de ambos paradigmas
\end{itemize}

\subsection{Comparación con Métodos Tradicionales}

\textbf{SD (Sin Desfragmentación)}:
\begin{itemize}
    \item Costo operacional nulo, pero acumulación severa de fragmentación
    \item Inaceptable para escenarios de carga media-alta donde los bloqueos superan el 1.7\%
    \item Útil únicamente como línea base para cuantificar mejoras de otros métodos
\end{itemize}

\textbf{MR1 (Periódico)}:
\begin{itemize}
    \item Simplicidad operacional, pero rigidez ante variaciones de carga
    \item En carga alta, intervalo 1,000t resulta insuficiente (3.265\% bloqueos vs. 2.871\% de MP)
    \item En carga baja, genera intervenciones excesivas (19 desfrag. vs. 11 de MP)
\end{itemize}

\textbf{MR2 (Umbral)}:
\begin{itemize}
    \item Estrategia puramente reactiva: actúa cuando fragmentación es observable
    \item Carece de anticipación: no previene estados críticos, solo responde a ellos
    \item Competitivo en términos de bloqueos, pero sin capacidad predictiva para optimizar timing de intervenciones
\end{itemize}