\chapter{Introducción}
\section{Justificación}

Debido al incremento de la popularidad de internet y del uso de servicios en la nube, tales como \textit{Content Delivery Network} (CDN) y \textit{Video on Demand} (VoD), las demandas de tasas de bits en las redes han crecido de manera exponencial \cite{aibin2016defragmentation}, lo que obliga a estudiar nuevas y mejores tecnologías relacionadas a la transmisión de datos.

Las Redes de Multiplexación por División de Longitud de Onda o \textit{Wavelength Division Multiplexing} (WDM), utilizan una grilla fija, de 50 o 100 GHz, dan una gran ventaja logrando velocidades muy superiores frente a las viejas tecnologías, pero a pesar de esta ventaja señalada, la gruesa granularidad lleva a un uso ineficiente del espectro, ya que cada demanda es asignada a un canal fijo y estas pueden requerir un ancho de banda menor al tamaño del canal.

Esta desventaja da lugar a las Redes Elásticas Ópticas o \textit{Elastic Optical Networks} (EON) \cite{jinno2009spectrum}, las cuales surgen como una solución al problema anteriormente citado, ya que estas proporcionan una mayor flexibilidad en la división del espectro y de esa forma lograr que los requerimientos sean asignados de manera más eficiente.

A las redes EON también se las conoce como redes de grilla flexible, debido a que las ranuras de frecuencia o FS (\textit{Frequency Slot}) que reemplazan a los ``Canales WDM'', cuentan con una división más flexible \cite{itut_g6941}. Cada FS tiene un ancho de banda de 12.5 GHz, de esta manera se logra una cantidad más apropiada de FS para satisfacer un requerimiento.

Sin embargo, a pesar de las mejoras introducidas por las redes EON, el crecimiento exponencial del tráfico de datos demanda soluciones aún más avanzadas. En este contexto, surgen las Redes Ópticas Elásticas Multicore o \textit{Elastic Optical Networks with Multicore Fibers} (EON-MCF) y por consecuente \textit{Space Division Multiplexing-Elastic Optical Networks} (SDM-EON), que incorporan fibras ópticas multinúcleo (MCF) \cite{takenaga2011large}, para multiplicar la capacidad de transmisión mediante la explotación de la dimensión espacial, además de las dimensiones espectral y temporal ya utilizadas en las redes EON convencionales.

Las fibras multinúcleo contienen múltiples núcleos dentro de una única fibra, donde cada núcleo puede transmitir señales de manera independiente \cite{takenaga2011reduction}. Esta arquitectura permite aumentar significativamente la capacidad de la red sin necesidad de desplegar nuevas fibras, ofreciendo una solución escalable y económicamente viable para satisfacer las crecientes demandas de ancho de banda.

Los métodos de ruteo y asignación del espectro y núcleo tienen gran impacto sobre el uso eficiente de los recursos de la red. Los algoritmos RSCA (\textit{Routing, Spectrum and Core Assignment}) \cite{lei2019crosstalk} se encargan de resolver dicho problema encontrando el camino más apropiado desde el origen hasta el destino, el núcleo a utilizar y las ranuras que utilizará el requerimiento dentro del espectro de los enlaces.

Adicionalmente, en las redes SDM-EON surge un nuevo fenómeno denominado \textit{Crosstalk} o diafonía entre núcleos \textit{inter-core crosstalk, XT}, que ocurre cuando las señales ópticas de núcleos adyacentes interfieren entre sí, degradando la calidad de la transmisión \cite{takenaga2011reduction}. Este fenómeno debe ser considerado como una restricción adicional en los algoritmos RSCA para garantizar la calidad del servicio.

Debido a las restricciones explicadas y a que las asignaciones de recursos son realizadas de manera dinámica, surge el fenómeno denominado "Fragmentación del Ancho de Banda y del Espacio", este problema es una de las principales dificultades de las redes SDM-EON \cite{xiong2019machine} ya que tiene un impacto directo en el uso eficiente del espectro y de los núcleos disponibles.

El fenómeno de la fragmentación espectro-espacial del ancho de banda sucede cuando en los enlaces se encuentran FS disponibles separados por FS que están siendo utilizados por otras conexiones, o cuando existen núcleos con recursos fragmentados que no pueden ser eficientemente asignados, por lo que estas podrían quedar inutilizables para nuevas conexiones por no poder satisfacer a la demanda debido a las restricciones citadas anteriormente, en consecuencia, la probabilidad de bloqueo \cite{shi2013effect} aumenta considerablemente.

Un bloqueo sucede cuando el algoritmo RSCA no puede encontrar núcleos y FS disponibles para una demanda, esto puede deberse a una alta saturación del espectro o de los núcleos, pero también debido al problema mencionado anteriormente, donde existe la cantidad de FS libres que se solicitan, pero sin respetar las restricciones de continuidad y contigüidad, o donde no hay núcleos disponibles que cumplan con las restricciones de crosstalk, es decir el espectro y el espacio se encuentran fragmentados.

El problema de la fragmentación de redes SDM-EON es ampliamente estudiado en la literatura actual \cite{trindade2020machine, xiong2019machine, zhang2020fragmentation}, para buscar manejarlo se han propuesto soluciones con distintos enfoques.

Uno de los enfoques es el llamado \textit{Enfoque proactivo} \cite{chatterjee2017fragmentation} el cual consiste en ejecutar un proceso de desfragmentación periódicamente o mediante un disparador. Tiene como principal objetivo prevenir futuros bloqueos en la red, este enfoque será el utilizado en este trabajo.

El proceso de desfragmentación consiste en la reconfiguración o re-ruteo de un sub-conjunto de conexiones ya establecidas en la red \cite{talebi2014spectrum}, teniendo como principal objetivo reducir la fragmentación del espectro y la fragmentación espacial mediante la eliminación de bloques de FS libres no contiguos y la redistribución eficiente de conexiones entre núcleos.

En el trabajo presentado por Zhang \cite{zhang2014dynamic}, se realizó un análisis del problema de desfragmentación en redes EON, en el cual lo dividen en cuatro subproblemas, los cuales son, (I) ¿Cómo reconfigurar?, (II) ¿Cómo migrar el tráfico?, (III) ¿Cuándo reconfigurar? y (IV) ¿Qué reconfigurar?. Estos subproblemas mantienen su vigencia en el contexto de las redes SDM-EON, con la complejidad adicional de considerar la dimensión espacial.

En este trabajo nos centraremos en el tercer subproblema, ¿Cuándo reconfigurar?, ya que considerando el enfoque proactivo para resolver el problema de la fragmentación \cite{comellas2018periodic}, encontramos que los procesos de desfragmentación podrían ejecutarse en periodos de tiempo donde no son del todo necesarios, es decir cuando la red se encuentra con una baja fragmentación, provocando desfragmentaciones ineficientes, una cantidad mayor de disrupciones de conexiones y una elevación innecesaria del costo de procesamiento.

En los siguientes capítulos presentamos modelos de predicción de índice de fragmentación, entrenados para detectar si supera unos umbrales predefinidos, implementados con técnicas de aprendizaje automático o \textit{Machine Learning} \cite{mitchell1997machine}, los cuales se utilizan como estrategia de disparo del proceso de desfragmentación pero en este caso redes SDM-EON Multinúcleo, proponiendo de esta manera una solución al sub problema planteado anteriormente.

En este contexto, la Inteligencia Artificial (IA) y específicamente el Aprendizaje Automático o \textit{Machine Learning} (ML) \cite{mitchell1997machine}, emergen como herramientas fundamentales para abordar estos desafíos de optimización en redes de telecomunicaciones. El ML se fundamenta en el desarrollo de algoritmos capaces de identificar patrones complejos en conjuntos de datos, mejorando su desempeño de manera iterativa mediante la experiencia acumulada, sin necesidad de instrucciones programáticas explícitas para cada escenario específico.

Dentro del paradigma de aprendizaje supervisado, el algoritmo Gradient Boosting \cite{friedman2001greedy} ha demostrado ser particularmente efectivo para problemas de clasificación y predicción con datos tabulares. Este método construye secuencialmente múltiples modelos predictivos débiles mediante principios de optimización matemática, corrigiendo iterativamente los errores de sus predecesores. Su capacidad para modelar relaciones no lineales complejas, manejar variables de diferentes escalas, y proporcionar estimaciones de importancia de características lo hacen especialmente adecuado para el análisis de métricas operacionales en redes ópticas.

La aplicación de estas técnicas de ML al problema de la desfragmentación en redes SDM-EON permite desarrollar sistemas predictivos que anticipen el estado futuro de fragmentación de la red, considerando múltiples factores como métricas de fragmentación espectral y espacial, utilización de recursos en los núcleos, y patrones temporales de tráfico. Esta capacidad de anticipación posibilita la implementación de estrategias de desfragmentación adaptativas que optimicen el balance entre la reducción de bloqueos y el costo operacional de las reconfiguraciones.

\section{Objetivos del trabajo}
\subsection{Objetivo General}
Diseñar modelos de predicción para el proceso de desfragmentación en redes ópticas elásticas multicore, utilizando técnicas de aprendizaje automático, con el propósito de maximizar la eficiencia en el uso de los recursos de la red mediante la reducción de reconfiguraciones innecesarias y la minimización de la cantidad de bloqueos.

\subsection{Objetivos Específicos}
\begin{itemize}
    \item Realizar una revisión bibliográfica del estado del arte en técnicas de desfragmentación para redes ópticas elásticas, con énfasis en métodos basados en aprendizaje automático y su aplicación en redes multicore.

    \item Identificar y definir métricas de fragmentación apropiadas para redes ópticas elásticas multicore, considerando las particularidades de la asignación de recursos en múltiples núcleos.
    
    \item Desarrollar e implementar un modelo de aprendizaje automático capaz de predecir el índice de la fragmentación de la red a futuro y determinar momentos óptimos para activar el proceso de desfragmentación en redes EON multicore.
    
    \item Diseñar e implementar una interfaz de integración entre el simulador de redes ópticas elásticas multicore y el modelo de aprendizaje automático entrenado, permitiendo la evaluación en tiempo real del sistema propuesto.
    
    \item Evaluar el desempeño del modelo propuesto mediante simulaciones, comparando sus resultados con técnicas de desfragmentación existentes en términos de probabilidad de bloqueo, número de reconfiguraciones y eficiencia en el uso de recursos espectrales.
\end{itemize}



\section{Análisis Bibliográfico}

El problema de la fragmentación en redes ópticas elásticas ha sido ampliamente estudiado en la literatura científica, con diversos enfoques propuestos para su gestión. En el trabajo presentado por Enciso y Silva \cite{enciso2021estrategia}, se aborda específicamente el problema del momento de disparo del proceso de desfragmentación en redes EON utilizando técnicas de aprendizaje automático con redes neuronales artificiales. Su enfoque se centra en la predicción de probabilidades de bloqueo para determinar cuándo ejecutar el proceso de desfragmentación de manera proactiva.

Diversos autores han propuesto diferentes estrategias para el disparo de procesos de desfragmentación. Takita y colaboradores \cite{takita2016wavelength} proponen un mecanismo basado en el valor del \textit{High-slot Mark} (HM), que indica el número máximo de una ranura ocupada en la red. El proceso se dispara cuando el HM supera un umbral predefinido. Por su parte, Fávero y colegas \cite{favero2015new} combinan enfoques reactivos y proactivos, considerando tanto los bloqueos como las conexiones liberadas para determinar el momento de desfragmentación.

Zhang y colaboradores \cite{zhang2013bandwidth} proponen un disparo basado en la cantidad de conexiones liberadas, ejecutando la desfragmentación cuando este número alcanza un umbral específico. En otro trabajo, Zhang \cite{zhang2012priority} utiliza la métrica de \textit{Spectrum Compactness} (SC) como criterio de disparo, comparando su valor actual con un umbral predefinido para determinar la necesidad de desfragmentación.

El enfoque de desfragmentación periódica, ampliamente utilizado en la literatura, ha sido analizado por Comellas, Vicario y Junyent \cite{comellas2018periodic}, quienes evaluaron los efectos de diferentes parámetros de desfragmentación en el rendimiento de la red bajo tráfico dinámico. Sus resultados indican que períodos de desfragmentación muy pequeños pueden generar complejidad excesiva, mientras que períodos muy grandes producen efectos insignificantes.

En el contexto del aprendizaje automático aplicado a redes ópticas, el algoritmo Gradient Boosting \cite{friedman2001greedy} ha demostrado ser particularmente efectivo para problemas de clasificación y predicción. Friedman introduce este método como una técnica que construye secuencialmente modelos predictivos mediante la optimización de funciones de pérdida, corrigiendo iterativamente los errores de modelos anteriores. La implementación práctica de este algoritmo ha sido facilitada por herramientas como Scikit-learn \cite{pedregosa2011scikit}, una biblioteca de código abierto para Python que proporciona implementaciones eficientes de algoritmos de aprendizaje automático.

El estado del arte en desfragmentación de redes ópticas elásticas muestra una tendencia hacia enfoques adaptativos e inteligentes. Chatterjee, Ba y Oki \cite{chatterjee2017fragmentation} realizaron un análisis exhaustivo de los problemas de fragmentación y los enfoques de gestión, clasificándolos en estrategias sin desfragmentación (mediante algoritmos sensibles a la fragmentación), desfragmentación reactiva (ante bloqueos) y desfragmentación proactiva (preventiva).

En el contexto de redes SDM-EON, Trindade y da Fonseca \cite{trindade2020machine} proponen un enfoque de aprendizaje no supervisado para la desfragmentación, identificando grupos de \textit{lightpaths} que pueden ser reordenados eficientemente entre núcleos. Xiong y colaboradores \cite{xiong2019machine} utilizan técnicas de ML para mitigar simultáneamente la fragmentación y el \textit{crosstalk} en redes con multiplexación por división de espacio.

El presente trabajo se posiciona en la intersección de estas líneas de investigación, extendiendo los enfoques de disparo inteligente de desfragmentación mediante ML al contexto más complejo de las redes SDM-EON multinúcleo, donde la dimensión espacial adicional introduce nuevos desafíos en la gestión de recursos y la prevención de la fragmentación tanto espectral como espacial.

\section{Organización del libro}
El presente trabajo se encuentra organizado de la siguiente manera:

En el capítulo dos se presentan las características y conceptos fundamentales de las redes ópticas elásticas multinúcleo (MC-EON), incluyendo la multiplexación por división de espacio (SDM), el fenómeno de diafonía entre núcleos (crosstalk), y el problema de la fragmentación espectro-espacial del ancho de banda. Se describen los diferentes enfoques para la gestión de la fragmentación, con énfasis en la estrategia de desfragmentación proactiva con re-ruteo adoptada en este trabajo.

En el capítulo tres se introduce el aprendizaje automático, enfocándose en los fundamentos del aprendizaje supervisado y el paradigma de aprendizaje por ensamble. Se presenta en detalle el algoritmo Gradient Boosting, describiendo su funcionamiento, configuración de hiperparámetros, y su aplicación específica a problemas de clasificación en redes MC-EON mediante el GradientBoostingClassifier.

En el capítulo cuatro se describe el método propuesto para la selección del momento óptimo de desfragmentación mediante predicción con aprendizaje automático. Se presenta la estrategia adaptativa de tres niveles basada en doble umbral, las características utilizadas como entrada al modelo, el proceso de generación de datos sintéticos con carga variable, el entrenamiento de dos modelos Gradient Boosting independientes, y la arquitectura de servicios REST para integración con el simulador de red.

El capítulo cinco presenta las pruebas experimentales realizadas sobre la topología USNET en tres escenarios de carga (baja, media y alta). Se muestran los resultados comparativos del método propuesto frente a estrategias tradicionales de desfragmentación, evaluando objetivos multiobjetivo mediante métricas de bloqueos, reconfiguraciones, soluciones en el Frente de Pareto y Cobertura de Pareto.

Por último, el capítulo seis presenta las conclusiones del trabajo, validando el cumplimiento de objetivos, destacando las contribuciones principales, identificando limitaciones, y proponiendo direcciones para trabajos futuros en extensiones metodológicas, arquitecturales y validación experimental en entornos reales.