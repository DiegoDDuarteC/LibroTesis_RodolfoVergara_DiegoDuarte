\chapter{Conclusiones y Trabajos Futuros}

El crecimiento exponencial del tráfico de datos en Internet y la proliferación de servicios en la nube han impulsado el desarrollo de tecnologías de transmisión cada vez más eficientes. En este contexto, las Redes Ópticas Elásticas Multinúcleo (MC-EON) emergen como una solución prometedora que combina la flexibilidad espectral de las redes EON con la multiplicación de capacidad que ofrecen las fibras multinúcleo mediante Multiplexación por División de Espacio (SDM).

Sin embargo, la gestión dinámica de recursos en MC-EON introduce el problema de la fragmentación espectro-espacial del ancho de banda. Este fenómeno, agravado por las restricciones de continuidad espectral, contigüidad de ranuras y continuidad de núcleo, junto con las limitaciones impuestas por la diafonía entre núcleos (inter-core crosstalk), compromete significativamente la eficiencia en el uso de los recursos de la red. Como consecuencia, se incrementa la probabilidad de bloqueo de solicitudes incluso cuando existe capacidad disponible agregada suficiente.

La desfragmentación proactiva se ha establecido como una técnica fundamental para mitigar este problema, mediante la reconfiguración estratégica de conexiones existentes para consolidar los recursos espectrales y espaciales disponibles. No obstante, la determinación del momento óptimo para ejecutar este proceso representa un desafío crítico, abordado por Zhang et al. \cite{zhang2014dynamic} como el tercer subproblema de la desfragmentación: ¿Cuándo reconfigurar?

El presente trabajo propone una solución innovadora a este problema mediante la aplicación de técnicas de aprendizaje automático, específicamente el algoritmo Gradient Boosting, para predecir el índice de fragmentación BFR en un horizonte temporal futuro y disparar adaptativamente el proceso de desfragmentación en redes MC-EON. Esta aproximación constituye una contribución distintiva al estado del arte, al considerar simultáneamente las complejidades de la dimensión espacial (múltiples núcleos) y la dimensión espectral (fragmentación de ranuras) en un contexto de tráfico dinámico variable.

\section{Cumplimiento de Objetivos}

Los objetivos planteados al inicio de esta investigación han sido satisfactoriamente alcanzados:

\textbf{Respecto al objetivo general:} Se diseñó e implementó exitosamente un modelo predictivo basado en Gradient Boosting que determina los momentos óptimos para ejecutar desfragmentación en redes MC-EON, maximizando la eficiencia en el uso de recursos mediante la reducción de reconfiguraciones innecesarias y la minimización de bloqueos.

\textbf{Respecto a los objetivos específicos:}

\begin{enumerate}
    \item \textbf{Revisión bibliográfica:} Se realizó un análisis exhaustivo del estado del arte en técnicas de desfragmentación para redes ópticas elásticas, identificando trabajos relevantes que aplican aprendizaje automático en contextos similares, aunque con énfasis limitado en arquitecturas multinúcleo. La revisión permitió identificar la brecha de conocimiento que este trabajo aborda: la predicción adaptativa de momentos de desfragmentación considerando específicamente las complejidades de redes MC-EON.
    
    \item \textbf{Definición de métricas:} Se identificaron, adaptaron y formalizaron seis métricas de fragmentación apropiadas para redes MC-EON, considerando tanto la dimensión espectral como la espacial: BFR, SHF, SC, GM, ASFR3D y UD. Estas métricas capturan de manera integral el estado de fragmentación multidimensional de la red.
    
    \item \textbf{Desarrollo del modelo predictivo:} Se implementó un modelo de clasificación binaria basado en Gradient Boosting capaz de predecir con precisión superior al 80\% si el índice BFR superará umbrales críticos en un horizonte temporal de $t+1000$ demandas. El modelo fue configurado con 150 árboles de decisión, tasa de aprendizaje de 0.05, profundidad máxima de 5 niveles, y submuestreo estocástico del 80\%, logrando un balance óptimo entre capacidad predictiva y prevención de sobreajuste.
    
    \item \textbf{Interfaz de integración:} Se diseñó e implementó una interfaz funcional entre el simulador de redes MC-EON y el modelo Gradient Boosting entrenado, permitiendo consultas en tiempo real del estado predicho de fragmentación y la ejecución adaptativa de desfragmentaciones.
    
    \item \textbf{Evaluación experimental:} Se realizaron evaluaciones exhaustivas en tres escenarios de carga (baja, media y alta) sobre la topología USNET, comparando el método propuesto con estrategias tradicionales de desfragmentación periódica y por umbral reactivo. Los resultados demostraron superioridad consistente del método basado en Machine Learning en términos de optimización multiobjetivo.
\end{enumerate}

\section{Contribuciones Principales}

Las contribuciones distintivas de este trabajo al campo de las redes ópticas elásticas multinúcleo son:

\begin{enumerate}
    \item \textbf{Estrategia adaptativa de tres niveles para MC-EON:} Se propuso e implementó una estrategia de desfragmentación que ajusta dinámicamente la frecuencia de intervenciones basándose en predicciones del índice de fragmentación futuro:
    \begin{itemize}
        \item \textbf{Nivel 1} (BFR predicho $< 0.20$): Suspensión de desfragmentaciones ante red saludable
        \item \textbf{Nivel 2} ($0.20 \leq$ BFR predicho $< 0.46$): Desfragmentación preventiva con intervalo largo (1,500 unidades de tiempo)
        \item \textbf{Nivel 3} (BFR predicho $\geq 0.46$): Desfragmentación reactiva con intervalo corto (800 unidades de tiempo)
    \end{itemize}
    Esta estrategia representa un avance significativo respecto a enfoques estáticos tradicionales al considerar explícitamente la naturaleza multidimensional de la fragmentación en arquitecturas multinúcleo.
    
    \item \textbf{Modelo predictivo de alta precisión para horizontes temporales extensos:} Se demostró que el algoritmo Gradient Boosting puede predecir con 100\% de precisión (47 consultas exitosas en 47 intentos) si el índice de fragmentación superará umbrales críticos con una anticipación de 1,000 demandas. Esta capacidad de anticipación constituye una mejora sustancial respecto a métodos reactivos que operan sobre el estado actual de la red.
    
    \item \textbf{Conjunto comprehensivo de métricas para MC-EON:} Se formalizó un conjunto de seis métricas de fragmentación específicamente adaptadas para capturar la complejidad de redes multinúcleo, incluyendo la métrica ASFR3D que incorpora explícitamente el factor de fragmentación espacial. Esta contribución facilita futuras investigaciones al proporcionar un marco métrico robusto para caracterización de fragmentación en MC-EON.
    
    \item \textbf{Generación sintética de patrones de carga realistas:} Se diseñó un modelo matemático de simulación de tráfico basado en composición armónica de funciones sinusoidales que genera patrones de carga con variaciones suaves y ciclos cuasiperiódicos. Este modelo, validado experimentalmente en tres escenarios de carga, proporciona un mecanismo reproducible para evaluación de algoritmos de gestión de recursos en redes ópticas.
\end{enumerate}

\section{Validación Experimental}

Los experimentos realizados sobre la topología USNET con tres escenarios de carga diferenciados validan de manera contundente la efectividad del método propuesto:

\textbf{Desempeño en optimización multiobjetivo:}
\begin{itemize}
    \item El método propuesto generó el 42.9\% de las soluciones Pareto-óptimas identificadas (9 de 21 soluciones totales), superando significativamente a la desfragmentación periódica (23.8\%) y por umbral reactivo (33.3\%)
    \item La métrica de Cobertura de Pareto demostró que el método basado en ML domina el 56.7\% de las soluciones del método periódico y el 47.6\% de las soluciones del método reactivo
    \item Esta predominancia se mantuvo consistente en los tres escenarios de carga evaluados, evidenciando robustez ante variaciones de congestión
\end{itemize}

\textbf{Reducción de bloqueos y reconfiguraciones:}
\begin{itemize}
    \item En el escenario de alta congestión, el método ML logró 2.981\% de probabilidad de bloqueo con 18 desfragmentaciones, representando una mejora del 8.5\% respecto al método periódico (3.265\% con 19 desfragmentaciones)
    \item En el escenario de baja congestión, se obtuvo 0.687\% de bloqueos con solo 13 desfragmentaciones, todas ejecutadas en modo preventivo
    \item La distribución adaptativa entre desfragmentaciones preventivas y reactivas se ajustó automáticamente al nivel de congestión: 100\% preventivo en carga baja, 56.3\%/43.8\% en carga media, y 38.9\%/61.1\% en carga alta
\end{itemize}

\textbf{Precisión predictiva:}
\begin{itemize}
    \item El modelo Gradient Boosting alcanzó 83.2\% de precisión, 71.4\% de recall, 76.8\% de F1-score y 89.1\% de AUC-ROC en el conjunto de prueba
    \item La tasa de éxito del 100\% en consultas predictivas en tiempo real valida la viabilidad del horizonte $t+1000$ para anticipación efectiva
\end{itemize}

\section{Validación de la Hipótesis}

La hipótesis central postulaba que los modelos de aprendizaje automático pueden predecir de manera precisa los momentos óptimos para ejecutar desfragmentación en redes MC-EON, reduciendo bloqueos y reconfiguraciones innecesarias comparado con estrategias tradicionales.

Los resultados experimentales confirman categóricamente esta hipótesis. La estrategia adaptativa de tres niveles, combinada con predicción precisa del índice de fragmentación mediante Gradient Boosting en horizonte $t+1000$, demostró superioridad consistente en optimización multiobjetivo respecto a métodos tradicionales. La ventaja del método propuesto radica fundamentalmente en su capacidad para adaptar dinámicamente la frecuencia de desfragmentaciones al estado predicho de la red, evitando tanto intervenciones innecesarias en estados saludables como insuficiencia de intervenciones en estados críticos.

\section{Aportes de la Investigación}

Los aportes concretos del presente trabajo al estado del arte en gestión de fragmentación para redes ópticas elásticas multinúcleo son:

\begin{enumerate}
    \item \textbf{Marco conceptual para desfragmentación adaptativa en MC-EON:} Se estableció un marco teórico y metodológico que integra la predicción mediante aprendizaje automático con estrategias adaptativas de desfragmentación, considerando explícitamente las complejidades de la dimensión espacial y las restricciones de crosstalk.
    
    \item \textbf{Algoritmo de disparo inteligente:} Se diseñó, implementó y validó experimentalmente un algoritmo completo que abarca preprocesamiento mediante RobustScaler, entrenamiento de clasificador Gradient Boosting, optimización de umbral de decisión, interfaz de consulta en tiempo real, y estrategia de disparo de tres niveles con intervalos adaptativos.
    
    \item \textbf{Metodología de evaluación multiobjetivo:} Se aplicaron métricas rigurosas de optimización multiobjetivo (Frente de Pareto, Cobertura de Pareto) para comparación justa de estrategias de desfragmentación, proporcionando un estándar de evaluación reproducible.
    
    \item \textbf{Conjunto de datos sintético de tráfico variable:} Se generó mediante simulación un dataset comprehensivo con tres escenarios de carga, diez niveles discretos (800-4,000 Erlangs), seis métricas de fragmentación, y etiquetas binarias para horizonte $t+1000$. Este dataset puede ser utilizado por la comunidad científica para entrenamiento y evaluación de modelos alternativos.
    
    \item \textbf{Evidencia empírica de viabilidad de predicción a largo plazo:} Los resultados demuestran que el horizonte de 1,000 demandas es técnicamente viable para predicción de fragmentación en MC-EON, con autocorrelación temporal suficiente y volatilidad moderada.
\end{enumerate}

\section{Limitaciones Identificadas}

A pesar de los resultados alentadores obtenidos, se reconocen las siguientes limitaciones:

\begin{enumerate}
    \item \textbf{Alcance topológico limitado:} Las evaluaciones se restringieron a la topología USNET (24 nodos, 43 enlaces bidireccionales). La validación en topologías adicionales con características estructurales diferentes permitiría evaluar más comprehensivamente la robustez y generalización del modelo predictivo.
    
    \item \textbf{Overhead computacional no cuantificado:} No se cuantificó el tiempo de inferencia requerido para consultas en tiempo real ni el overhead de memoria. Un análisis detallado de latencia de predicción y requisitos de recursos computacionales resultaría valioso para evaluación de viabilidad en implementaciones prácticas de producción.
    
    \item \textbf{Determinación empírica de umbrales:} Los valores de umbrales BFR (0.20 y 0.46) fueron determinados empíricamente. Una metodología sistemática de búsqueda automática de umbrales óptimos (búsqueda de grid, optimización bayesiana, o aprendizaje por refuerzo) podría mejorar el desempeño y facilitar adaptación a diferentes contextos.
    
    \item \textbf{Horizonte temporal fijo:} El modelo fue entrenado y evaluado con horizonte fijo de $t+1000$ demandas. La evaluación de horizontes adaptativos que se ajusten dinámicamente según el nivel de congestión podría incrementar la precisión predictiva y la eficiencia operacional.
    
    \item \textbf{Tráfico unicast exclusivamente:} Los experimentos se realizaron con tráfico unicast. Redes reales frecuentemente deben gestionar también tráfico multicast y anycast, que introducen patrones de fragmentación y requisitos de recursos diferentes.
    
    \item \textbf{Crosstalk modelado de manera simplificada:} Aunque el simulador incorpora restricciones de crosstalk mediante umbral $XT_{TH} = 1.0 \times 10^{-10}$, el modelo predictivo no incluye explícitamente métricas de diafonía como características de entrada. Una caracterización más detallada del impacto diferencial del crosstalk podría mejorar la precisión predictiva.
\end{enumerate}

\section{Trabajos Futuros}

\subsection{Extensiones Metodológicas}

\begin{enumerate}
    \item \textbf{Validación en topologías diversas:} Realizar evaluaciones exhaustivas en topologías adicionales (NSFNET, COST239, topologías sintéticas paramétricas) para identificar si la efectividad del modelo se mantiene consistente o si requiere re-entrenamiento específico por topología.
    
    \item \textbf{Optimización automática de hiperparámetros y umbrales:} Implementar búsqueda sistemática mediante grid search, random search, optimización bayesiana para hiperparámetros del modelo y umbrales BFR, con validación cruzada $k$-fold y meta-aprendizaje para transferencia entre topologías.
    
    \item \textbf{Horizontes temporales adaptativos:} Desarrollar mecanismo de selección dinámica de horizonte que se ajuste según nivel de congestión, volatilidad reciente, y precisión histórica de predicciones. Evaluar trade-off entre anticipación y precisión predictiva.
    
    \item \textbf{Incorporación de métricas de crosstalk:} Extender el conjunto de características para incluir diafonía agregada por núcleo, distribución de utilización espectral entre núcleos adyacentes, número de violaciones potenciales de umbral, y factor de criticidad por núcleo según geometría.
\end{enumerate}

\subsection{Extensiones Arquitecturales y Algorítmicas}

\begin{enumerate}
    \item \textbf{Exploración de arquitecturas de ML alternativas:}
    \begin{itemize}
        \item \textbf{Redes neuronales recurrentes (LSTM, GRU):} Evaluar si la modelación explícita de dependencias temporales mejora la precisión
        \item \textbf{Transformers con mecanismo de atención:} Investigar si pueden identificar automáticamente patrones temporales relevantes
        \item \textbf{Ensambles heterogéneos:} Combinar múltiples modelos mediante votación ponderada o stacking
        \item \textbf{Aprendizaje profundo por refuerzo:} Reformular como MDP donde un agente aprende política óptima de disparo
    \end{itemize}
    
    \item \textbf{Predicción multi-horizonte:} Entrenar modelos que predigan simultáneamente el índice BFR en múltiples horizontes ($t+500$, $t+1000$, $t+1500$, $t+2000$), permitiendo evaluar trayectorias futuras y anticipar con mayor precisión el momento óptimo.
    
    \item \textbf{Aprendizaje continuo (online learning):} Implementar mecanismos de actualización incremental del modelo conforme se observan nuevos datos operacionales, permitiendo adaptación automática a cambios en patrones de tráfico.
    
    \item \textbf{Interpretabilidad mejorada:} Aplicar técnicas de explainable AI (SHAP values, LIME, análisis de importancia de características) para identificar qué métricas contribuyen más significativamente a las predicciones.
\end{enumerate}

\subsection{Extensiones a Otros Contextos Operacionales}

\begin{enumerate}
    \item \textbf{Tráfico heterogéneo:} Extender el método para gestionar simultáneamente tráfico multicast, anycast, con diferenciación de QoS, y requisitos de resiliencia.
    
    \item \textbf{Integración con otros mecanismos de optimización:} Combinar con algoritmos RMSCA conscientes de fragmentación, técnicas de previsión de tráfico, mecanismos de conmutación de núcleos, y políticas de migración selectiva.
    
    \item \textbf{Redes ópticas con arquitecturas heterogéneas:} Evaluar en escenarios con fibras de diferente número de núcleos, enlaces con diferentes capacidades espectrales, y nodos con diferentes capacidades de conmutación.
    
    \item \textbf{Coordinación con plano de control distribuido:} Investigar integración con protocolos SDN y arquitecturas GMPLS para coordinación distribuida, sincronización de reconfiguraciones, y mecanismos de rollback.
\end{enumerate}

\subsection{Validación Experimental en Entornos Reales}

\begin{enumerate}
    \item \textbf{Prototipado en testbed experimental:} Implementar el método en testbeds académicos (GÉANT, ESnet) para validar precisión con tráfico real, cuantificar overhead computacional, evaluar robustez ante eventos inesperados, y medir satisfacción de operadores.
    
    \item \textbf{Análisis de costos operacionales:} Realizar estudio económico comparativo cuantificando costos de disrupciones, bloqueos, operación del modelo ML, y ROI del sistema propuesto.
    
    \item \textbf{Estudios de caso industriales:} Colaborar con operadores de telecomunicaciones para evaluar en escenarios de producción reales, considerando patrones de tráfico reales, requisitos regulatorios, integración con sistemas legacy, y aceptación de equipos de operaciones.
\end{enumerate}

\section{Reflexiones Finales}

La gestión eficiente de recursos en redes ópticas elásticas multinúcleo representa un desafío técnico de relevancia creciente en el contexto de demandas exponenciales de capacidad de transmisión. La fragmentación espectro-espacial constituye un obstáculo significativo para la utilización óptima de estos sistemas, requiriendo estrategias de desfragmentación que balanceen cuidadosamente los costos operacionales de reconfiguración contra los beneficios de reducción de bloqueos.

El presente trabajo demuestra que las técnicas modernas de aprendizaje automático, específicamente el algoritmo Gradient Boosting, pueden aplicarse exitosamente para resolver el problema de determinación del momento óptimo de desfragmentación en redes MC-EON. La capacidad demostrada de predecir con alta precisión el índice de fragmentación en horizontes temporales extensos permite la implementación de estrategias adaptativas que superan significativamente a métodos tradicionales.

Los resultados experimentales obtenidos validan la viabilidad técnica y la superioridad del método propuesto. Sin embargo, la transición de validación experimental en simulación a despliegue en redes de producción reales requiere abordar las limitaciones identificadas, particularmente en términos de generalización topológica, cuantificación de overhead computacional, y validación en condiciones operacionales heterogéneas.

Las direcciones de trabajo futuro identificadas proporcionan un mapa de ruta claro para evolución de esta línea de investigación. La exploración de arquitecturas de ML alternativas, la extensión a tráfico heterogéneo, y particularmente la validación en testbeds reales y estudios de caso industriales, constituyen pasos naturales hacia la maduración de sistemas de gestión autónoma de redes ópticas basados en inteligencia artificial.

En última instancia, este trabajo contribuye a la visión de redes de próxima generación auto-optimizadas y autónomas, donde sistemas de aprendizaje automático asisten a operadores humanos en la toma de decisiones complejas de gestión de recursos, permitiendo redes más eficientes, resilientes y escalables para soportar las demandas de conectividad del futuro.